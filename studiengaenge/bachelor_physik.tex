
\section{Bachelor Physik}

\subsection{Erstes Semester}

Das erste Semester ist, verglichen mit späteren Semestern, sehr stark durchstrukturiert, was euch den Wechsel von der Schule an die Uni erleichtert, jedoch auch nicht besonders viele Freiheiten lässt.  Allgemein werdet ihr bis ins vierte Semester im Pflichtprogramm je eine Experimentalphysik und eine Theoretische Physik Vorlesung besuchen, wobei im fünften Semester noch die Experimentalphysik V folgt. Zusätzlich dazu hört ihr an Mathe im ersten Semester die Lineare Algebra I und wenn ihr wollt auch die Analysis I; aber zur Mathe später mehr.

Das ganze Semester über werdet ihr jede Woche in jeder Vorlesung, die ihr hört, einen Übungszettel bekommen, den ihr dann auf die nächste Woche lösen sollt. Diese dienen zum Einüben des in den Vorlesungen behandelten Stoffes und zugleich häufig als Klausurzulassung. Man benötigt meistens insgesamt 50 - 60 \% der Gesamtpunktzahl auf allen Zetteln, aber wenn es nur um ein paar Punkte geht, sind die Tutoren meistens kulant. Wenn euer Tutor den Eindruck hat, ihr verschwendet nicht nur einen Prüfungsversuch mit der Klausur, lässt er euch meistens zu. Macht euch darum aber erstmal keinen Kopf, wenn man am Ball bleibt, ist das überhaupt kein Problem, ihr müsst sie ja nicht alle alleine lösen.

\subsubsection{Orientierungsprüfung}

Die Orientierungsprüfung ist eine Prüfung, die ihr bis zum Ende des 3. Semesters bestehen müsst, um weiterhin Physik studieren zu dürfen. Im Bachelor Physik ist dies die ganz normale Klausur in der Experimentalphysik I. Dort werden meist sehr schulnahe physikalische Grundlagen der Mechanik behandelt. Macht euch also keine Sorgen. Das schafft ihr!

\subsubsection{Basiskurs}

Der Basiskurs beginnt, wie ihr wahrscheinlich schon wisst, in der letzten Vorkurswoche und soll euch den Einstieg in das Studium erleichtern. Dabei werden euch sogenannte Schlüsselkompetenzen beigebracht, die euch unter anderem in Zeitmanagement, Literatursuche und das Textsatzsytem \LaTeX{} einführen. Auch wenn ihr vermutlich schon einiges davon kennt, ist es doch ganz nett, nochmal alles kompakt zu sehen und vor allem dabei neue Menschen und vielleicht auch spätere „Zettelpartner“ kennenlernen zu können. Leider ist die Qualität des Kurses oft stark abhängig von eurem Tutor. Ihr müsst also für euch bestimmen, wie viel ihr aus diesem Kurs mitnehmt. Bedenkt, der Tutor bekommt Geld für diesen Kurs, ihr dürft also auch etwas von ihm erwarten. %(Stimmen die Inhalte des Basiskurses? Ja, die stimmen.)
\subsubsection{Analysis vs. Höhere Mathematik für Physiker}
\subsection{Höhere Mathematik für Physiker}
\label{mathephysik}

Die Mathematikausbildung für Physikerinnen sieht vor, dass ihr die Lineare Algebra 1 (\gls{LA}) im ersten Semester hört. Laut Modulhandbuch könnt ihr euch dann vor dem zweiten Semester entscheiden, ob ihr mit Analysis 2 und 3 (\gls{Ana}, was auch die Mathematikerinnen hören) oder mit Höhere Mathematik für Physiker (\gls{HoMa}) 2 und 3 weitermacht.

\subsubsection{Was spricht für HöMa?}
Die Vorlesung ist extra auf euch als Physikerinnen zugeschnitten und legt ihren Schwerpunkt auf die Vorlesungen Ana 1-3 in strafferer Form. Während Mathematikvorlesungen einer gewissen Freiheit unterliegen und es durchaus vorkommen kann, dass die Dozentin einen Schwerpunkt auf ihr Forschungsgebiet legt, hört ihr in HöMa größtenteils nur jene Dinge, die in der Physik auch verwendet werden. Außerdem kommen Beispiele gerne aus der Physik und liegen euch deshalb vielleicht näher. Trotzdem handelt es sich nicht um eine Schmalspurversion, sondern um eine vollwertige Mathematikvorlesung, die auch für zukünftige Theoretikerinnen nicht ungeeignet ist.

\subsubsection{Was spricht für die Analysis (Ana)?}
Die Analysis bietet als eine für Mathematikerinnen konzipierte Veranstaltung eine präzisere Formulierung der Definitionen, Sätze und insbesondere Beweise. Somit wird es möglich die mathematischen Hintergründe in der Physik besser zu durchdringen und weitergehende Verbindungen der Gebiete zu erkennen. Dieses tiefere Verständnis kann unter anderem in der theoretischen Physik oder auch in weiterführenden Matheveranstaltungen von Vorteil sein und lässt euch insgesamt mehr Freiheiten im weiteren Studienverlauf, besonders bezüglich der Mathematik. Zudem ist je nach Dozentin und Forschungsbereich auch eine gewisse Schwerpunktlegung (vor allem in der Ana 3) möglich.\\

Auf den ersten Blick mag es verwundern, dass ihr in die Ana 2 einsteigen sollt, ohne die erste Vorlesung dazu gehört zu haben. Dies ist theoretisch zumindest möglich, jedoch vermutlich mit ein wenig Mehraufwand verbunden. Trotzdem können mathematisch Ambitionierte natürlich auch die Ana 1 im ersten Semester hören, da diese eine schöne Einführung in den Themenbereich Analysis und die damit verbundenen Methoden darstellt. Das kann euch vor allem in weiterführenden Vorlesungen weiterhelfen; auch wird es eine große Erleichterung für die Ana 2 sein, wenn ihr schon ein wenig mehr mit der Materie und der Dozentin vertraut seit. Andererseits werdet ihr mit dem Kursprogramm auch so schon stark ausgelastet sein. Wenn ihr es mit vier Vorlesungen versuchen wollt, solltet ihr euch nach zwei bis drei Wochen entschieden haben, ob ihr das im ersten Semester durchhaltet oder nicht, da es für den Übungsbetrieb ziemlich blöd ist, wenn in der Mitte des Semesters viele Leute aussteigen.


\subsection{Weiteres Studium}

Euer weiteres Studium sieht, wie oben kurz angedeutet, so aus, dass ihr immer ein Grundgerüst an Vorlesungen habt und euch darum herum andere Vorlesungen und Seminare selbst auswählen könnt. Die Experimentalphysikvorlesungen sind bis zum fünften Semester und die theoretischen bis zum vierten Semester Pflicht. Hinzu kommen im zweiten und dritten Semester entweder die Höhere Mathematik für Physiker (HöMa) II und III, oder die Analysis II und III. Ansonsten seid ihr jedoch bis auf ein Pflichtseminar, das ihr aus einem relativ großen Topf an Seminaren auswählen könnt und den Pflichtpraktika frei alles zu hören, was euch so in den Sinn kommt. Ihr müsst einzig darauf achten, dass ihr in den einzelnen Bereichen ausreichend „Punkte sammelt“: Im Wahlpflichtbereich sind das 14 \gls{LP}, bei den Übergreifenden Kompetenzen 19 \gls{LP} und im Wahlbereich bis zu 17 \gls{LP}. Nutzt frühzeitig die Gelegenheit Veranstaltungen zu besuchen, die euch interessieren, dann macht das Studium gleich doppelt so viel Spaß!

\subsection{Wahlbereich, Wahlpflichtbereich und Übergreifende Kompetenzen}

Im Laufe Eures Bachelorstudiums müsst ihr ggf. ein oder zwei Wahlfächer belegen, welche ihr aus einem recht weit gefächerten Angebot wählen könnt. Diese Wahlfächer müssen auch nicht zwingendermaßen aus Bereichen der Physik kommen, sondern können z.B. Mathe, Chemie oder Philosophie sein. Was ihr alles für Möglichkeiten habt, könnt ihr genauer in der Prüfungsordnung nachlesen und selbst Fächer, die dort nicht aufgezählt sind, lassen sich möglicherweise nach Absprache mit dem Prüfungsausschuss auch anrechnen lassen.\\

Der Wahlpflichtbereich besteht, im Gegensatz zum Wahlbereich und den Übergreifenden Kompetenzen, aus vertiefenden oder weiterführenden Physikveranstaltungen. Schaut doch einfach in das Vorlesungsverzeichnis im LSF und sucht euch interessante Vorlesungen oder auch Seminare aus. Besonders Seminare können viel Spaß machen, da diese zwar oftmals viel selbstständiges Arbeiten verlangen, aber oft auch forschungsnäher sind als Vorlesungen. Zudem findet ihr auch Anregungen dazu in der Prüfungsordnung und im Modulhandbuch.\\

Der Bereich Übergreifende Kompetenzen soll euch ein wenig dazu bewegen fachunabhängige Kompetenzen zu erlernen. Darunter fallen der mathematische Vorkurs, der Basiskurs, sowie alle als „Überfachliche Kompetenzen“ gekennzeichneten Module der Mathematik, Informatik und den Naturwissenschaften. Außerdem lassen sich oft nach Rücksprache mit dem Prüfungssekretariat auch weitere Veranstaltungen anrechnen lassen, wie zum Beispiel Sprachkurse, Programmierkurse, usw.\\

Allgemein gilt: Versucht, möglichst früh Dinge aus Gebieten, die euch wirklich interessieren, zu hören; denn das sind die Fächer die euch auch wirklich Spaß machen und ihr erlangt ein möglichst breites, und vor allem tiefes Wissen, welches euch bei eurer Bachelorarbeit und vermutlich auch sonst zugutekommt.

\subsection{Prüfungen und Noten}

Es wird euch sicher freuen zu hören, dass eine nicht bestandene Klausur nicht gleich das Ende für euer Studium bedeutet. Im Grunde ist die Wiederholungsregelung sogar recht studifreundlich; so habt ihr in jedem Modul zwei Versuche, wobei in der Regel ein Versuch aus einer Klausur und, wenn nötig, der dazugehörenden Nachklausur besteht. Darüber hinaus habt ihr für euer Studium noch zwei sogenannte Joker, die euch jeweils einen dritten Versuch geben, falls die zwei regulären nicht reichen sollteni (dies gilt nicht für eure Orientierungsprüfung, welches die Prüfung in der Experimentalphysik I darstellt). Darüber hinaus ist für euch recht interressant, dass es die Möglichkeit gibt, zwei Noten aus unterschiedlichen Bereichen der tendenziell schlechteren Pflichtmodule zu streichen. Macht euch also nicht zu viele Gedanken darüber, wenn eure Noten zunächst nicht ganz so gut sind. Auch könnt ihr jederzeit zusätzlich gehörte Module in den Zusatzqualifikationenbereich verschieben, wenn ihr diese nicht benötigt, euch entsteht also kein Nachteil dadurch, dass ihr mehr hört, als ihr müsst.\\

Sonst ist noch zu erwähnen, dass ihr, um in Heidelberg in den Master zugelassen zu werden, eine Bachelornote von 2,9 oder besser braucht. Das war bisher, so wurde uns versichert, aber noch nie ein Problem, für Heidelberger Studis, der Notenschnitt lag im Bachelor bei etwa 1,7.

\subsection{Prüfungsordung und Modulhandbuch}

Immer dann, wenn sich euch Fragen zu eurem Studium stellen oder ihr euch einfach über den weiteren Studienverlauf informieren wollt, sind die ersten Anlaufstellen die Prüfungsordung und das Modulhandbuch. In der Prüfungsordung ist formal geregelt, wie der Studienablauf, die Prüfungen und die Anrechenbarkeit von Modulen aussehen. Das Modulhandbuch wiederum ist im Großen und Ganzen eine Auflistung der in der Physik (und benachbarten Gebieten) angebotenen Veranstaltungen. Zurzeit findet ihr beide Dokumente auf der Hauptseite des Bachelors Physik. Leider ist das Modulhandbuch oft nicht ganz aktuell, falls euch Verbesserungen auffallen, meldet euch einfach bei uns.\\

Ansonsten könnt ihr auch gerne zu uns in die Fachschaft auf eine Tasse Kaffee oder Tee vorbeikommen. In vielen Fällen können wir euch auch weiterhelfen.
