\section{50\%-Bachelor Mathematik (Lehramt)}

Das Studium beginnt mit einem Zweifach-Bachelor in Mathematik und im anderen
gewählten Fach. Der Schwerpunkt liegt in den ersten sechs Semestern also auf
dem fachlichen Studium. In der Mathematik startet man mit den Grundvorlesungen
Ana I und II und LA I und II; jede dieser vier Vorlesungen gibt jeweils 8~LP.

Je nach der Wahl des zweiten Fachs können die ersten beiden Semester damit sehr
voll sein. Viele Lehramtsstudierende konzentrieren sich deshalb entweder auf
Mathematik und hören nur wenige Vorlesungen aus dem zweiten Fach oder
entscheiden sich für eines der beiden Module, also entweder den Analysis oder
Lineare Algebra-Zyklus. Wichtig ist allerdings, dass Lineare Algebra I die
sogenannte Orientierungsprüfung ist, die ihr bis zum Ende des dritten Semester
bestanden haben müsst, sonst dürft ihr nicht weiterstudieren.

Im Zweifach-Bachelor gibt es da leider noch keine Erfahrungen. Solltet ihr
aber feststellen, dass euch vier Vorlesungen im ersten Semester zu viel sind,
dann ist das kein Weltuntergang. Ihr solltet euch dann entweder ganz auf
Mathematik oder nur auf eines der beiden Module konzentrieren. Das kann
natürlich dazu führen, dass sich euer Studium um ein oder zwei Semester
verlängert. Aber auch in anderen Bachelor-Studiengängen, zum Beispiel im
Mathematik Bachelor 100\%, ist die durchschnittliche Studienzeit höher als die
Regelstudienzeit.  Wenn ihr länger als die Regelstudienzeit studiert, kann es
sein, dass ihr für die zusätzlichen Semester kein BAföG mehr bekommt. Aber auch
da gibt es Ausnahmen.  Ihr solltet euch, falls ihr Vorlesungen in spätere
Semester schiebt, frühzeitig bei den zuständigen Stellen informieren.

Nachdem ihr die Grundvorlesungen absolviert habt, könnt ihr euch aussuchen, in
welcher Reihenfolge ihr die Vorlesungen aus dem Wahlpflichtbereich hören
wollt. Das sind
\begin{itemize}
  \item Algebra 1,
  \item Funktionentheorie 1,
  \item Einführung in die Wahrscheinlichkeitstheorie und Statistik und
  \item Einführung in die Numerik,
\end{itemize}
alle wieder jeweils zu 8~CP. Im Bachelor müsst ihr mindestens drei der vier zur
Wahl stehenden Vorlesungen hören. Eine vierte Vorlesung  dürft ihr frei aus dem
Angebot der Fakultät wählen. Außerdem müsst ihr im Bachelor noch jeweils ein
Proseminar und ein Seminar machen; die bringen jeweils 6~CP.

Dabei müsst ihr beachten, dass ihr am Ende des Master of Education alle vier
Vorlesungen bestanden haben müsst. Dazu kommt noch Einführung in die Geometrie,
die ihr nur im Master hören könnt. Ihr könnt aber eine der Vorlesungen erst im
Master hören. Wenn ihr obige vier Vorlesungen schon im Bachelor macht, dann
könnt ihr im Master eine Vorlesung frei aus dem Angebot der Fakultät auswählen.
Ihr entscheidet also, ob ihr eine frei gewählte Veranstaltung im Bachelor oder
erst im Master belegt.  Erwähnenswert ist, dass diese frei gewählte
Veranstaltung auch ein weiteres Seminar sein kann. Da diese im Allgemeinen nur
mit 6 LP eingehen, kann es sein, dass der Dozent beispielsweise eine
schriftliche Ausarbeitung oder ähnliches als „Extra“-Leistung für die
zusätzlichen 2 LP fordert. Hier solltet ihr einfach die DozentInnen frühzeitig
ansprechen.

Als fachübergreifenden Kompetenzen (FÜK) müsst ihr bildungswissenschaftliche
Veranstaltungen belegen, da diese Zulassungsvoraussetzung für den Master of
Education sind. Welche genau das sind findet ihr im Artikel über das allgemeine
Lehramt. Damit bleiben euch keine Leistungspunkte im Bereich der FÜK mehr übrig.
Beachtet, dass pro Fach bis zu 2 LP der insgesamt 20 FÜK-LP bereits in den
Pflichtvorlesungen euer beiden Studienfächer integriert sind. In Proseminar und
Seminar im Mathematikanteil sind bereits 2 LP für Fachdidaktik integriert.
Diese zählen sozusagen bildungswissenschaftlich.

Als letztes bleibt noch die Entscheidung zu fällen, in welchem Fach ihr eure
Bachelor-Arbeit schreiben wollt. Diese Entscheidung bestimmt, welches euer
erstes Hauptfach ist und ob ihr dann mit einem Bachelor of Science oder einen
Bachelor of Arts abschließt.
In der Mathematik ist es per se leider nur mit bestimmten Fachkombinationen
möglich, überhaupt eine Bachelorarbeit in Mathe zu schreiben.  Für andere
Kombinationen müsst ihr bei der Fakultät einen Antrag stellen, eine Ausnahme
sollte aber bei entsprechender Motivation gut machbar sein.

Solltet ihr nach dem Doppelbachelor doch einen Fach-Master machen wollen, ist in
den meisten Fächern dann auch die Bachelorarbeit in diesem Fach
Zulassungsvoraussetzung für den Fach-Master.

Für das Lehramt folgt der Master of Education. Hier liegt der Schwerpunkt auf
der Didaktik und auf den Bildungswissenschaften. Außerdem macht ihr im zweiten
Master-Semester euer Schulpraxissemester. Mehr Details wissen wir leider noch
nicht, da die Prüfungsordnung für den Master of Education noch erarbeitet wird.
