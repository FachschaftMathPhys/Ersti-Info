\section{Bachelor Allgemein}
Der Bachelor ist in den ersten Semestern sehr stark von Pflichtvorlesungen
geprägt, die kaum Wahlmöglichkeiten lassen. Gerade die ersten zwei Semester
sind in der Regel mit den Grundvorlesungen gut gefüllt. Ab dem dritten Semester
ist es dann aber meist möglich, euren Interessen nachzugehen.
Ein Bachelor lässt sich in drei bis vier Bereiche einteilen: Pflichtbereich,
Wahlpflichtbereich und Übergreifende Kompetenzen. Im Bachelor Physik kommt noch
der Wahlbereich hinzu, in der Informatik ein Anwendungsgebiet.
Der Pflichtbereich dient dazu, euch die Grundlagen eures Fachs beizubringen und
euch mit der fachlichen Methodik vertraut zu machen.  Im Wahlfpflichtbereich
vertieft ihr dann eure Kenntnisse und spezialisiert euch oft auf ein Gebiet. Im
Bereich Übergreifende Kompetenzen geht es um Soft- und Social Skills, aber auch
um fachlich übergreifendes Wissen.  Im Wahlpflichtbereich könnt ihr
schließlich all das einbringen, was euch Spaß macht und nicht durch einen der
anderen Bereiche abgedeckt ist.

Die verschiedenen Veranstaltungen, die ihr während eures Studiums besucht, sind
in sogenannte Module gegliedert. Nach Definition ist ein Modul eine „thematisch
und zeitlich abgeschlossene Lehr- und Lerneinheit“. Ihr müsst euch allerdings
nicht lange mit dieser doch etwas umständlichen Formulierung befassen. Für euch
ist ein Modul nämlich nichts anderes als eine Vorlesung, die meist mit einer
schriftlichen Prüfung abgeschlossen wird, oder ein Seminar, in dem die Prüfung
aus einem Vortrag zu einem der Seminarthemen. Hin und wieder können euch auch
Module begegnen, die aus mehreren Veranstaltungen bestehen. Dann müsst ihr zum
Abschließen des Moduls eben nicht nur eine Prüfung bestehen oder einen Vortrag
halten, sondern eben alle\footnote{rechtlich heißt es zwar „ein Modul, eine
Prüfung“, aber man kann da Ausnahmen machen.} Verstaltungen des Moduls bestehen.

Um den Stoff einer Vorlesung über das Semester hin zu vertiefen, gibt es in
fast allen Vorlesungen jede Woche einen Übungszettel mit Aufgaben zum aktuellen
Thema. Diese Zettel sind einerseits gut, um den Vorlesungsstoff aufzuarbeiten
und anzuwenden, andererseits braucht ihr auf den Zetteln in der Regel 50\% der
Punkte, um überhaupt zur Klausur zugelassen zu werden. Wie genau das dann
funktioniert, wird euch in den einzelnen Vorlesungen am Anfang jedes Semesters
ausführlich erklärt.

Für bestandene Module bekommt ihr dann Leistungspunkte (LP)\footnote{manchmal
auch Credit Points (CP) genannt}, von denen ihr in eurem Studium insgesamt 180
sammeln müsst (mit einigen Bedingungen verknüpft), um euren Bachelor zu
bekommen. Die Anzahl der Punkte, die ein Modul gibt, berechnet sich aus dem
„Workload“ einer Vorlesung. Das ist der Arbeitsaufwand, den eine Vorlesung und
die dazugehörige Übung mit sich bringt. Zum einen Teil ist das also Zeit, die
ihr in der Vorlesung und in den Übungen sitzt, aber auch die Zeit, die ihr für
das Vor- und Nachbereiten der Vorlesung und für das Rechnen des zugehörigen
Zettel braucht. Dabei entspricht ein LP in etwa 30 Stunden Arbeit im Semester,
also gute zwei Stunden pro Woche. Da es natürlich stark von euch abhängt, wie
lange ihr für das alles braucht und wie viel Zeit ihr wirklich investieren
wollt, kann das nur eine grobe Abschätzung sein, ist aber eine gute
Orientierung, wenn ihr überlegt, wie viel ihr euch im Semester aufbürden wollt.
