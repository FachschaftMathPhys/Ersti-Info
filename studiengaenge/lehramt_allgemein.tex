\section{Die 50\%-Bachelor-Studiengänge (Lehramt)} % Dereinst "Lehramt allgemein"
\label{lehramt_allg}

Lehramtsstudiengang, Staatsexamen, Referendariat -- so ging es vor 2015 zurück an die Schule. Die ersten beiden Schritte haben sich nun verändert, denn die Umstellung der Studiengänge in Deutschland auf das zweistufige Bachelor-Master-System macht auch vor dem Lehramt in Baden-Württemberg nicht halt. Und daher heißt es jetzt: Doppel-Fachbachelor, Master of Education, Referendariat. Andernorts heißt es manchmal auch: Bachelor of Education, Master of Education, Referendariat. Das zweite Konzept versucht, das alte Lehramtsstudium soweit wie möglich zu erhalten, während in Heidelberg mit dem heiEDUCATION-Konzept eine Umstellung auf das Erste vorgenommen wird -- mit dem Nebeneffekt, dass es künftig auch möglich sein wird, Informatik, Mathematik und Physik als 50\%-Bachelorstudiengang zu studieren.

Bisher ist vieles noch nicht endgültige geklärt und einiges noch völlig unklar. Wenn sich etwas neues ergibt oder für Studierende wichtige Dinge entschieden werden, werden wir dies natürlich im das FS-Info\footnote{Eine Mailing Liste, in der wir jeden Monat über die wichtigesten Dinge informieren. Anmeldung dafür unter \url{https://mathphys.fsk.uni-heidelberg.de/mailman/listinfo/fsinfo}} oder in den News unserer Webseite\footnote{\url{https://mathphys.fsk.uni-heidelberg.de/w/}} euch erzählen.

\subsection{heiEDUCATION}

heiEDUCATION ist ein von Universität und Pädagogischer Hochschule (PH) Heidelberg gemeinsam entwickeltes Konzept um „Heidelberg zu einem Ort exzellenter Lehrerbildung auszubauen“\footnote{\url{http://www.hei.education/de/hauptmenue/startseite/}}. Exzellente Lehrerbildung bedeutet dabei folgendes:

Lehramtsstudierende absolvieren zunächst parallel zwei 50\%-Ba\-che\-lor\-stu\-di\-en\-gän\-ge an der Universität Heidelberg. Diese Studiengänge haben fast keinen Lehramtsbezug, lediglich vereinzelte Veranstaltungen für Fachdidaktik und Bildungswissenschaften können belegt werden. Vorgesehen ist, dass man im 6. Semester in einem der beiden Fächer eine Bachelorarbeit schreibt. Während diesem Studium lernt ihr also alle nötigen fachlichen Kompetenzen für das Lehramt.

Anschließend kann man wahlweise auf den „Master of Education“ oder einen Fachmaster in einem der beiden Fächer wechseln. Der „Master of Education“ wird von der PH angeboten und die soll die notwendigen Kompetenzen für das Unterrichten vermitteln. Diese sogenannte „Polyvalenz“ hat den Vorteil, dass dass Lehramtsstudierende bis zum 6. Semester Zeit für die Entscheidung zwischen Fach- und Lehramtsausbildung haben, und die 50\%-Bachelorstudiengänge eine interessante Option für Studierende bieten, die an zwei Fächern großes Interesse haben.

Da dieses neue Konzept von denen, die die Umstellung zu verantworten haben, reichlich positiv gesehen wird, sollen an dieser Stelle einige skeptische Worte nicht fehlen:

Zunächst ist der Master of Education momentan noch ein recht abstraktes Konzept, das erst in den kommenden Semestern Gestalt annehmen wird. Zwar steht beispielsweise schon fest, dass das Praxissemester im Rahmen des Master of Education weiterbestehen wird, vieles andere ist aber noch offen. Außerdem muss man Lehramtsstudierenden ganz klar sagen: Ihr werdet in euren ersten Semestern kaum auf eure spätere Berufspraxis vorbereitet. Sicherlich ist ein tiefes Verständnis von Inhalten auch jenseits des Schulstoffes wichtig, um später an Schulen gut unterrichten zu können. Doch die tatsächliche Planung von Unterricht, der Schritt vom eigenen Verständnis von Inhalten hin zur Fähigkeit, anderen das Verständnis von Inhalten zu ermöglichen, geschieht in diesem Konzept erst spät. Daher der Appell an all diejenigen, die am liebsten sich schon morgen vor eine Schulklasse stellen und unterrichten würden: Habt Geduld und beißt euch durch!

\subsection{Bachelor of Everything\dots}

\dots oder zumindest mal in zwei Fächern gleichzeitig: Einer der Vorteile dessen, wie in Heidelberg das Lehramt umgestellt wurde, ist die hinzugewonnene Möglichkeit, zwei Fächer zu einem Bachelorstudiengang zu kombinieren („Doppelbachelor“) und sich damit für beide Fach-Mas\-ter\-stu\-di\-en\-gän\-ge zu qualifizieren. Zum einen hat man so weitere zwei bis drei Jahre Zeit, sich zwischen den Gebieten zu entscheiden. Zum anderen kann man diesen Studiengang wählen, wenn man zum Ziel hat, an den Grenzflächen zweier Wissenschaften zu arbeiten. Physikalische Chemie, Wissenschaftliches Rechnen und Mathematische Physik sind hier gute Beispiele.

Es muss allerdings darauf hingewiesen werden, dass man sich die größere fachliche Breite auf Kosten der Spezialisierung in den Fächern aneignet. Das kann je nach Fach bedeuten, dass man im Fachmaster später noch Grundlagen aufholen muss, um dieselbe fachliche Tiefe zu erreichen wie die 100\%-Bachelor-Studierenden. Dazu kommt, dass man nur eine Bachelorarbeit schreibt, für die man zwischen den Fächern wählen muss. Teilweise ist an dieses Wahl die Zulassung für den Fach-Master geknüpft, teils treffen Fächer Einschränkungen auf bestimmte Fächerkombinationen. Es ist für Studierende des Doppelbachelors daher besonders wichtig, schon frühzeitig sich über die formalen Regelungen in ihren Fächern zu informieren. Dies könnte beispielsweise eine überblicksartige Lektüre der Prüfungsordnungen sein. Auch die Hinweise für die Fächer Informatik, Mathematik und Physik in den folgenden Abschnitten sind vermutlich hilfreich, doch es können dort natürlich nicht alle möglichen Fächerkombinationen diskutiert werden.


%TODO: \subsection{Immatrikulation für das Lehramtsstudium}
% -- davon habe ich keine Ahnung...
%
% Früher stand da mal:
% \subsection{Immatrikulation für das Lehramtsstudium}
%
% Grundsätzlich immatrikuliert man sich im Staatsexamensstudiengang Lehramt
% immer in mindestens zwei Hauptfächern. Gegebenenfalls kann nach der
% Zwischenprüfung in den Hauptfächern noch ein drittes Fach -- ein sogenanntes
% Erweiterungsfach -- hinzukommen.
%
% Aber Vorsicht! Manche Fächer können nicht auf Lehramt studiert werden, andere
% Fächer nur in bestimmten Kombinationen. Obwohl das Studentensekretariat darauf
% achten sollte, kommt es immer wieder vor, dass Leute mit „falschen“
% Kombinationen immatrikuliert werden. Seht auf jeden Fall nochmal selber nach:
% Welche Fächerkombinationen in Baden-Württemberg auf Lehramt studiert werden
% können, entnehmt ihr einer Tabelle des Kultusministeriums, die ihr bei
% verschiedenen Beratungsstellen (meist auch online auf deren Webseite) einsehen
% könnt: Es gibt für jedes Fach eine eigene Lehramtsberatung sowie eine zentrale
% Beratung für Lehramtsstudierende durch das Oberschulamt; für
% rechtsverbindliche Auskünfte sollte man sich an letztere wenden.
%
% \emph{Achtung:} In anderen Bundesländern gelten andere Regelungen!
%
%
% Zur Immatrikulation für einen Lehramtsstudiengang muss eine Bescheinigung über
% die Ableistung eines sogenannten Orientierungspraktikums vorgelegt werden.
% Sofern diese noch nicht vorliegt, kann sie aber für eine begrenzte Zeit noch
% nachgereicht werden. Die konkrete Frist orientiert sich meist an den sonstigen
% Fristen der Immatrikulation, lest hierzu am besten auf der Seite des
% Studierendensekretariats nach oder ruft dort an (06221 -- 54\,54\,54).


\subsection{Modularisierung}

Die 50\%-Bachelorstudiengänge sind modularisiert, d.h. die Lerninhalte werden in kleinen, abgeschlossenen Einheiten, sogenannten Modulen, vermittelt. Ein Modul umfasst beispielsweise wöchentlich zwei Vorlesungen und eine Übung mit Übungszetteln über ein Semester oder ein Praktikum. Auch die Bachelorarbeit ist ein Modul. Oftmals werden Module sowohl von Studierenden des 100\%- und des 50\%-Bachelorstudienganges belegt. Alle Module werden mit Leistungspunkten (\gls{LP}) versehen, die den Arbeitsaufwand messen sollen. Dabei entspricht 1 \gls{LP} ca. 25-30 Stunden Arbeit. Für einen Bachelor müssen insgesamt 180 \gls{LP} gesammelt werden. Die Gesamtnote des Bachelorabschlusses ergibt sich aus den nach \gls{LP} gewichteten Noten der Module.

\subsection{Fachdidaktik und Bildungswissenschaften}

Für die Zulassung zum Master of Education ist das Absolvieren von Modulen mit insgesamt 20 Leistungspunkten für Fachdidaktik und Bildungswissenschaften erforderlich. Welche Verantstaltungen hier so angeboten werden, ist uns leider auch noch nicht ganz klar, aber da das zum einen „nur“ 20 Punkte, zum anderen in den ersten Semestern vermutlich recht uninteressant ist, könnt ihr das getrost auf euch zukommen lassen.

% TODO: Weitere Informationsangebote schaffen und kommunizieren

%%%%%%%%%%%%%%%%%%%%%%%%%%%%%%%%%%%%%%%
% KEEP THIS!!
%
% Sobald bekannt ist, wie der Master of Education aussehen wird, sollte es hier
% einen Abschnitt darüber geben. Insbesondere ist auf das Praxissemester
% einzugehen. Das wurde früher(TM) mal so beschrieben:
%
% \subsection{Praxissemester}
%
% Gemäß der Lehramtsprüfungsordnung müssen Lehramtsstudierende ein
% Schulpraxissemester absolvieren oder eine vergleichbare Schulpraxis (z.B.
% Assistant Teacher im Ausland) nachweisen. Das Praxissemester soll -- so das
% Kultusministerium -- schon früh den Bezug zur Schulpraxis herstellen. (Dass
% für die Einführung aber auch Kostengründe gesprochen haben, ist kaum zu
% leugnen. Schließlich wird das Schulpraxissemester nicht bezahlt, im Gegensatz
% zu dem halben Jahr Referendariat, welches dadurch ersetzt wird.) Das
% Praxissemester soll in der Regel nach dem dritten oder vierten
% Studiensemester, also gegen oder nach Ende des Grundstudiums absolviert
% werden. Empfehlenswert ist häufig das fünfte Fachsemester, wie es auch in den
% Studienordnungen der meisten Fächer vorgeschlagen wird. Letztlich könnt ihr
% aber den Termin frei wählen. Das Praxissemester dauert 13 Wochen – es beginnt
% zum Schuljahresbeginn im September und endet mit Beginn der Weihnachtsferien.
% Während des Praxissemesters besucht man nachmittags Kurse beim Staatlichen
% Seminar für Didaktik und Lehrerbildung, wo man etwas über Pädagogik und
% Fachdidaktik lernen soll.
%
% Ihr solltet bereits ein einführenden Vorlesung in Bildungswissenschaft sowie
% pro Fach eine Fachdidaktikveranstaltung besucht haben, bevor ihr in der Regel
% im 5. Semester ins Schulpraxissemester geht. Im Frühjahr, bis das kommende
% Semester startet, habt ihr eventuell die Möglichkeit Blockveranstaltungen zu
% belegen.
%
% Die Vergabe der Praktikumsplätze wird über das Internet geregelt -- nähere
% Infos dazu findet man auf der Homepage der jeweiligen Fakultät.
%%%%%%%%%%%%%%%%%%%%%%%%%%%%%%%%%%%%%%%


\subsection{Übersicht} %TODO: Punktezahlen überprüfen! %TODO: die Tabelle muss evtl. angepasst werden (Zeilenumbrüche)
Hier nochmal eine Übersicht über die Leistungspunkte, die im Lehramt, bzw. Doppelbachelor erbracht werden müssen:

\begin{table*}[htb]
	\centering

	\begin{tabular}{ll}
		\toprule
		Bereich & Leistungspunkte\\
		\midrule
		Fach A, Fachstudium & 74\\
		Fach A, Fachdidaktik & \phantom{0}2\\
		\addlinespace
		Fach B, Fachstudium & 74\\
		Fach B, Fachdidaktik & \phantom{0}2\\
		\addlinespace
		Bildungswissenschaften oder Fachwissenschaften* & 16\\
		\addlinespace
		Bachelorarbeit (in einem der Fächer) & 12\\
		\bottomrule
	\end{tabular}

\end{table*}
Dabei sollten im mit * gekennzeichneten Teil Module in den Bildungswissenschaften gewählt werden, wenn ein Übergang in den Master of Education angestrebt wird, andernfalls können je nach Fach diese Leistungspunkte durch fachwissenschaftliche Module erbracht werden.
