%\Large\mathphyssubsubsec{Lehramt Informatik}\normalfont\small
\section{50\%-Bachelor Informatik (Lehramt)}
\sidebar{
	\centering
	\includegraphics[width=3.5cm]{bilder/xkcd_responsible_behaviour_1.png}\\\vspace{10mm}
	\includegraphics[width=3.5cm]{bilder/xkcd_responsible_behaviour_2.png}\\\vspace{10mm}
	\includegraphics[width=3.5cm]{bilder/xkcd_responsible_behaviour_3.png}\\\vspace{10mm}
	\includegraphics[width=3.1cm]{bilder/xkcd_responsible_behaviour_4.png}
}

Wenn ihr den 50\%-Bachelor Informatik studiert, sind zwei Fälle zu unterscheiden: Entweder hört ihr in eurem zweiten Fach Ma\-the\-ma\-tik-Ver\-an\-stal\-tun\-gen oder ihr tut es nicht. Diese Unterscheidung rührt daher, dass ihr in der Informatik auf jeden Fall etwas Mathematik können solltet. Wenn ihr in eurem anderen Fach Mathe hört, habt ihr das aber schon da abgedeckt und könnt euch in dieser Hälfte eures Bachelors ganz auf die Informatik konzentrieren.

Auf jeden Fall hört ihr im ersten Semester \hyperref[info1]{„Einführung in die Praktische Informatik“ (Info 1)} und den Programmierkurs. Hier ist besonders die erste Veranstaltung wichtig, da es euere Orientierungsprüfung in Informatik ist. Im zweiten Semester solltet ihr dann in Informatik die „Einführung in die Theoretische Informatik“\footnote{„Theo“ kann zu Verwirrungen mit Physikern führen, da diese das als „Theoretische Physik“ verstehen} und „Algorithmen und Datenstrukturen“ (AlDa) hören. Im Dritten Semester macht sich der Unterschied, ob ihr in eurem zweiten Fach Mathe hört, bemerkbar, da ihr hier „Mathematik für Informatiker 1“ (MafIn 1) hören sollt. Wenn ihr aber bereits andere Matheveranstaltungen bestanden habt, könnt ihr beim Prüfungsausschuss beantragen, dass ihr stattdessen ein weiteres Info-Modul aus den Wahlpflicht-Veranstaltungen hören könnt. Außerdem ist vorgesehen das ihr im Dritten Semester die „Einführung in die Technische Informatik“ (ITE oder Technische Info)\footnote{dafür hat leider noch niemand eine schön Abkürzung gefunden} hört. Im Vierten und Fünften Semester stehen dann „Betriebssysteme und Netzwerke“ (BeNe), „Software Engineering“ (ISW) und ein Proseminar auf dem Plan. Im sechsten Semester müsst ihr dann neben der Bachelorarbeit, die ihr in einem eurer beiden Fächer schreibt, noch „Datenbanken 1“ (IDB1) hören. Hinzu kommen dann noch ein Anfängerpraktikum und ein Seminar. Falls ihr nach dem Bachelor den „Master of Education“ machen wollt, müsst ihr das Proseminar und das Anfängerpraktikum im Themenbereich „Informatik und Gesellschaft“ (IuG) machen. Da im Seminar und Proseminar und Seminar schon \gls{LP} als „Fachübergreifende Kompetenzen“ (FÜK) vergeben werden habt ihr nun noch 4 \gls{LP} die ihr mit einer FÜK Veranstaltung eurer Wahl holen könnt.

Ihr solltet euch bei jeder Veranstaltung genau darüber informieren, was ihr benötigt, um zur Prüfung zugelassen zu werden (z.B. 50\% auf den Übungszetteln) und was genau \emph{eine} Prüfung beinhaltet (z.\,B.\ das Bestehen von einer von zwei Klausuren). Insbesondere der letzte Teil ist wichtig, denn ihr könnt Prüfungen grundsätzlich zweimal versuchen. Je nach Veranstaltung und Dozent/in \emph{können} zwei Klausuren als eine Prüfung zählen, müssen aber nicht -- dann würde jede geschriebene Klausur als ein Prüfungsversuch gelten. Wenn ihr aus Gründen, die ihr nicht zu verantworten habt (wie krank sein), nicht an einer \emph{Prüfung} teilnehmen konntet, erhöhen sich entsprechend eure Versuche. Wenn ihr alle \emph{Klausur}termine verpasst habt, kann die Klausur auch durch eine mündliche Prüfung ersetzt werden. Sollte das bei euch mal der Fall sein, fragt ihr jedoch am besten den/die Dozent/in.

Im 50\%-Bachelor habt ihr leider nur sehr wenig Wahlmöglichkeiten, da die meisten Veranstaltungen verpflichtend sind. Einen Anhaltspunkt für die Planung eures Studiums können die Musterstudienpläne in eurer Prüfungsordnung sein, die „nach Vorgabe“ erstellt wurden. Das heißt, sie enthalten ca.\ 15 Leistungspunkte pro Semester und versuchen gleichzeitig, die Veranstaltungen möglichst sinnvoll anzuordnen.
