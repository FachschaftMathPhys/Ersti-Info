% !TEX ROOT = ../ersti.tex
\section{Beratung und Informationen zum Lehramt}
\label{lehramtkontakte}
%\newpage\subsection{\Large Beratung und Informationen zum Lehramt}%FIXME

Zum Studium mit Berufsziel Lehrerin in der gestuften Bachelor-Master-Struktur informieren die Heidelberg School of Education (HSE) mit ihren Informationsveranstaltungen und die Zentrale Studienberatung/Career Service (ZSB).
\begin{itemize}
\item \textbf{Service Portal für Studierende} \newline
    Seminarstraße 2, im Erdgeschoss (am Haupteingang links)
    Öffnungszeiten: Mo - Do 10 - 16 Uhr, Fr 10 - 14 Uhr; \newline
    Termin für ein Beratungsgespräch: Tel.: 0 62 21 / 54 - 54 54, E-Mail: studium@uni-heidelberg.de \newline
    Allgemeine Informationen\footnote{\url{https://www.uni-heidelberg.de/studium/interesse/abschluesse/lehramt.html}}

\item Wer sich für den Master of Education interessiert, kann sich hier über die fachspezifischen Zugangs- und Zulassungsvoraussetzungen informieren: \newline
    Teilstudiengang Mathematik\footnote{\url{https://www.uni-heidelberg.de/studium/interesse/abschluesse/mathematik_masterofeducation.html}} \newline
    Teilstudiengang Physik\footnote{\url{https://www.uni-heidelberg.de/studium/interesse/abschluesse/physik_masterofeducation.html}} \newline
    Teilstudiengang Informatik\footnote{\url{https://www.uni-heidelberg.de/studium/interesse/abschluesse/angewand_informatik_masterofeducation.html}}

\item Auskünfte zu den Staatsprüfungen gibt es beim \textbf{Landeslehrerprüfungsamt} (LLPA) Baden-Württemberg, Außenstelle Karlsruhe\footnote{\url{http://www.llpa-bw.de/,Lde/Startseite/Aussenstellen+des+LLPA/beim+Regierungspraesidium+Karlsruhe}} \newline
    Hausanschrift: Hebelstraße 2, 76133 Karlsruhe, Postanschrift: 76247 Karlsruhe \newline
    Frau Zimmer-Kraft (Leiterin): Tel. 0721/926-4500, hannelore.zimmer-kraft@rpk.bwl.de \newline
    Prüfungsberatung (Staatsexamen) findet \beratungpaedagogik, in der Zentralen Universitätsverwaltung (ZUV), Seminarstraße 2, 1. OG, Raum 161 statt.

\item Der AK Lehramt des StuRa\footnote{\url{http://www.stura.uni-heidelberg.de/arbeitskreise/ak-lehramt.html}}, dieser gibt auch den Newsletter „Lehrerzimmer“ über aktuelle Entwicklungen beim Lehramt und die Arbeit des Arbeitskreis heraus, außerdem haben sie einen Hitchhikers-Guide für das Lehramtsstudium erarbeitet \footnote{\url{https://www.stura.uni-heidelberg.de/fileadmin/Dokumente/AKs/Lehramt/LehramtsGuideHeidelberg.pdf}, der Physik-Abschnitt ist jedoch nicht aktuell.}. Der AK freut sich immer über neue Mitstreiterinnen bei ihren regelmäßigen Treffen.  \newline AK Lehramt des StuRa, c/o~StuRa-Büro, Albert-Ueberle-Straße~3-5; Tel:~0\,62\,21/54\,24\,56

\item Die (studentischen) Mitglieder der Studienkommissionen\footnote{\url{https://mathphys.stura.uni-heidelberg.de/w/en/gremienmitglieder/}} sind insbesondere in Angelegenheiten im Zusammenhang mit den Modulhandbüchern und Prüfungsordnungen gute Ansprechpartnerinnen.

\item Bei allen Lehramtsfragen, bei denen ihr nicht wisst, an wen ihr sie stellen sollt, oder die sich aufs Lehramt beziehen und bisher niemand beantworten konnte, könnt ihr diese der Online Lehramtsberatung der HSE \footnote{\url{https://onlineberatunglehramt.hse-heidelberg.de/}} stellen. Sie beantworten euch alle Fragen zur Lehramtsoption im Bachelor, zur Zulassung zum Master und zum Master of Education. Hier erhaltet ihr eine schnelle Auskunft, an wen ihr euch bei Schwierigkeiten wenden könnt.


%\item Noch mehr Infos gibt es im Café mit Lehramtsberatung im Erziehungswissenschaftlichen Seminar, Do 14 bis 18 Uhr, anschließend von 18 bis 20 Uhr Vorträge, Filme und Infos für Lehramtsstudierende.
\end{itemize}


%~ \begin{figure}[h]
%~ \centering{
    %~ \includegraphics[width=3.5cm]{bilder/newton_1.png}
    %~ \hspace{0.55cm}
    %~ \includegraphics[width=3.5cm]{bilder/newton_2.png}
    %~ \hspace{0.55cm}
    %~ \includegraphics[width=3.5cm]{bilder/newton_3.png}
%~ }
%~ \end{figure}
