\section{Mathematik 100\%}

\subsection{Das erste und zweite Semester}

In den ersten zwei oder drei Semestern hört ihr eure Grundvorlesungen und einige Einführungsvorlesungen. Die Grundvorlesungen sind die „Lineare Algebra“ 1 und 2 (\gls{LA}), in denen ihr hauptsächlich Abbildungen zwischen Vektorräumen betrachtet, und „Analysis“ 1 und 2, sowie „Höhere Analysis“ (\gls{Ana}). Dort behandelt ihr Grenzwerte mit allem, was folgt, das heißt Differential- und Integralrechnung.

Auf der Webseite der Fakultät findet ihr nochmal eine Auflistung\footnote{\url{https://www.mathematik.uni-heidelberg.de/modellstudienplaene.html}} wie euer Studium aufgebaut ist. Auch das sind nur Vorschläge, wie ihr euer Studium im Endeffekt einteilt, bleibt natürlich euch überlassen.

\subsubsection{Orientierungsprüfungen}

Orientierungsprüfungen sind Prüfungen, die bis Ende des dritten Semesters bestanden sein müssen. In der Mathe sind das Ana 1 und LA 1 für den 100\%-Bachelor, bzw. LA 1 für 50\%-Bachelor. Diese Vorlesungen müssen also im ersten Semester gehört werden. Sollte man im ersten Anlauf, das heißt in der ersten Klausur und in der Nachklausur nicht bestehen, ist das noch lange kein Weltuntergang. Ihr seid damit nicht die Ersten und nicht die Letzten an der Uni und sicher auch nicht die Einzigen in eurem Semester. Es fallen jedes Jahr etwa 50\%, mal mehr, mal weniger, durch diese Klausuren durch. Macht euch deshalb also keinen Kopf! Ihr habt dann im WS darauf noch einen Versuch, bestehend aus Klausur und Nachklausur. Der muss dann aber klappen, da euch sonst die Exmatrikulation droht.

\subsubsection{Einführungsvorlesungen}

Die Einführungsvorlesungen bestehen aus der „Einführung in die praktische Informatik“ (\gls{IPI}), „Einführung in die Numerik“ (\gls{Num0}) und „Einführung in die Wahrscheinlichkeitstheorie und Statistik“ (\gls{WTheo0}). In allen Vorlesungen müsst ihr Übungszettel lösen, die dazu dienen, den Stoff zu vertiefen. Ihr habt dann zu jeder Vorlesung Übungsgruppen und Tutorinnen, die nicht nur eure Zettel korrigieren und sie mit euch besprechen, sondern vor allem dazu da sind, euch zu helfen. Tutoren werden dafür bezahlt, eure Fragen zu beantworten, egal wie dumm sie euch in dem Moment vorkommen, und machen das auch gerne. Auch die anderen Studis verstehen meist nicht mehr als ihr, sehr wahrscheinlich haben sie ähnliche Fragen. Sucht euch am besten am Anfang eine Gruppe von Kommilitonen, mit denen ihr gut rechnen und überlegen könnt, damit ihr die Zettel nicht alleine bearbeiten müsst. Abgeben sollt ihr sie auch nicht alleine, sondern im Allgemeinen zu zweit.

Zusätzlich kann im zweiten Semester ein Proseminar besucht werden. Das Proseminar setzt eine aktive Teilnahme voraus und wird mit einem mündlichen Vortrag abgeschlossen, eventuell mit zusätzlicher schriftlicher Ausarbeitung. Der Vortrag ist eine sehr gute Möglichkeit \LaTeX zu lernen. \LaTeX ist ein Programm, um pdf-Dokumente zu schreiben, ihr werdet es in eurem Mathestudium noch häufig brauchen und wie heißt es so schön: Früh übt sich, wer ein Meister werden will \textit{(Schiller)}.

\subsubsection{Ist Mathe wirklich richtig für mich?}

In den ersten beiden Semestern kann es passieren, dass eure Noten nicht so gut sind, wie ihr das vielleicht gerne hättet oder aber gewohnt seid. Das ist aber noch lange kein Weltuntergang und auch noch kein Grund, das Studium abzubrechen, wenn ihr Spaß an Mathematik habt. Mathematik an der Uni ist etwas ganz anderes als die Mathematik, die man aus der Schule kennt. Es wird auch gerne die Tatsache unterschätzt, dass man plötzlich in einer ganz neuen Situation ist, eine für die meisten fremde Stadt, neue Leute, ein ganz anderes System als die Schule. Man muss sich erstmal an die geänderten äußeren Umstände gewöhnen und das braucht Zeit. Auch die Klausuren sind nicht so schlimm, wie es einem am Anfang vorkommen mag. Ihr habt im Regelfall eine zweite Chance, wenn ihr durch die erste durchfallt. Sollte auch der zweite Anlauf nicht klappen, dann könnt ihr die Vorlesung im nächsten Jahr noch einmal hören. Es ist also nichts verloren. Verzweifelt nicht zu schnell an der noch sehr ungewohnten Uni-Mathe, ihr werdet euch schnell daran gewöhnen. Ebenfalls wird aus LA 1 und 2 sowie Ana 1 und 2 auch nur die jeweils bessere Lineare-Algebra- und Analysis-Note in eurem Bachelor berücksichtigt.

Wenn ihr sonst über irgendetwas stolpert, gibt es immer noch die Fachschaft, die euch gerne weiterhilft. Ihr könnt entweder persönlich bei uns vorbei kommen oder eine Mail schreiben. Wir versuchen dann, euch zu helfen, wo es nur geht.

\subsection{Alle weiteren Semester}

Der weitere Verlauf des Studiums kann von jeder ganz individuell gestaltet werden. Man kann sich dabei nach einem der Stundenpläne, die auf der Homepage der Fakultät\footnote{\url{https://www.mathematik.uni-heidelberg.de/modellstudienplaene.html}} zu finden sind, richten. Bei der Frage, welche Vorlesungen und Veranstaltungen ihr belegen müsst und könnt, helfen euch Modulhandbuch, Prüfungsordnung und Homepage der Fakultät.

\subsection{Prüfungsordnung}

In der Prüfungsordnung (PO) findet ihr alle wichtigen Informationen darüber, welche Veranstaltungen gehört werden müssen, welche Prüfungen wann und wie abgelegt werden müssen und wie euer Studium sonst geregelt ist. In der Prüfungsordnung könnt ihr nachschlagen, welche Pflicht- und Wahlpflichtveranstaltungen es gibt und wie viele und welche Wahlpflichtveranstaltungen gehört werden müssen (Anlage 1 und 2).  Im Modulhandbuch findet ihr kurze Beschreibungen der Veranstaltungen und welche Vorkenntnisse für sie empfohlen werden. Neben Pflicht- und Wahlpflichtmodulen gibt es noch das Anwendungsgebiet und fachübergreifende Kompetenzen. Über letztere braucht ihr euch keine großen Sorgen zu machen, die hat man meist schneller als man denkt. Möglichkeiten sind hier Praktika, Auslandssemester, Sprachkurse, die das Zentrale Sprachlabor der Universität Heidelberg anbietet, Vorlesungen anderer Fächer und so weiter. 8 der 20 Leistungspunkte für fachübergreifende Kompetenzen sind bereits in andere Veranstaltungen integriert. In Anlage 3 der PO findet ihr mehr Informationen darüber, wie und wann ihr diese Leistungspunkte erwerbt.

\subsection{Anwendungsgebiet}

Zusätzlich zum puren Mathematik-Studium habt ihr das Anwendungsgebiet, was ein Fach eurer Wahl sein kann. In der Prüfungsordnung gibt es aber nur für die Fächer Informatik, Physik, Astronomie, Biologie, Chemie, Wirtschaftswissenschaften und Philosophie eine feste Regelung. Nach Rücksprache mit dem Prüfungsausschuss können jedoch auch andere Fächer angerechnet werden. Habt ihr zum Beispiel ein Studium vorher abgebrochen, kann das oft als Anwendungsgebiet angerechnet werden. Die meisten der oben genannten Fächer fangen laut Musterstudium in eurem dritten Semester an, die interessanten Informatikvorlesungen sind dagegen im Sommersemester. Niemand wird euch daran hindern, euer Anwendungsgebiet schon im ersten Semester zu beginnen, unterschätzt jedoch die Arbeitsbelastung des Grundstudiums nicht.

\subsection{Am Ende Bachelor?}

Ganz am Ende erwarten euch dann noch ein weiteres Seminar, welches häufig als Grundlage für die Bachelorarbeit verwendet wird, und zuletzt die Bachelorarbeit und das zugehörige Bachelorseminar, in dem ihr in der Regel die Arbeit präsentiert.

