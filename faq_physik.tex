\section{Die FAQ-Seite des Pr�fungsausschuss Physik}
Als Studi bekommt man nat�rlich immer gesagt, dass man seine Pr�fungsordnung gelwesen haben sollte, was auch keine vergeudete Zeit darstellt. Allerdings ist es meistens so, dass man konkrete Fragen hat bez�glich Klausurversuchen, Anrechenbarkeit von Vorlesungen bei einem Erasmusaufenthalt oder Sprachkursen, Abgabe der Bachelorarbeit etc. und man deswegen nicht die gesamte Pr�fungsordnung durchforsten will. Oft sind dies Fragen die sich schon viele Studis bereits gestellt haben und die das Pr�fungssekreteriat nicht immer einzeln beantworten will. 

Aus diesem Grunde wurde die sehr sinnvolle Frequently Asked Questions-Seite des Physik-Pr�fungsausschusses\footnote{\url{http://www.ita.uni-heidelberg.de/~dullemond/infopa/faq.html} (auf Englisch, um allen Studierenden Auskunft zu geben)} eingerichtet. In Form von Fragen und Antworten sind hier die wichtigsten Erkenntnisse zusammengefasst und es lohnt sich durchaus pr�ventiv mal dort hereinzuschauen. Auch als Ersti findet man dort Informationen, die einen direkt betreffen werden, sei es nur um festzustellen, dass gewisse Dinge von den Dozierenden frei entschieden werden d�rfen (z.B. ob eine Klausurzulassung ein Jahr sp�ter noch g�ltig ist).

Die letzte Instanz bei konkreten Entscheidungen bleibt nat�rlich der Pr�fungsausschuss, welcher versucht auf Grundlage der Pr�fungsordnung nachvollziehbar zu entscheiden. In einigen komplizierten F�llen deckt die FAQ-Seite auch nicht alle n�tigen Informationen ab, dann lohnt sich eine konkrete Nachfrage oder ein Besuch beim Pr�fungssekreteriat auf jeden Fall.

