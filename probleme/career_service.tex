\section[Zentrale Studienberatung und Career Service]{Zentrale Studienberatung und \\Career Service}
Worüber ihr euch klar werden solltet, ist, was ihr euch vom Studium erwartet. Fragt euch also: Warum studiere ich? Welches Berufsziel schwebt mir vor? Warum gerade dieses Fach? Ihr müsst diese Entscheidungen für euch selbst treffen und Prioritäten setzen – der Stundenplan bietet keine Anleitung dazu, die eigenen Fähigkeiten und Erwartungen im Studium umzusetzen. Bei der Beantwortung dieser Fragen unterstützt euch die zentrale Studienberatung. Hier könnt ihr einen Termin zur Einzelberatung vereinbaren (natürlich kostenfrei und vertraulich) oder Workshops zu den Themen Zeitmanagement, Präsentieren, Prüfungsangst oder Aufschieberitis besuchen. Auch bei Studienzweifeln ist die Studienberatung eine passende Anlaufstelle. Außerdem könnt ihr euch beim Stipendieninfoabend, der regelmäßig im Juni stattfindet, über Stiftungen und verschiedene Möglichkeiten ein Stipendium zu erhalten informieren.
Ihr fragt euch, welche beruflichen Möglichkeiten ihr nach dem Studium habt oder wie ihr am besten an einen Praktikumsplatz kommt? Dann ist eure Anlaufstelle der Career Service, der Beratungsgespräche und Kurse rund um die Bewerbung und den Übergang in die Arbeitswelt anbietet, aber auch viele Infoveranstaltungen abhält. 
Alle Infos zur Zentralen Studienberatung und dem Career Service findet ihr hier: \url{https://www.uni-heidelberg.de/studium/studieneinstieg.html}
