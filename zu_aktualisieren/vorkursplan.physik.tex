% !TEX ROOT = ../ersti.tex
\section{Vorkurs Physik}
Der Vorkurs der Physik besteht aus einem Mathematik-Vorkurs der Fakultät für Physik, gelesen von \dozentvorkurs. Seit Einführung des Bachelor wird auch ein Kurs zu Schlüsselkompetenzen angeboten, der zum Bachelor-Programm gehört, jedoch nicht verpflichtend ist. Ihr könnt also bereits durch den Vorkurs eure ersten Credit-Points erhalten. Treffpunkt ist am 25.09.2017 um 9 Uhr im Hörsaal 1 der Physik, \Gls{INF} 308. Die genauen Inhalte des Vorkurses legt der Dozent je nach Kenntnisstand der Hörerschaft fest, wobei das zugehörige Skript\footnote{\url{http://www.thphys.uni-heidelberg.de/~hefft/vk1/}} einen guten Anhaltspunkt bietet.

Einen genauen Plan der Vorkursveranstaltungen und Orte findet man im Internet \marginQR{http://mathphys.info/vorkurs/plan\#physik} unter \url{mathphys.info/vorkurs/plan\#physik}.

Das Rahmenprogramm, welches ihr (ab der zweiten Woche) gemeinsam mit den Mathematikerinnen und Informatikerinnen habt, wird von der Fachschaft MathPhys organisiert und durchgeführt.
% In der ersten Woche findet ein Spiel statt, bei dem ihr die Stadt kennenlernen könnt.
% Mit "der ersten Woche" ist die 2. Woche gemeint!

Den Vorkurs gibt es inzwischen auch als gebundenes Buch (siehe Buchliste). Geht ruhig mal in der ersten Vorkurswoche in die Universitätsbibliothek und leiht es euch sozusagen als Begleitbuch aus. (Vom Kauf möchten wir dennoch abraten.)
