% !TEX ROOT = ../ersti.tex
\section{Rahmenprogramm}
\subsection{Wanderungen}
An den Wochenenden (Sonntag 06.10. und Sonntag 13.10.) werden wir jeweils eine kleine Wanderung unternehmen. Die erste Tour führt auf den Heiligenberg, die zweite auf den Königstuhl -- die beiden Hausberge (Hügel) Heidelbergs. Am Anschluss an die erste Tour werden wir an der Thing-Stätte\footnote{alle Infos hierzu während der Wanderung erfragen} grillen, im Anschluss an die zweite Tour im Schlossgarten picknicken. Bringt euch für das Picknick alles Wichtige selber mit.\\

\noindent\emph{Wir treffen uns jeweils um 11 Uhr am Bismarckplatz.}

\vfill


\eject

\subsection{Spieleabende}
Das Abendprogramm wird spontan entschieden, meist handelt es sich um Gesellschaftsspiele -- bringt auf jeden Fall eigene Spiele mit, was auch immer ihr unter Spielen versteht.

Bisher sind Gesellschaftsspiele super angekommen (da lernt man sich kennen), Alternativvorschläge sind natürlich trotzdem immer gern gesehen. Falls euch etwas einfällt meldet euch doch einfach, schließlich wird das Ganze ja für euch veranstaltet.

\subsection{Brunch}
Am Tag der deutschen Einheit (3. Oktober) bereiten wir euch einen wunderschönen, leckeren Brunch im \gls{Mathematikon}, der das beste Frühstück wird, das ihr in euren ersten zwei Wochen bekommen werdet. Bringt bitte euer eigenes Geschirr/Besteck mit, damit wir Plastikquatsch sparen.

\vspace{4cm}

\begin{figure}[h]
\centering
\includegraphics[width=.7\linewidth]{bilder/su_doku.png}
\end{figure}
