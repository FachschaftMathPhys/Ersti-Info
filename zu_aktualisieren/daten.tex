% !TEX ROOT = ../ersti.tex
% hier werden die Posten definiert, damit bei Neuwahlen nicht der
% Text durchsucht werden muss

% BITTE BEACHTEN:
%  - Titel sind Böse. Keine Titel. Keine Titel.
%  - Keine abschließenden Leerzeichen. Das ist einfach falsch und flahsc.

%Physik
\newcommand{\dekanphysik}{Weidemüller}
\newcommand{\dekanphysiklang}{Matthias Weidemüller}
\newcommand{\dekanphysikfoto}{bilder/weidemueller_mon.jpg}
\newcommand{\prodekanphysik}{Hans-Christian Schulz-Coulon und Ulrich Schwarz}
\newcommand{\prodekanphysikA}{H.-C. Schulz-Coulon}
\newcommand{\prodekanphysikfotoA}{bilder/hcsc_mon.jpg}
\newcommand{\prodekanphysikB}{U. Schwarz}
\newcommand{\prodekanphysikfotoB}{bilder/schwarz_mon.jpg}
\newcommand{\studiendekanphysik}{Jörg Jäckel}
\newcommand{\studiendekanphysikfoto}{bilder/herrmann_mon.jpg}
	%hier muss ein Foto von Prof. Jäckel eingefügt werden
\newcommand{\pruefausschussvorsitzphysik}{Cornelis Dullemond}
\newcommand{\studienberatungphysik}{Ulrich Uwer und Michael Hausmann}
\newcommand{\dozentvorkurs}{Herrn Thommes und Herrn Weigand}
\newcommand{\bafogphysik}{Jörg Jäckel}
\newcommand{\dekanatphysik}{Dewald-Klussmann}
\newcommand{\dekanatphysiktelefon}{+49\,62\,21 / 54\,-\,19\,646}
\newcommand{\frauenbeauftragtephysik}{Loredana Gastaldo} %Gleichstellungsbeauftragte
\newcommand{\pruefsekphysik}{Frau Hiemenz und Frau Nerger}
\newcommand{\pruefsekphysikfotoA}{bilder/hiemenz_mon.jpg}
\newcommand{\pruefsekphysikA}{Frau Hiemenz}
\newcommand{\pruefsekphysikfotoB}{bilder/nerger_mon.jpg}
\newcommand{\pruefsekphysikB}{Frau Nerger}


%Mathe
\newcommand{\dekanmathe}{Bastian}
\newcommand{\dekanmathelang}{Peter Bastian}
\newcommand{\dekanmathefoto}{bilder/gertz_mon.jpg}
	%Porträit von Bastian einfügen
\newcommand{\prodekanmathe}{Johannes Walcher}
\newcommand{\prodekanmathefoto}{bilder/venjakob_sw.jpg}
	%Porträit von Walcher einfügen
\newcommand{\studiendekanmathe}{Enno Mammen}
\newcommand{\studiendekanmathefoto}{bilder/kanschat_mon.jpg}
	%Porträit von Mammen einfügen
\newcommand{\pruefausschussvorsitzmathe}{Gebhard Böckle, Guido Kanschat (Scientific Computing)}
\newcommand{\studienberatungmathe}{Karl Oelschläger und Winfried Kohnen, Hendrik Kasten und Martin Rheinländer (Lehramt), Michael Winckler (Scientific Computing)}
\newcommand{\frauenbeauftragtemathe}{Ekaterina Kostina} %Gleichstellungsbeauftragte
\newcommand{\bafogmathe}{Markus Banagl}
\newcommand{\dekanatmathe}{Frau Heukäufer}
\newcommand{\dekanatmathetelefon}{+49\,62\,21 / 54\,-\,14\,014}
\newcommand{\pruefsekmathe}{Frau Kiesel}
\newcommand{\pruefsekmathefoto}{bilder/kiesel_mon.jpg}

%Info
\newcommand{\studiendekaninformatik}{Artur Andrzejak}
\newcommand{\studiendekaninformatikfoto}{bilder/andrzejak_mon.jpg}
\newcommand{\studienberatunginformatik}{Wolfgang Merkle}
\newcommand{\pruefausschussvorsitzinformatik}{Gerhard Reinelt}
\newcommand{\bafoginformatik}{Artur Andrzejak}
\newcommand{\pruefsekinfo}{Frau Sopka}
\newcommand{\pruefsekinfofoto}{bilder/sopka_mon.jpg}

%Uni
\newcommand{\frauenbeauftragteuni}{Dominique Lattard}
\newcommand{\rektor}{Bernhard Eitel}
\newcommand{\fskstudisimsenat}{keinen der vier}

%Rest
\newcommand{\redaktionsschluss}{12.08.2016}
\newcommand{\volley}{im Dezember}                 % "Der nächste Termin ist ...   ."
\newcommand{\anfi}{14.\,Oktober 2016}               % Termin für die AnfiFete
\newcommand{\fsraum}{\Gls{INF} 205, Raum 01.301}
\newcommand{\auflage}{850}                        % wie viel Erstiinfos sollen
                                                  % gedruckt werden
\newcommand{\semester}{Wintersemester 2016/17}    % bislang nur fuer titel.tex
\newcommand{\drucktag}{2016-druck}
\newcommand{\webtag}{2016-web}

% wo fehlt noch
\newcommand{\fswe}{2. bis 4. Dezember} % Dieses Semester findet es am <Datum> statt.

\newcommand{\mathphystheotermin}{21.\,Oktober 2016}

% diverse Minister

\newcommand{\wissenschaftsministerbawue}{Theresia Bauer (Grüne)}
\newcommand{\wissenschaftsministerbund}{Johanna Wanka (CDU)}


% Geldbeträge
%\newcommand{\studiengebuehren}{500}
\newcommand{\verwaltungsbetrag}{60}
\newcommand{\studentenwerksbeitrag}{49}
\newcommand{\vsbeitrag}{7,50}
\newcommand{\beitragssumme}{142,30}
%\newcommand{\quasimi}{280}

\newcommand{\lebenshaltungskosten}{\EUR{735 -- 800}}
\newcommand{\studentenwohnheim}{\EUR{168 -- 350}}

% http://www.vrn.de/vrn/tickets/zeitkarten/studenten/vrn-semesterticket/index.html
\newcommand{\semesterticket}{158,50}
\newcommand{\sockelbeitrag}{25,80}

%%% Local Variables:
%%% mode: latex
%%% TeX-master: "ersti"
%%% End:
