% !TEX ROOT = ../ersti.tex
% hier werden die Posten definiert, damit bei Neuwahlen nicht der
% Text durchsucht werden muss

% BITTE BEACHTEN:
%  - Titel sind Böse. Keine Titel. Keine Titel.
%  - Keine abschließenden Leerzeichen. Das ist einfach falsch und flahsc.

%Physik
\newcommand{\dekanphysik}{Plehn}
\newcommand{\dekanphysiklang}{Tilman Plehn}
\newcommand{\dekanphysikfoto}{bilder/plehn_mon.jpg}
\newcommand{\prodekanphysik}{Stephanie Hansmann-Menzemer, Jürgen Berges und Selim Jochim}
\newcommand{\prodekanphysikA}{S. Hansmann-Menzemer}
\newcommand{\prodekanphysikfotoA}{}
\newcommand{\prodekanphysikB}{S. Jochim}
\newcommand{\prodekanphysikfotoB}{}
\newcommand{\prodekanphysikC}{J. Berges}
\newcommand{\prodekanphysikfotoC}{}
\newcommand{\studiendekanphysik}{Selim Jochim}
\newcommand{\studiendekanphysikfoto}{bilder/jochiml_mon.jpg}
\newcommand{\pruefausschussvorsitzphysik}{Cornelis Dullemond}
\newcommand{\studienberatungphysik}{Cornelis Dullemond und Tilman Enss}
\newcommand{\dozentvorkurs}{Herr Jäckel, Herr Flörchinger und Herr Thommes}
\newcommand{\bafogphysik}{Michael Hausmann}
\newcommand{\dekanatphysik}{Dewald-Klussmann}
\newcommand{\dekanatphysiktelefon}{+49\,62\,21 / 54\,-\,19\,648}
\newcommand{\gleichstellungsbeauftragtephysik}{Loredana Gastaldo}
\newcommand{\pruefsekphysik}{Frau Hiemenz und Frau Nerger}
\newcommand{\pruefsekphysikfotoA}{bilder/hiemenz_mon.jpg}
\newcommand{\pruefsekphysikA}{Frau Hiemenz}
\newcommand{\pruefsekphysikfotoB}{bilder/nerger_mon.jpg}
\newcommand{\pruefsekphysikB}{Frau Nerger}


%Mathe
\newcommand{\dekanmathe}{Walcher}
\newcommand{\dekanmathelang}{Johannes Walcher}
\newcommand{\dekanmathefoto}{bilder/walcher.png}
\newcommand{\prodekanmathe}{Artur Andrzejak}
\newcommand{\prodekanmathefoto}{bilder/andrzejak_mon.jpg}
\newcommand{\studiendekanmathe}{Hans Knüpfer}
\newcommand{\studiendekanmathefoto}{bilder/knuepfer_mon.jpg}
\newcommand{\pruefausschussvorsitzmathe}{Gebhard Böckle, Guido Kanschat (Scientific Computing)}
\newcommand{\studienberatungmathe}{Karl Oelschläger und Winfried Kohnen, Hendrik Kasten und Martin Rheinländer (Lehramt), Michael Winckler (Scientific Computing)}
\newcommand{\gleichstellungsbeauftragtemathe}{Ekaterina Kostina}
\newcommand{\bafogmathe}{Markus Banagl}
\newcommand{\dekanatmathe}{Frau Heukäufer}
\newcommand{\dekanatmathetelefon}{+49\,62\,21 / 54\,-\,14\,014}
\newcommand{\pruefsekmathe}{Frau Kiesel}
\newcommand{\pruefsekmathefoto}{bilder/kiesel_mon.jpg}

%Info
\newcommand{\studiendekaninformatik}{Holger Fröning}
\newcommand{\studiendekaninformatikfoto}{bilder/froening_mon.jpg}
\newcommand{\studienberatunginformatik}{Wolfgang Merkle}
\newcommand{\pruefausschussvorsitzinformatik}{Michael Gertz}
\newcommand{\bafoginformatik}{Artur Andrzejak}
\newcommand{\pruefsekinfo}{Frau Sopka}
\newcommand{\pruefsekinfofoto}{bilder/sopka_mon.jpg}

%Uni
\newcommand{\frauenbeauftragteuni}{Christiane Schwieren}
\newcommand{\rektor}{Bernhard Eitel}
\newcommand{\kanzler}{Holger Schroeter}
\newcommand{\fskstudisimsenat}{Philipp Strehlow}

%Rest
\newcommand{\redaktionsschluss}{18.09.2020}
%\newcommand{\anfi}{23.\,Oktober 2019}             % Termin für die AnfiFete
\newcommand{\fsraum}{\Gls{INF} 205, Raum 01.301}
\newcommand{\auflage}{800}                        % wie viel Erstiinfos sollen
                                                  % gedruckt werden

\newcommand{\semester}{Wintersemester 2020/21}    % bislang nur fuer titel.tex
%\newcommand{\drucktag}{2017-druck}
%\newcommand{\webtag}{2017-web}
\newcommand{\vorsitzVS}{C. Chiara Citro und Leon P. Köpfle} % Mantelbogen -> Impressum

%\newcommand{\mathphystheotermin}{19.\,Oktober 2018} es gibt leider keine MathPhysTheo im WS19/20

% diverse Minister

\newcommand{\wissenschaftsministerbawue}{Theresia Bauer (Grüne)}
\newcommand{\wissenschaftsministerbund}{Anja Karliczek (CDU)}


% Geldbeträge
%\newcommand{\studiengebuehren}{500}
\newcommand{\verwaltungsbetrag}{70}
\newcommand{\studentenwerksbeitrag}{54}
\newcommand{\vsbeitrag}{10}
\newcommand{\beitragssumme}{171,75}
%\newcommand{\quasimi}{280}

\newcommand{\lebenshaltungskosten}{\EUR{750 -- 860} }
\newcommand{\studentenwohnheim}{\EUR{170 -- 350} }
\newcommand{\bafoeghoechstsatz}{\EUR{861}}

% http://www.vrn.de/vrn/tickets/zeitkarten/studenten/vrn-semesterticket/index.html
\newcommand{\semesterticket}{175}
\newcommand{\sockelbeitrag}{35,30}

% Öffnungszeiten für diverses im Format \newcommand{\ortundzeit}{start & ende}

\newcommand{\ubAltAusMoFr}{9 & 20}
\newcommand{\ubAltAusSa}{13 & 17}
\newcommand{\ubFeldAusMoFr}{9 & 18}
\newcommand{\ubFeldAusSa}{13 & 17}

\newcommand{\ubAltLesMoFr}{8:30 & 1}
\newcommand{\ubAltLesSa}{9 & 1}
\newcommand{\ubFeldLesMoFr}{8:30 & 22}
\newcommand{\ubFeldLesSa}{9 & 22}

\newcommand{\mathekonMoFr}{9 & 21}
\newcommand{\mathekonSa}{9 & 15}

\newcommand{\physikMoFr}{9 & 12:30}
\newcommand{\physikSa}{13:30 & 16:30}

\newcommand{\auslandsinfooeff}{Mo: 10 -- 15, Di: 10 -- 16, Mi und Do: 10 -- 15, Fr: 10 -- 13} %https://www.uni-heidelberg.de/studium/imstudium/ausland/allgemein.html
\newcommand{\urrmelOeff}{Öffnungszeiten: Di von 16 - 20 Uhr \& Do von 18 - 20 Uhr}
%%% Local Variables:

%%% mode: latex
%%% TeX-master: "ersti"
%%% End:
