% !TEX ROOT = ../ersti.tex
\section{Vorkurs Informatik}
Da im Informatikstudium in den ersten beiden Semestern verhältnismäßig viele Mathe-Module belegt werden, ist der Mathe-Vorkurs auch für alle zukünftigen Infostudentinnen zu empfehlen. Er widmet sich einer Einführung in die universitäre Mathematik.
Parallel zu den Mathevorträgen zum Thema „Folgen“ und „Gruppen“ finden ab dem Donnerstag der zweiten Woche für die Info-Studis Informatik-Vorträge statt. Wo genau ihr diese und alle anderen Veranstaltungen findet, guckt ihr am besten im Internet\footnote{\url{http://mathphys.info/vorkurs/plan\#info}} nach.%


\subsection{Programmiervorkurs}
Der Programmiervorkurs richtet sich an alle diejenigen von euch, die noch keinerlei Kenntnisse im Bereich des Programmierens haben und sich im allgemeinen Umgang mit dem Computer unsicher fühlen. Der Kurs findet vom 12. bis 16. Oktober im Mathematikon statt und ist auch für Mathematikerinnen zu empfehlen.

Die Kenntnisse werden in der \vl{Einführung in die praktische Informatik} (\gls{IPI}, erstes Semester) bzw. spätestens in der \vl{Einführung in die Numerik} (\gls{Num0}) hilfreich sein. Davon abgesehen ist es nützlich, erste Erfahrungen mit UNIX-artigen Betriebssystemen („Linux“) zu machen, da sie im naturwissenschaftlichen Bereich weit verbreitet sind.

Für den Programmiervorkurs ist aufgrund der begrenzten Plätze im Computer-Pool eine vorherige Anmeldung erforderlich!
