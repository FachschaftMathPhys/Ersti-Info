\section{Praktika}

In der Physik werdet ihr früher oder später Praktika absolvieren. Dabei geht es darum, dass ihr eure ganzen erworbenen Kenntnisse aus den Experimentalphysik-Vorlesungen (\gls{Ex}) endlich anwendet, um eigene „Forschung“ zu betreiben.

\subsection{Wann ist das?}
In den Semesterferien zwischen dem zweiten und dem dritten Semester geht der Praktikums-Spaß mit dem Physikalische Praktikum für Anfänger 1 (kurz Anfängerpraktikum oder \gls{AP}) los. In eurem dritten Semester, den Semesterferien zwischen drittem und viertem Semester und im vierten Semester absolviert ihr die beiden Teile des AP 2. Ihr könnt euch dabei selber einteilen, welchen Teil ihr wann macht. Erfahrungsgemäß ist AP 2.1 etwas mehr und etwas komplizierter, daher ist es empfehlenswert, diesen Teil in den Semesterferien zu machen.

Sobald ihr eure Anfängerpraktika hinter euch habt, dürft ihr mit dem Physikalische Fortgeschrittenen-Praktikum (\gls{FP}) anfangen. Auch hier gibt es wieder zwei Teile und ihr müsst aus beiden Teilen mindestens 3 Versuche machen, insgesamt stehen 8 Versuche auf dem Plan. Zu einem davon müsst ihr dann auch noch einen Vortrag halten, und zu einem müsst ihr eine ganz besonders ausführliche Auswertung schreiben. Das FP wird aber nicht mehr -- wie die AP -- als Blockveranstaltung angeboten, sondern ihr dürft euch eure Versuche selber so legen, wie es euch gerade passt -- vorausgesetzt der Versuch wird dann, wann ihr wollt, angeboten.

\subsection{Und wie geht das?}
Das grundlegende Konzept der Praktika ist, dass ihr selber einige Ergebnisse aus Ex 1-5 reproduziert. Sei es jetzt die Erdbeschleunigung über Pendelschwingungen oder die Boltzmannkonstante aus Brown'scher Bewegung, ihr messt selber, wertet eure Ergebnisse selber aus und müsst selbst bewerten, ob das, was ihr gemacht habt, signifikant ist.

Dabei gehen fast alle Versuche nach dem gleichen Schema vor: Ihr bekommt ein Skript, in dem die Theorie des Versuchs, die Durchführung und die Auswertung ausführlich beschrieben sind. Ihr schreibt dann mithilfe dieses Skripts eine Einleitung zu eurem Versuch, in der ihr in eigenen Worten formuliert, was ihr eigentlich tun werdet. Am Tag des Versuchs selber werdet ihr von eurer Betreuerin dann abgefragt, ob ihr Bescheid wisst, was zu tun ist. Danach messt ihr alle wichtigen Größen für diesen Versuch selber an den entsprechenden Apparaturen. Ist das geschehen, schreibt ihr eine Auswertung, in der ihr die gemessenen Größen auswertet und das eigentliche Resultat produziert.
