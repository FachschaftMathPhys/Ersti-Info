\section{Praktika}

In der Physik und Informatik werdet ihr früher oder später Praktika absolvieren. Dabei geht es darum, dass ihr eure ganzen erworbenen Kenntnisse aus den Experimentalphysik-Vorlesungen bzw. dem Programmierkurs und der Software-Engineering-Vorlesung endlich anwendet, um eigene „Forschung“ zu betreiben oder selber Software zu entwickeln.

\subsection{Wann ist das?}

Für die Physikerinnen geht der Praktikums-Spaß mit dem \vl{Physikalische Praktikum für Anfänger 1} (kurz Anfängerpraktikum oder \gls{AP}) in den Semesterferien zwischen dem zweiten und dem dritten Semester los. In eurem dritten Semester, den Semesterferien zwischen drittem und viertem Semester und im vierten Semester absolviert ihr die beiden Teile des AP 2. Ihr könnt euch dabei selber einteilen, welchen Teil ihr wann macht. Erfahrungsgemäß ist AP 2.1 etwas mehr und etwas komplizierter, daher ist es empfehlenswert, diesen Teil in den Semesterferien zu machen.

Sobald ihr eure Anfängerpraktika hinter euch habt, dürft ihr mit dem \vl{Physikalischen Fortgeschrittenen-Praktikum} (\gls{FP}) anfangen. Auch hier gibt es wieder zwei Teile und ihr müsst aus beiden Teilen mindestens 3 Versuche machen, insgesamt stehen 8 Versuche auf dem Plan. Zu einem davon müsst ihr dann auch noch einen Vortrag halten und zu einem müsst ihr eine ganz besonders ausführliche Auswertung schreiben. Das \gls{FP} wird aber nicht mehr -- wie die \gls{AP} -- als Blockveranstaltung angeboten, sondern ihr dürft euch eure Versuche selber so legen, wie es euch gerade passt -- vorausgesetzt der Versuch wird dann, wann ihr wollt, angeboten.

Für die Informatikerinnen ist der Zeitpunkt im Studium freier wählbar. Häufig finden die Praktika auch während der Vorlesungszeit statt und können dann in der vorlesungsfreien Zeit beendet werden. Die meisten Studis machen ihr Anfängerpraktikum (\gls{AP}) im 4. Semester, nachdem sie Einführung in Software Engineering gehört haben. Je nach Dozentin findet am Anfang der Vorlesungszeit eine Vorbesprechung statt, in der Projekte vorgestellt und verteilt werden. Man sollte also auf den üblichen Websiten nachschauen, welche Praktika angeboten werden und dann zur Vorbesprechung kommen oder sich einfach bei der Dozentin melden.
Anders als bei Anfängerpraktikum, wo hauptsächlich das gemeinsame Entwickeln von Software im Team einmal praktisch geübt werden soll, liegt der Fokus des Fortgeschrittenen-Praktikum (\gls{FP}) mehr auf dem Inhalt. Es bietet sich an als Vorbereitung auf eine Abschlussarbeit ein FP durchzuführen, um bereits die Arbeitsgruppe und den Themenbereich näher kennenzulernen.

\subsection{Und wie geht das?}

Das grundlegende Konzept der Praktika in der Physik ist, dass ihr selber einige Ergebnisse aus \gls{Ex} 1-5 reproduziert. Sei es jetzt die Erdbeschleunigung über Pendelschwingungen oder die Boltzmannkonstante aus Brown'scher Bewegung, ihr messt selber, wertet eure Ergebnisse selber aus und müsst selbst bewerten, ob das, was ihr gemacht habt, signifikant ist.

Dabei gehen fast alle Versuche nach dem gleichen Schema vor: Ihr bekommt ein Skript, in dem die Theorie des Versuchs, die Durchführung und die Auswertung ausführlich beschrieben sind. Ihr schreibt dann mithilfe dieses Skripts eine Einleitung zu eurem Versuch, in der ihr in eigenen Worten formuliert, was ihr eigentlich tun werdet. Am Tag des Versuchs selber werdet ihr von eurer Betreuerin dann abgefragt, ob ihr Bescheid wisst, was zu tun ist. Danach messt ihr alle wichtigen Größen für diesen Versuch selber an den entsprechenden Apparaturen. Ist das geschehen, schreibt ihr eine Auswertung, in der ihr die gemessenen Größen auswertet und das eigentliche Resultat produziert.

In der Informatik ist die Idee der Praktika, dass ihr Softwareentwicklungsprojekte selbst plant, durchführt und dokumentiert. Ihr lernt dabei verschiedene Techniken zur Analyse und Beschreibung von Problemen und vertieft euch in der verwendeten Programmiersprache. Die Projekte werden in Teams von bis zu drei Studis durchgeführt und je nach Dozent und Fachbereich werdet ihr entsprechende Anleitungen bekommen.

Bewertet werden am Ende die Software, die ihr entwickelt habt, der Projektbericht und eure Vorstellung des Ganzen in einem Vortrag, der eine Diskussion beinhaltet.