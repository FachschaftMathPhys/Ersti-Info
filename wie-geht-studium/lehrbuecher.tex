\section{Lehrbücher}

Bücher sind in erster Linie eine Geschmackssache. Die meisten Bücher, die
hier aufgelistet werden, behandeln Elemente des Stoffes, der für die ersten
zwei Semester gebraucht wird. Das richtige Buch für sich selbst zu finden,
geht aber nur durch Ausprobieren! Jedem Lerntyp liegen unterschiedliche
Herangehensweisen und damit auch unterschiedliche Bücher. Hier für sich
selbst den richtigen Weg zu finden, kann durchaus seine Zeit dauern -- macht
euch deshalb nicht verrückt wegen der Büchersuche, die Physik und Mathe ist
letztendlich in allen Werken die Gleiche.

Es ist jedoch durchaus empfehlenswert, in einer ruhigen Minute mal ein Thema
in verschiedenen Büchern nachzulesen. Dabei lernt man nicht nur viel über die
eigene bevorzugte Herangehensweise, die unterschiedlichen Blickwinkel können
auch dazu beitragen, ein Thema umfassender (oder überhaupt erst) zu verstehen.

Neben der Möglichkeit, sich in der Universitätsbibliothek Bücher auszuleihen,
darf hier auch ein Hinweis auf den Lesesaal nicht fehlen: Im Erdgeschoss der UB
befindet sich ein großer Bereich mit Arbeitsplätzen, in dem ein großer Teil der
Lehrbücher als Ansichtsexemplare ausliegen.  Dort kann man sich die
verschiedenen Bücher in Ruhe genauer ansehen, außerdem wird dieser Bereich von
manchen Studierenden auch als Lernumgebung sehr geschätzt, da dort absolute
Ruhe und eine sehr konzentrierte Lernatmosphäre herrscht und man alle nötigen
und unnötigen Nachschlagewerke direkt vor Ort hat. Ähnliches gilt übrigens auch
für die Präsenzbibliotheken der Fakultäten (s.o.).

Falls ihr vorhabt, ein Buch zu kaufen, dann lasst Euch Zeit dafür und leiht euch
die Bücher lieber erst einmal aus (alle hier vorgestellten Bücher sind im Bestand
der Leihbibliothek). Beim ersten Durchblättern eines Buches kann man meistens
nicht feststellen, ob einem die Art und Weise der Stoffvermittlung liegt. Das
führt häufig im ersten Semester dazu, dass in Extremfällen kleine dreistellige
Beträge für Fachliteratur ausgegeben werden, die anfangs als absolut notwendig
erscheint und sich dann nach einiger Zeit doch als mehr oder weniger nutzlos
erweist, weil einem die spezielle Herangehensweise des Buches zufällig nicht liegt\dots

Für viele Vorlesungen haben die Dozierenden ein Skript erstellt, was häufig eine gute
Hilfe bei der Nachbereitung des Stoffes ist. Einige dieser Skripte werden auch gedruckt
und im Laufe des Semesters in den Vorlesungen ausgeteilt. Anschließend sind sie im 
Fachschaftsraum kostenlos erhältlich -- kommt einfach bei uns vorbei und fragt nach,
ob das auch bei euren Vorlesungen der Fall ist.

Weiterhin solltet ihr auf jeden Fall die Skriptensammlung\footnote{\url{http://mathphys.fsk.uni-heidelberg.de/w/hauptseite/skripte/}}
der Fachschaft im Netz durchstöbern. Unter „Skriptensammlung (PDFs)“ findet
ihr unsere Links zu Vorlesungsskripten (dort finden sich passende Skripte zu fast
jeder Vorlesung).

\subsubsection{Analysis}
\begin{description}
%\item[Barner/Flohr: Analysis~I,~II; deGruyter]
%{Dieses zweibändige Werk führt ausführlich in das Gebiet der
%Analysis ein und ist zum Lernen des Stoffs gut, zur
%Prüfungsvorbereitung allerdings nur bedingt geeignet.}

\item[Forster] {Die ersten zwei
Bände behandeln recht knapp und kompakt den Stoff der ersten
zwei Semester des Analysis-Kurses. Der dritte Band ist ebenfalls
knapp geschrieben, allerdings sehr umfangreich, so dass meist
nicht einmal die Hälfte des Buches im dritten Semester behandelt
werden kann. Ein Standardbuch, da es auch sehr preisgünstig ist.
Aber zum erstmaligen Lernen nur bedingt geeignet,
dagegen zur Prüfungsvorbereitung relativ gut geeignet (Viele
Übungsaufgaben mit Lösungen in einem Extra-Band).}

%\item[Heuser: Lehrbuch der Analysis~I,~II; Teubner]{
%Sehr umfangreich mit vielen Beispielen und "Ubungsaufgaben (zum Teil mit
%Lösungen), dafür artet die Beschreibung streckenweise in Laberei aus.
%Das Buch enthält dafür allerdings viele historische Bemerkungen, die
%sehr interessante Einblicke in den Stoff und die Entwicklung der
%entsprechenden Themengebiete ermöglichen. Damit ist es zur
%Prüfungsvorbereitung nicht so gut  geeignet -- höchstens für
%Verständnisprobleme. Der hohe Umfang schlägt sich leider auch im
%Preis nieder.}

\item[Königsberger]{
Ein gut strukturiertes Standardbuch. Es wird nicht nur der Stoff der
ersten beiden Semester behandelt, sondern darüber hinaus auch einige
damit zusammenhängende oder weiterführende Themen. Es ist
deutlich ausführlicher geschrieben als Forster und ist so nicht nur
hervorragend zur Prüfungsvorbereitung geeignet, sondern auch
begleitend zur Vorlesung.}

%\item[Walter: Analysis~I,~II; Springer]{
%Ein Buch, das sich sehr gut liest und in etwa in der Mitte zwischen
%Forster und Heuser liegt. Für diejenigen, die mit dieser Art der
%Wissensübermittlung zurechtkommen, ist dies ein „Buch für
%alle Fälle\grqq.}

%\item[Behrens: Analysis ~I, ~II; Vieweg]{
%Mit studentischer Hilfe geschrieben. Wem andere Bücher zu kompliziert
%und unverständlich sind, hat hier oft an der richtigen Stelle die
%richtige Erklärung. Geht allerdings nicht so in die Tiefe, wie
%Königsberger oder Amann, Escher.}

\item[Amann, Escher]{
Ein sehr umfangreiches Buch, welches extrem in die Tiefe geht und eine
schöne Querverbindung zur linearen Algebra schlägt. Wer dieses Buch
durchgearbeitet bekommt, hat wohl alles Wissenswerte gut genug verstanden.
Allerdings ist es ziemlich kompliziert und damit für die meisten Studis 
als erstes Buch nicht geeignet.}

%\item[Weiterführende Werke]{ sind von vielen Autoren erhältlich, hier seien
%nur Dieudonn\'e und S.~Lang erwähnt. Diese Bücher eignen sich aber
%nur zum Vertiefen von schon vorhandenen Analysiskenntnissen und nicht
%zum Studienbeginn.}
\end{description}


\subsubsection*{Lineare Algebra}
\begin{description}
\item[Fischer]{
Ein Standardwerk, das durch seinen günstigen Preis und seine kompakte
Darstellung zum wohl meistgelesenen LA-Buch geworden ist. Es ist
empfehlenswert, wenn man sich nicht von der etwas abstrakten Darstellung
abschrecken lässt. Es bringt den vollständigen Stoff der ersten zwei
Semester in einem Band (natürlich profabhängig). Zur
Prüfungsvorbereitung
ist es relativ gut geeignet. Wem die Mathematik zum Studienbeginn
sowieso
schon zu abstrakt ist, dem sei eher der Beutelspacher empfohlen.
Man sollte die älteren Auflagen (alles vor 10.) meiden, da sie
unübersichtlich sind.}

%\item[Anton: Lineare Algebra; Spektrum]{
%Gut geeignet für alle Studis, die zur oft trockenen Theorie der
%Vorlesung
%viele, viele Beispiele brauchen. Sehr leicht verständlich, umfasst er
%dennoch fast den gesamten Stoff aus LA 1 bis hin zur
%Hauptachsentransformation.}

%\item[Jänich: Lineare Algebra; Springer]{
%Dieses Buch ist wohl die einfachste Hinführung zu den ersten
%Begriffen der LA.
%Allerdings hat es den sehr großen Nachteil, dass in dem Buch nicht
%einmal
%der Stoff der ersten $\frac{2}{3}$ des ersten Semesters behandelt wird. Ein
%Buch, das
%Ihr Euch ausleihen solltet, aber zum Kauf eher nicht geeignet ist. Zur
%Prüfungsvorbereitung ist es absolut ungeeignet.}

%\item[Lipschutz: Lineare Algebra; Schaum's Outline]{
%Dieses Buch ist nicht als Lehrbuch konzipiert, sondern im wesentlichen
%zur
%Prüfungsvorbereitung gedacht, also wenn man einen Haufen
%Beispielaufgaben
%durchrechnen will. Zur Theorievermittlung ist es nicht geeignet.}

%\item[Lorenz: Lineare Algebra I, II; BI-Verlag]{
%Eine etwas theoretischere Einführung in die LA, die vor allem auch
%schon
%Begriffe aus der Algebra übermittelt. Besonders geeignet ist dieses
%Buch für
%Studierende, die sich später in Richtung reine Mathematik
%spezialisieren
%wollen. Aber auch alle anderen, die sich mit dieser Art der
%Präsentation des
%Stoffes zurechtfinden, ist dies ein empfehlenswertes Buch zum
%erstmaligen
%Lernen und zur Prüfungsvorbereitung. Der Stoff\-umfang des Buches geht
%etwas über den in den ersten zwei Semestern vermittelten Stoff
%hinaus.}

\item[Beutelspacher]{
Die wohl zugänglichste Einführung in die Lineare Algebra. Der Autor
verzichtet weitestgehend auf Formeln und versucht, die Ideen möglichst
intuitiv und sprachlich zu vermitteln. Daher ist es für Studis, denen
die
mathematische Vorgehensweise im Studium zu abstrakt ist, sehr zu
empfehlen.
Problematisch ist allerdings, dass gerade diese Fähigkeit zum
Abstrahieren eines der wichtigsten Ziele des ersten Semesters ist.
Außerdem ist der Stoffumfang nicht sehr
groß, wer also mal die ersten Hürden der Mathematik überwunden hat,
sollte
das Buch wechseln.}

%\item[Koecher: Lineare Algebra und analytische Geometrie; Springer]{
%Dieses Buch bringt den Stoff von zwei Semestern zusammen mit Einigem an
%analytischer Geometrie, aufgelockert mit historischen Bemerkungen und
%sehr gut
%gegliedert, allerdings auch ein wenig theoretischer als das Buch von
%Fischer.
%Für die Einführung ist das Buch nur mäßig geeignet, da es manches
%an
%Vorwissen fordert und Beweise teilweise sehr kurz abgehandelt werden.}

%\item[Kowalsky: Lineare Algebra; deGruyter]{
%Ein gutes Lehrbuch, das ähnlich wie der Fischer eher abstrakt
%geschrieben
%ist, aber dennoch verständlich (und nicht zu kurz) erklärt. Der
%behandelte
%Stoff ist mal abgesehen vom Brieskorn am unfangreichsten. Sowohl
%zum Lernen als auch zum Nachschlagen zu empfehlen.}

%\item[Walter: Einführung i. d. LA und analytische Geometrie; Vieweg]{
%Ein sehr gut aufgebautes Buch, das allerdings viel mehr als den
%üblichen Stoff
%%(in den ersten zwei Bänden) vermittelt. Dieses Buch ist zwar teuer
%für ein Grundvorlesungsbuch, doch didaktisch gut aufgebaut und sowohl
%zum Lernen
%als auch zur Prüfungsvorbereitung sehr gut geeignet.}

%\item[Brieskorn: Lineare Algebra und analytische Geometrie ~I,~II;
%Vieweg]{
%Die Bibel der LA. Hier steht (fast) alles drin, was zur Folge hat,
%daß das
%Lesen dieses Buches schnell zur Qual werden kann. Als Nachschlagewerk
%nur bedingt geeignet. Die Anschaffung dieses Buches lohnt sich
%wirklich nur
%für die, die sich mehr mit LA beschäftigen wollen.}

%\item[Bröcker: Lineare Algebra und analytische Geometrie; Birkhäuser]{
%Wenn man Frau Böge glaubt, DAS Buch für eine Prüfung in linearer
%Algebra. Der Stoffumfang ist groß doch
%muss man leider vorne anfangen zu lesen, der Umstieg von anderen
%Büchern auf den Bröcker wird einem nicht
%immer leicht gemacht. Er behandelt einiges, was nicht in jeder
%LA-Vorlesung dran kommt, liest sich dennoch
%ganz angenehm, wenn man sich die Zeit nimmt mit ihm zu arbeiten.}

\item[Bosch]{
Ein gutes Lehrbuch, welches in Heidelberg oft begleitend zur Vorlesung
verwendet wird. Es führt zuweilen relativ abstrakt in die Lineare
Algebra ein und legt bereits dort Grundlagen für die weitergehenden
Algebra-Vorlesungen, die man in anderen LA-Büchern nur zum Teil findet.
}

%\item[dtv Atlas für Mathematik]{
%Handliches und sehr preiswertes Nachschlagewerk. Aufgelistete
%Stichworte werden kurz und verständlich beschrieben und teilweise mit
%bunten Bildern erklärt. Sehr zu empfehlen.}
\end{description}

\subsubsection{Experimentalphysik}
\begin{description}

%\item[Alonso/Finn: Physik; Addison-Wesley]{
%Ein Komplettwerk der Experimentalphysik für die ersten zwei Semester.
%Die
%Darstellung der Physik ist ein wenig mathematischer als üblich, aber
%trotzdem
%ist es ein sehr empfehlenswertes Buch.}

%\item[Bergmann/Schäfer; deGruyter]{
%%Hier steht alles drin, was Mensch über Experimentalphysik wissen muß
%und noch
%viel, viel mehr. Diese Reihe ist nur zum Nachschlagen geeignet, denn
%alles zu
%lernen, was in ihr steht, ist unmöglich. Trotzdem sind diese Bücher
%gut
%geschrieben und eignen sich auch zum Lernen, es muß allerdings stark
%ausgewählt werden.}

%\item[Berkeley Physikkurs; Vieweg]{
%Ein insgesamt sechsbändiger Kurs, dessen erste drei Bände für die
%ersten
%zwei Semester bei weitem ausreichen. Die Reihe ist, ähnlich wie der
%Alonso/Finn, etwas theoretischer gehalten als die deutschen Lehrbücher
%(Gerthsen, Bergmann/Schäfer, ...) und geht an vielen Stellen deutlich
%tiefer,
%als es für das erste Lesen notwendig ist. Diese Stellen können aber
%auch
%guten Gewissens erst einmal überblättert werden und sind später zum
%Verständnis sehr wertvoll. Liest sich ansonsten sehr schön und hat
%auch
%mal bessere Erklärungen, als die deutschen Bücher.}

\item[Demtröder]{
Ein insgesamt vierbändiges Werk. Die Erklärungen sind gut und
tiefgehend, dafür ist das Buch stellenweise sehr theoretisch. Beim
ersten
Lesen empfiehlt es sich, einige Paragraphen zu überspringen. Zum Lernen
und zur Prüfungsvorbereitung ist es sehr empfehlenswert, doch muss
man bei
speziellen Themen und komplizierteren Formeln aufpassen, da selbst die
dritte
Auflage noch stark von Fehlern durchsetzt ist. Inzwischen hat sich das
Buch
trotzdem zu einer Art Standardwerk entwickelt, besonders in den höheren
Experimentalphysikvorlesungen eignet sich das Buch bei vielen Profs
sehr gut
für die Vorlesungsnachbereitung.}

\item[Feynman]{
Diese Bücher sind wunderschön zu lesen, da sie weniger aus Formeln,
sondern
hauptsächlich aus Erklärungen bestehen. Manche finden sie einfach
genial,
andere halten es nur für Gelaber. Es ist aber das einzige Buch, das
wirklich versucht,
Verständnis zu vermitteln (und nicht nur Wissen). Zum Nachschlagen
ist dieses Buch
denkbar ungeeignet -- für die verzweifelten Studierenden, die gerade dabei
sind, den Spaß
am Studium zu verlieren finden sich hier aber zahlreiche Passagen, die
sehr anschaulich
und auf eine unnachahmliche Art und Weise Physik vermitteln und so die
Freude am
Studium wiederbeleben können.}

%\item[Vogel: Gerthsen Physik; Springer]{
%Das Standardwerk schlechthin. Hier steht so ziemlich alles drin,
%was Mensch wissen muß. Die Meinungen sind allerdings auch hier geteilt.
%Die Erklärungen sind meist zu knapp, um Verständnis zu übermitteln
%und
%manche kommen mit dem Stil des Buches nicht zurecht. Dafür ist
%es locker geschrieben, geht sehr in die Tiefe und enthält interessante
%Aufgaben.}

\item[Tipler]{
Das Buch enthält den Stoff der ersten drei Semester. Die Erklärungen
sind
sehr ausführlich, das Buch eignet sich daher hervorragendend zum
Lernen und
zur Prüfungsvorbereitung. Es wird viel Wert auf Verständnis und
Aufgaben
gelegt und es ist einfach nett, im Tipler zu lesen. Allerdings werden
die
Themen nicht immer in der nötigen Tiefe behandelt. Es ist ein
Buch zum Lernen, nicht zum Nachschlagen. Super ist der Aufgabenteil,
zu dem es
ein Lösungsheft mit ausführlichen Beschreibungen gibt. (Den Tipler
könnt ihr euch auch
im FS-Raum ansehen.)}

%\item[dtv Atlas für Physik]{
%Handliches und sehr preiswertes Nachschlagewerk. Aufgelistete
%Stichworte werden
%kurz und verständlich beschrieben und teilweise mit bunten Bildern
%erklärt. Sehr zu empfehlen.}
\end{description}

\subsubsection*{Theoretische Physik}

Bei Büchern der theoretischen Physik muss man leider immer damit rechnen, dass
sie einen recht abstrakten Blick auf die Welt haben und didaktisch nicht
dermaßen hervorragend sind, wie man sich das oft wünscht. Löbliche Ausnahmen
sollen hier vorgestellt werden.

\begin{description}

\item[Fließbach]{
Eines der einfacheren Bücher, allerdings auch nicht so umfangreich
und auch nur bedingt für das erste Semester geeignet, da die Newton'sche Mechanik
nur sehr knapp behandelt wird. Wer mit der theoretischen Physik 
Schwierigkeiten hat, findet hier ab dem zweiten Semester ein gutes Buch.
Dazu passend gibt auch ein Arbeitsbuch, in dem alles noch mal zusammengefasst
und an Aufgaben erläutert wird.}

\item[Nolting]{Mehrbändige Theo-Reihe, die vor allem beim Lösen von
Übungsaufgaben und bei der Klausurvorbereitung hilfreich ist. Sehr
gut und nachvollziehbar strukturiert und eignet sich deshalb auch
zum Wiederholen. Entspricht bis auf die Reihenfolge von der
Vorgehensweise auch den Heidelberger Theorievorlesungen.
}

\item[Bartelmann]{Diese Komposition aus Heidelberger Gefilden deckt alle für
das Studium relevanten Bereiche der Theoretischen Physik in einem Band ab.
Geschrieben von Dozenten, die für gute Lehre bekannt sind wird das Buch von
vielen Studierenden als gut lesbar empfunden. Ist in den Bibliotheken dermaßen
vorhanden, dass sich sogar eine Kopie in die Altstadt verirrt hat.}
\end{description}


\subsubsection{Mathematische Methoden}

Eine Vorlesung, die ihr zwar nicht mehr hört, doch trotzdem sind die
Themen für
Physiker wichtig, da hier die Mathematik auf Gebrauchsniveau gehievt
wird.

\begin{description}

\item[Boas: Mathematical Methods in the Physical Science]{
In diesem Buch wird die Mathematik so gebracht, wie sie in der Physik
gebraucht
wird. Es ist wohl das beste Buch zu diesem Thema. In der Boas werden
zahlreiche
für Physiker wichtige Vorgehensweisen anschaulich erklärt und in
vielen
Beispielen ausführlich vorgerechnet. Außerdem enthält das
Buch Aufgaben mit Lösungen. Vor dem Englisch braucht ihr keine Angst zu
haben, denn mathematical english ist immer sehr viel einfacher als
normal
english. Sehr zu empfehlen, sowohl als Nachschlagewerk als auch um
verschiedene
generelle Schwierigkeiten zu beheben.}

\item[Lang, Pucker: Mathematische Methoden in der Physik]{
Dieses Buch wird zur Zeit von den DozentInnen empfohlen und ist in etwa
äquivalent zur Boas -- nur auf Deutsch und etwas günstiger in der
Anschaffung.
Teilweise etwas weniger ausführlich.}

%\item[Großmann: Mathematischer Einführungskurs i. d. Physik]{
%Behandelt den Stoff des Vorkurses und etwas mehr. Es ist
%wahrscheinlich besser,
%gleich ein weiterführenderes Werk zu erwerben, ansonsten aber gut zum
%Einsteigen.}

%\item[Hefft: Mathematischer Vorkurs zum Studium der Physik]{
%DAS Begleitbuch zu eurem Vorkurs, später wohl weniger nützlich.
%Glaubt nicht,
%dass ihr alles verstehen müsst, was euch da serviert wird -- es ist
%aber schön, von den
%ein oder anderen Dingen schonmal etwas gehört zu haben.}

\item[Otto: Rechenmethoden für Studierende der Physik im ersten Jahr]{
In diesem Buch ist die Mathe, die man in den ersten beiden Semestern eines Physikstudiums braucht, anschaulich und ausführlich erklärt. Dabei wird bewusst auf mathematische Beweise verzichtet und mehr auf die physikalische Interpretation eingegangen. Sehr gut geeignet für Studies, die sich am Anfang in Theo ein wenig von der Mathe überrumpelt fühlen.}
\end{description}

\subsubsection*{Nachschlagewerke}

Als eine kommentierte Formelsammlung können folgende Bücher dienen:

\begin{description}

\item[Bronstein/Semendjajew]{
Eines der Bücher, die man als Physiker von jedem Prof empfohlen
bekommt -- zu
Recht. Da das Buch nur die Formeln und Beispiele enthält und keine
Beweise,
ist es für Mathematiker nicht so interessant, trotzdem aber nützlich,
wenn
man irgendetwas berechnen muss. Der Bronstein hat sich bei"
PhysikerInnen
zu einer Art Bibel entwickelt (jeder hat es, jeder benutzt es) da
hierin jede Menge Integrale,
Taylorreihenentwicklungen, usw. aufgelistet werden.}

\item[Stöcker: Taschenbuch der Physik]{
Experimentell orientiertes, sehr kompaktes Nachschlagewerk mit
teilweise sehr
guten und einprägsamen Erklärungen für die stabile Studi-
Jackentasche.
Auch praktisch zum Lernen in Bus und Straßenbahn.}
\end{description}
