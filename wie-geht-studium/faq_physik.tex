\section{Die FAQ-Seite des Prüfungsausschuss Physik}
Als Studentin bekommt man natürlich immer gesagt, dass man seine Prüfungsordnung gelesen haben sollte, was auch keine vergeudete Zeit darstellt. Allerdings ist es meistens so, dass man konkrete Fragen hat bezüglich Klausurversuchen, Anrechenbarkeit von Vorlesungen bei einem Erasmusaufenthalt oder Sprachkursen, Abgabe der Bachelorarbeit etc. und man deswegen nicht die gesamte Prüfungsordnung durchforsten will. Oft sind dies Fragen die sich schon viele Studentinnen bereits gestellt haben und die das Prüfungssekreteriat nicht immer einzeln beantworten will. 

Aus diesem Grunde wurde die sehr sinnvolle Frequently Asked Questions-Seite des Physik-Prüfungsausschusses\footnote{\url{http://www.ita.uni-heidelberg.de/~dullemond/infopa/faq.html} (auf Englisch, um allen Studierenden Auskunft zu geben)} eingerichtet. In Form von Fragen und Antworten sind hier die wichtigsten Erkenntnisse zusammengefasst und es lohnt sich durchaus präventiv mal dort hereinzuschauen. Auch als Ersti findet man dort Informationen, die einen direkt betreffen werden, sei es nur um festzustellen, dass gewisse Dinge von den Dozentinnen frei entschieden werden dürfen (z.B. ob eine Klausurzulassung ein Jahr später noch gültig ist).

Die letzte Instanz bei konkreten Entscheidungen bleibt natürlich der Prüfungsausschuss, welcher versucht auf Grundlage der Prüfungsordnung nachvollziehbar zu entscheiden. In einigen komplizierten Fällen deckt die FAQ-Seite auch nicht alle nötigen Informationen ab, dann lohnt sich eine konkrete Nachfrage oder ein Besuch beim Prüfungssekreteriat auf jeden Fall.
