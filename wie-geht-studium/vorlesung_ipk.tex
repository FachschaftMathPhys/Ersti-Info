\subsection{Programmierkurs}
\label{ipk}
Der Programmierkurs ist für alle Informatik-Studentinnen eine Pflichtveranstaltung. Je nach Angebot wird er als Vorlesung mit zwei Semesterwochenstunden oder als Blockkurs in den Ferien gelesen. Im Gegensatz zur praktischen Informatik lernt Ihr hier keine informatischen Konzepte, sondern ihr erlernt das Programmieren in der Sprache C++. Trotzdem sollte man diesen Kurs nicht zu sehr auf die leichte Schulter nehmen, denn die Klausur wird an einem PC mit Linux durchgeführt und eure Programme müssen korrekt laufen. In der Regel lernt ihr zunächst die grundlegenden Datentypen, Operationen und erschiedene Kontrollstrukturen, und am Ende wird objektorientiert programmiert (z.B. Vererbung, Templates und die Methoden aus der Standard-Bibliothek)
