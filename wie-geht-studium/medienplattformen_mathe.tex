\subsection{Mathematische Medienplattform -- MaMpf}
\label{mampf}

Bücher sind nicht für alle das geeignete Lernmittel, daher existiert seit dem Wintersemester 2016 \emph{MaMpf}\footnote{Erreichbar unter \url{https://mampf.mathi.uni-heidelberg.de/}}, die Mathematische Medienplattform. Initiiert wurde sie von Dr. Denis Vogel. Geboten wird ein vielfältiges Angebot, welches beim Lernen sehr nützlich sein kann.

\subsubsection{KaViaR -- Kameraloses Videoaufzeichnungssystem zur Rekapitulationsunterstüzung}
\emph{KaViaR} ist das Herzstück von MaMpf, kurz gesprochen handelt es sich um eine Videoplattform speziell für Mathevideos. Besonders praktisch ist, dass ihr auch auf Videos von vergangenen Vorlesungen zugreifen könnt. Zurzeit ist das Projekt mit einem Passwort geschützt, dies erfahrt ihr in eurer ersten „MaMpf“--Vorlesung oder fragt eine Studierende aus höheren Semester.

\subsubsection{KeKs -- Kompetenzerweiterendes Kurzfragensystem}
In vielen Prüfungen der ersten Semester werden eure Klausuren einen Multiple--Choice--Teil haben. Darauf vorbereiten könnt ihr euch mit über 5000 Fragen mit \emph{KeKs}, dem Kompetenzerweiterenden Kurzfragensystem. Zum Teil werden die Antworten nicht nur mit „wahr“ oder „falsch“ korrigiert, sondern auch durch Erklärvideos verständlich vermittelt.

\subsubsection{ErDBeere -- Erkenntnisfördernde Datenbank zur Beispielerfassung und -entwicklung}
Die \emph{ErDBeere} ist das jüngste Kind der Plattform. In der Datenbank könnt ihr „intelligent“ nach mathematischen Objekten suchen und euch diese erklären lasse, inklusive Verweise auf Videos im KaViaR.

\subsubsection{RestE und SeSAM}
Mit \emph{SeSAM} (Sammlung exemplarischer Schilderungen von Algorithmen und Methoden) steht euch eine immer weiterwachsende Anzahl an Erklärungen zu Algorithmen zur Verfügung. Abgerundet wird das Paket durch \emph{RestE} (Restliche E-Learning-Angebote), wo sonstige Angebote zusammengefasst werden.
