\section{MaMpf}
\label{mampf}

Bücher sind nicht für alle das geeignete Lernmittel, daher existiert seit dem Wintersemester 2016 \emph{MaMpf}\footnote{Erreichbar unter \url{https://mampf.mathi.uni-heidelberg.de/}}, die \emph{Mathematische Medienplattform}. Initiiert wurde sie von Dr. Denis Vogel. Ihr findet dort ein umfangreiches E-Learning-Angebot, das stetig erweitert wird. Schaut auf jeden Fall mal vorbei, denn viele der Ressourcen richten sich speziell an Erstis und können beim Lernen sehr hilfreich sein!

Nach einer Registrierung mit eurer E-Mail-Adresse habt ihr vollen Zugriff auf die folgenden Angebote:

\subsubsection{KaViaR}
\emph{KaViaR (Kameraloses Videoaufzeichnungssystem zur Rekapitulationsunterstüzung)} ist ein System, das speziell entwickelt wurde, um Mathematikvorlesungen aufzuzeichnen. Ihr findet dort Videos vieler aktueller und vergangener Vorlesungen, die euch ermöglichen, zu Hause noch einmal die Erklärungen eurer Dozierenden anzusehen.

\subsubsection{KeKs}
In vielen Prüfungen der ersten Semester werden eure Klausuren einen Multiple--Choice--Teil haben. Darauf vorbereiten könnt ihr euch mit über 5000 Fragen mit \emph{KeKs (Kompetenzerweiterendes Kurzfragensystem)}. Zum Teil werden die Antworten nicht nur mit „wahr“ oder „falsch“ korrigiert, sondern auch durch Erklärvideos verständlich vermittelt.

\subsubsection{SeSAM}
Mit \emph{SeSAM (Sammlung exemplarischer Schilderungen von Algorithmen und Methoden)} steht euch eine immer weiterwachsende Anzahl an Mathevideos zur Verfügung. Dort werden typische Anwendungen von Algorithmen vorgerechnet und wichtige Begriffe aus den ersten Semestern anhand von Beispielen erklärt. 

\subsubsection{KIWi}
Im Mathematikstudium bauen häufig Vorlesungen stark aufeinander auf. In \emph{KIWi (Katalog von Inhalten zur Wissensauffrischung)} werden Inhalte vergangener Vorlesungen zu kompakten Wiederholungshäppchen zusammengefasst, so dass ihr euch die benötigten Inhalte mithilfe eines kurzen Videos ins Gedächtnis rufen könnt.

\subsubsection{ErDBeere}
Die \emph{ErDBeere (Erkenntnisfördernde Datenbank zur Beispielerfassung und -entwicklung)} befindet sich zur Zeit noch im Aufbau. In der Datenbank könnt ihr „intelligent“ nach mathematischen Objekten suchen und euch diese erklären lassen, inklusive Verweise auf Videos im KaViaR.

\subsubsection{RestE}
In \emph{RestE (Restliche E-Learning-Angebote)} werden sonstige Angebote zusammengefasst, etwa Vorlesungsmanuskripte.

\subsubsection{MÜSLI}
Auch \emph{MÜSLI (Mathematisches Übungsgrup"=pen"~ und Scheinlisten"-interface)} wird als Teil von MaMpf verstanden. Es wird von der Fakultät für Mathematik und Informatik zur Verwaltung von Übungsgruppen eingesetzt und wird euch daher in den ersten Semestern auf jeden Fall begegnen.
