\subsection{Einführung in die Technische Informatik}
\label{info2}
Die Einführung in die Technische Informatik (\gls{ITE}) ist eine Pflichtvorlesung für alle Informatik-Studentinnen. Hier lernt ihr Konzepte aus der binären Logik, mit denen Prozessoren arbeiten. Nebenbei beschäftigt ihr euch mit verschiedenen Schaltungen und mit Rechenschemata in verschiedenen Zahlensystemen (z.B. Binärsystem, Umwandlung vom Quarternär- ins Hexadezimal-System und umgekehrt, Addition und Multiplikation auf diesen Systemen usw.). Im letzten Drittel der Vorlesung lernt ihr etwas Allgemeines zu Rechnerarchitekturen. Insgesamt ist es für das erste Semester eine sehr schöne Vorlesung, die euch sanft ins Informatik-Studium einführt.
