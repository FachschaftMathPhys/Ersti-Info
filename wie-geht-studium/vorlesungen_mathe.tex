\section*{Vorlesungen Mathematik}

\subsection{Lineare Algebra 1}
\label{la1}
Die Lineare Algebra 1 (\gls{LA}) hören Mathematikerinnen, Physikerinnen und (eventuell) Informatikerinnen zusammen, bereitet euch also auf eine sehr große Vorlesung vor. Neben den vielen Grundlagen die euch hier vermittelt werden,  handelt es sich inhaltlich um die Vektorrechnung, wie ihr sie bereits aus der Schule kennt. Diese wird jedoch viel allgemeiner und abstrakter als bisher eingeführt, was am Anfang etwas umständlich erscheint, später die Vorteile jedoch offenbar werden und „Sie sehen, dass das wieder eine Matrix wird und wir das selbe wie immer können“ (Prof. Wingberg). Im weiteren Verlauf kommen auf der Vektorrechnung aufbauend noch lineare Operatoren und Innenprodukträume hinzu, die euch Begriffe und Sätze wie Determinanten, Eigenwerte oder den Spektralsatz näher bringen. Die Inhalte werden euch euer ganzes Studium nicht mehr verlassen, da praktisch alle höheren Vorlesungen auf das mächtige Werkzeug der Linearen Algebra zurückgreifen.

\subsection{Analysis 1}
\label{ana1}
Die Analysis 1 (\gls{Ana}) muss von den Mathematikerinnen gehört werden, jedoch sind auch Physikerinnen und Informatikerinnen nicht unbedingt schlecht beraten, daran teilzunehmen. Hier lernt ihr richtige, fundierte Mathematik in all ihrer Schönheit und Abstraktion (was sehr abstrakt sein kann). Diese Vorlesung hat mit dem Matheunterricht aus der Schule ähnlich viel gemeinsam wie mit dem Sportunterricht.

Ihr lernt das formell richtige Argumentieren und Beweisen und erhaltet einen Einblick darin, was das Gebäude der Mathematik eigentlich ausmacht und wie dieses aufgebaut ist. Inhaltlich beginnt sie mit der Konstruktion der reellen Zahlen, führt über Folgen und Reihen zur Stetigkeit von (reellen) Funktionen und schließlich zur Differential- und Integralrechnung. Der Arbeitsaufwand für die Vorlesung schwankt (je nach Prof) zwischen lächerlich und immens, auch hier sind zehn oder mehr Stunden für einen Zettel nicht unbedingt Seltenheit (es lohnt sich Zettelgruppen zu bilden und gemeinsam zu rechnen). Was das unglaublich frustrierende und hilflose Gefühl betrifft, man würde nichts verstehen und wäre völlig falsch in seinem Studiengang, keine Angst, das haben alle. Wenig härtet so gut gegen Frust ab wie eine erbarmungslose Mathevorlesung im ersten Semester. Aber lasst Euch davon nicht täuschen, dass Begriffe, die ihr meint aus der Schule zu kennen, unnötig umständlich eingeführt werden. Es hat alles durchaus seine mathematische Berechtigung und schafft ein Fundament, auf das ihr später aufbauen werdet. Sich ein sauberes, formelles Vorgehen und Denken anzugewöhnen ist unabdingbar.

\subsection{Mathematik für Informatiker}
Die Mathematikausbildung für Informatikerinnen ist etwas kompliziert. Ihr habt die Auswahl zwischen den Vorlesungen Lineare Algebra 1 (\gls{LA}) und Analysis 1 (\gls{Ana}) oder den Vorlesungen Mathematik für Informatiker 1 und 2 (\gls{MafIn}). Ihr entscheidet euch also bereits im ersten Semester für eine der beiden Varianten. Deshalb solltet ihr euch schon jetzt mit diesem Thema auseinandersetzen.
\subsubsection{Was spricht für MafIn?}
Die Vorlesung ist extra auf Informatikerinnen zugeschnitten und versucht euch den Stoff besser verständlich näher zu bringen. Die MafIn 1 orientiert sich dabei an den grundlegenden Inhalten der LA 1, die MafIn 2 an denen der Ana 1. Grundsätzlich ist die Vorlesung solide konzipiert und kann euch die wichtigen Mathematik-Grundlagen vermitteln. Wie bei jeder Vorlesung hängt natürlich auch hier das Konzept und die Qualität stark vom jeweiligen Lehrpersonal ab, das sich in den letzten Jahren nicht geändert hat. Ihr solltet euch aber selbst ein Bild von der Vorlesung machen, ob sie euch zusagt oder nicht. Lasst uns gerne Feedback zur Vorlesung zu kommen\footnote{Feedback könnt ihr uns am besten per Mail, oder bei der Informatik-Vollversammlung zukommen lassen.}.
\subsubsection{Was spricht für Lineare Algebra 1 (LA 1) \\und Analysis 1 (Ana 1)?}
Die Vorlesungen\footnote{Details zur LA 1 siehe Seite \pageref{la1}, zur Ana 1 siehe Seite \pageref{ana1}} sind für Mathematikerinnen konzipierte Veranstaltungen. Diese bieten eine präzisere Herangehensweise an, vor allem bei Formulierung der Definitionen, Sätze und insbesondere Beweisen. Ihr bekommt dadurch eine solidere Mathematikausbildung, die ihr in manchen Bereichen der Informatik auch dringend benötigt. Die LA 1 und Ana 1 bedeutet für euch aber ziemlich sicher auch mehr Arbeit und Zeitaufwand, da ihr auch eine intensivere und vertiefendere Ausbildung bekommt. Es heißt auch dass ihr im Ersten Semester neben \gls{IPI}, LA 1 und Ana 1 keine Zeit mehr habt noch die Einführung in die Technische Informatik (\gls{ITI}) zu hören\\

Schlussendlich müsst ihr selbst wissen, was für euer Studium und euren Studienplan das beste ist. Informiert euch möglichst zu Beginn eures Studiums was für euch besser passt. Im Zweifel könnt ihr euch auch in beide Vorlesungen rein setzen und die Art der Vorlesungen vergleichen. Macht dies aber nicht länger als ein paar Wochen, da der Mehraufwand gerade im ersten Semester sehr hoch ist. Im Zweifel bekommt ihr in der Regel während des Vorkurses auch noch einen Vortrag zum Studienaufbau, bei dem ihr noch Fragen stellen könnt. Ansonsten freuen wir uns wenn ihr mit Fragen und Anregungen zu dieser schwierigen Entscheidung zu uns\footnote{Mehr Infos zur Fachschaft im \autoref{diefsmathphys}.} kommt.

\subsection{Höhere Mathematik für Physiker}
\label{mathephysik}
Die Mathematikausbildung für Physikerinnen sieht vor, dass ihr die Lineare Algebra 1 (\gls{LA}) im ersten Semester hört. Laut Modulhandbuch könnt ihr euch dann vor dem zweiten Semester entscheiden, ob ihr mit Analysis 2 und 3 (\gls{Ana}, was auch die Mathematikerinnen hören) oder mit Höhere Mathematik für Physiker (\gls{HoMa}) 2 und 3 weitermacht.
\subsubsection{Was spricht für HöMa?}
Die Vorlesung ist extra auf euch als Physikerinnen zugeschnitten und legt ihren Schwerpunkt auf die Vorlesungen Ana 1-3 in strafferer Form. Während Mathematikvorlesungen einer gewissen Freiheit unterliegen und es durchaus vorkommen kann, dass die Dozentin einen Schwerpunkt auf ihr Forschungsgebiet legt, hört ihr in HöMa größtenteils nur jene Dinge, die in der Physik auch verwendet werden. Außerdem kommen Beispiele gerne aus der Physik und liegen euch deshalb vielleicht näher. Trotzdem handelt es sich nicht um eine Schmalspurversion, sondern um eine vollwertige Mathematikvorlesung, die auch für zukünftige Theoretikerinnen nicht ungeeignet ist.
\subsubsection{Was spricht für die Analysis (Ana)?}
Die Analysis bietet als eine für Mathematikerinnen konzipierte Veranstaltung eine präzisere Formulierung der Definitionen, Sätze und insbesondere Beweise. Somit wird es möglich die mathematischen Hintergründe in der Physik besser zu durchdringen und weitergehende Verbindungen der Gebiete zu erkennen. Dieses tiefere Verständnis kann unter anderem in der theoretischen Physik oder auch in weiterführenden Matheveranstaltungen von Vorteil sein und lässt euch insgesamt mehr Freiheiten im weiteren Studienverlauf, besonders bezüglich der Mathematik. Zudem ist je nach Dozentin und Forschungsbereich auch eine gewisse Schwerpunktlegung (vor allem in der Ana 3) möglich.\\

Auf den ersten Blick mag es verwundern, dass ihr in die Ana 2 einsteigen sollt, ohne die erste Vorlesung dazu gehört zu haben. Dies ist theoretisch zumindest möglich, jedoch vermutlich mit ein wenig Mehraufwand verbunden. Trotzdem können mathematisch Ambitionierte natürlich auch die Ana 1 im ersten Semester hören, da diese eine schöne Einführung in den Themenbereich Analysis und die damit verbundenen Methoden darstellt. Das kann euch vor allem in weiterführenden Vorlesungen weiterhelfen; auch wird es eine große Erleichterung für die Ana 2 sein, wenn ihr schon ein wenig mehr mit der Materie und der Dozentin vertraut seit. Andererseits werdet ihr mit dem Kursprogramm auch so schon stark ausgelastet sein. Wenn ihr es mit vier Vorlesungen versuchen wollt, solltet ihr euch nach zwei bis drei Wochen entschieden haben, ob ihr das im ersten Semester durchhaltet oder nicht, da es für den Übungsbetrieb ziemlich blöd ist, wenn in der Mitte des Semesters viele Leute aussteigen.


