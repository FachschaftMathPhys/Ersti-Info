%\section{Was sind diese „Übungsgruppen“?}
\newpage\mathphyssecnobar{Was sind diese „Übungsgruppen“?}

%\subsection{Übungszettel}
\noindent In den meisten Vorlesungen müsst ihr, um zur Klausur zugelassen zu werden, Übungszettel rechnen. Das bedeutet, jede Woche werden irgendwo\footnote{Wo, wird in der ersten Vorlesung bekannt gegeben} Übungsaufgaben hochgeladen, die ihr euch ausdrucken und dann rechnen sollt. Weil das erfahrungsgemäß nicht so einfach ist, und meistens viele viele Fragen auftreten, gibt es Übungsgruppen.

\subsection{Übungsgruppen}
In fast allen Vorlesungen werden vorlesungsbegleitend sogenannte „Übungsgruppen“ angeboten. Diese Übungsgruppen sind dazu da, euch bei euren Problemen mit der Vorlesung zu helfen und die Zettel nachzubesprechen. Dazu seid Ihr natürlich nicht auf euch allein gestellt, sondern euch wird eine Tutorin zur Seite gestellt, die Euch bei allen Fragen, die auftreten, kompetente Hilfe bietet. Hier unterscheiden sich die Mathe/Informatik und die Physik: In der Physik werden die Übungsgruppen meistens mindestens von Masterstudentinnen, häufig auch von Doktorandinnen oder gar Profs gehalten. Dementsprechend studierenden fern sind diese oftmals, aber natürlich gibt es auch Tutorinnen, die geradezu geniale Übungsgruppen halten. Grundsätzlich gilt es aber, so viele Fragen wie möglich zu stellen, auch um der Tutorin Feedback zu eurem Kenntnisstand zu geben. In der Mathe werden die Übungsgruppen in den meisten Fällen von Studierenden höherer Semester gehalten, es ist durchaus nicht unüblich, eine Tutorin zu haben, der die Vorlesung selber erst vor zwei Semestern gehört hat.

\subsection{Was bringt mir das?}
Die Übungszettel und die dazugehörigen Übungsgruppen sind erfahrungsgemäß der Ort, an dem euch der Stoff der Vorlesung nahe gebracht wird und ihr anfangt zu verstehen, was eigentlich in der Vorlesung vor sich geht. Eigentlich ist die Übungsgruppe also dazu da, die Vorlesung mit euch nachzuarbeiten, an den Stellen, an denen nicht klar ist was passiert, Hilfe zu bieten, mit euch die Zettel zu besprechen und einfach nochmal eine andere Darstellung zu liefern. Leider kommt es viel zu oft vor, dass in der Übung nur die Aufgaben „runtergerechnet“ werden, man irgendwann nicht mehr aufpasst und am Ende auch nicht mehr weiß, als vorher. Wenn das passiert, fragt penetrant nach dem Sinn der Aufgabenstellung, nach dem Zusammenhang mit der Vorlesung oder nach was euch sonst noch so durch den Kopf geht. Das ist eure einzige Chance, noch etwas aus der Rechnerei zu lernen.

\subsection{Anmeldung}
Ganz häufig stellen sich anfangende Studierende die Frage, ob sie sich zu den Übungen anmelden müssen. Die Antwort ist meistens „ja“. In der Physik nutzt ihr dazu das physikinterne Übungsgruppensystem\footnote{\url{https://uebungen.physik.uni-heidelberg.de/uebungen/}} Das System ist leider nicht besonders leistungsfähig. Da kann es schon mal vorkommen, dass es zu Zeiten, in denen die Übungsgruppenanmeldung großer Vorlesungen freigeschaltet werden, abkackt. Das ist aber auch kein Drama, und insbesondere kein Grund, das Rektorat zu alarmieren\footnote{Wann das wohl passiert ist…}, denn nach einigen Minuten hat sich das System dann auch wieder gefangen. Die Mathe geht wie immer ihren eigenen Weg, und hat mit dem Müsli\footnote{Mathematisches Übungsgruppen- und ScheinListenInterface:\\ \url{https://muesli.mathi.uni-heidelberg.de}} ihr eigenes System. Das ist deutlich leistungsfähiger, deutlich besser und viel schöner, aber man kann als Physikerin ja nicht alles haben. Die Informatik wiederum nutzt in 90\% aller Fälle das Moodle\footnote{\url{https://elearning2.uni-heidelberg.de/}}, das E-Learning-System der Uni Heidelberg. 90\% aller Nutzerinnen sind sich einig, dass dieses System ganz doll Grütze ist, aber davon hat sich die Informatik noch nie beeinflussen lassen.
