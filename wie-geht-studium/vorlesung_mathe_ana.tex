\subsection{Analysis I}
\label{ana1}

Die Analysis I (Ana I) muss von den Mathematikerinnen gehört werden, jedoch sind auch Physikerinnen und Informatikerinnen nicht unbedingt schlecht beraten. Hier lernt ihr richtige, fundierte Mathematik in all ihrer Schönheit und Abstraktion (was sehr abstrakt sein kann). Diese Vorlesung hat mit dem Matheunterricht aus der Schule ähnlich viel gemeinsam wie mit dem Sportunterricht.

Ihr lernt das formell richtige Argumentieren und Beweisen und erhaltet einen Einblick darin, was das Gebäude der Mathematik eigentlich ausmacht und wie dieses aufgebaut ist. Inhaltlich beginnt sie mit der Konstruktion der reellen Zahlen, führt über Folgen und Reihen zur Stetigkeit von (reellen) Funktionen und schließlich zur Differential- und Integralrechnung. Der Arbeitsaufwand für die Vorlesung schwankt (je nach Prof) zwischen lächerlich und immens, auch hier sind zehn oder mehr Stunden für einen Zettel nicht unbedingt Seltenheit (Es lohnt sich Zettelgruppen zu bilden und gemeinsam zu rechnen). Was das unglaublich frustrierende und hilflose Gefühl betrifft, man würde nichts verstehen und wäre völlig falsch in seinem Studiengang, keine Angst, das haben alle. Wenig härtet so gut gegen Frust ab wie eine erbarmungslose Mathevorlesung im ersten Semester. Aber lasst Euch davon nicht täuschen, dass Begriffe, die ihr meint aus der Schule zu kennen, unnötig umständlich eingeführt werden. Es hat alles durchaus seine mathematische Berechtigung und schafft ein Fundament, auf das ihr später aufbauen werdet. Sich ein sauberes, formelles Vorgehen und Denken anzugewöhnen ist unabdingbar.
