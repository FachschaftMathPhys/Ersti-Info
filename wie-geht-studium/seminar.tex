\section{Seminare}

In eurem Studium werdet ihr früher oder später über Seminare und Proseminare stolpern, in der Mathe und Info früher und mehr als in der Physik. Seminare dienen dazu, dass ihr euch selbst ein Thema erarbeitet und dann lernt, dieses vorzustellen.

Seminare fangen immer mit der Seminarvorbesprechung an. Alle Studentinnen, die an dem Thema des Seminars interessiert sind, treffen sich dort. Es werden die verschiedenen Vortragsthemen von der Seminarbetreuerin vorgestellt. Im Anschluss werden diese an die verschiedenen Teilnehmerinnen verteilt.

Sobald ihr also euer Thema habt, habt ihr (je nach Thema) zwei Wochen bis ein Semester Zeit, um euch vorzubereiten. Ihr lest also das entsprechende Kapitel in dem Buch oder dem Paper, nach dem das Seminar gehalten wird, oder beschäftigt euch sonst mit dem Thema. Im Unterschied zu Vorlesungen arbeitet ihr euch selber in das Thema ein, müsst euch Zusammenhänge selber überlegen und seid dann selbst gefordert, denn ihr müsst das zuvor Gelernte selber an die Tafel und in einen Votrag bringen.

Falls ihr irgendwo in eurer Vorbereitung über Dinge stolpert, die ihr nicht versteht, oder falls irgendwo Fragen auftauchen, die ihr selber nicht beantwortet bekommt, könnt ihr euren Seminarbetreuerin fragen, denn genau das ist ihr Job.

\subsection{Wie komme ich an ein Seminar?}

Die Suche beginnt für Physikerinnen in der Vorlesungsfreien Zeit, für Mathematikerinnen und Informatikerinnen bereits während dem alten Semester \-- zum Teil in der Klausurenphase. Die Seminarvorbesprechungen in der Physik finden in der ersten Woche des neuen Semesters statt. In der Mathe und Informatik liegen diese zum Teil auch in der letzten Woche des alten Semesters, daher müsst ihr euch rechtzeitig über das Seminarangebot des nächsten Semesters informieren.

Zunächst schaut ihr ins LSF. Dort stehen bis zum Anfang des Semesters alle angebotenen Seminare. Um früher Informationen über die Vorbesprechungen zu bekommen ist es in der Mathe und Info nötig zusätzlich an folgenden Orten zu suchen:

\begin{itemize}
	\item an Aushängen im Erdgeschoss des Mathematikons
	\item im Müsli
	\item auf Webseiten der Dozenten, die das gewünschte Fachgebiet abdecken.
\end{itemize}

Nachdem ihr wisst, wann diese stattfindet, geht ihr in die Vorbesprechung. Dort wählt ihr ein Thema aus und bekommt einen Termin für den Vortrag.

\subsection{Proseminar oder Seminar?}
Für viele Leute ist ein Seminar das erste Mal, dass sie an der Tafel stehen.  Weil in Seminaren der Inhalt doch stark im Vordergrund steht, gibt es in der Mathe und Informatik zusätzlich zu Seminaren auch noch sogenannte Proseminare. Der Unterschied ist der folgende: Seminare sind dazu da, euch ein bestimmtes Thema, das für eine Vorlesung zu speziell wäre und zu wenig Leute interessiert, nahe zu bringen. Diese Themen sind meistens relativ komplex und ihr müsst euch selber mit dem Inhalt beschäftigen, selber Gedanken machen und quasi selber „Mathe“ (oder ähnliches) produzieren. Der Fokus liegt also klar auf der Mathe bzw. Informatik und wenig auf eurem Vortrag. In Proseminaren hingegen sollte der Fokus auf dem Vortrag, und weniger auf dem Inhalt liegen. Der ist natürlich auch wichtig, und man sollte nicht erwarten, dass ein Proseminar inhaltlich einfach wird, aber meistens sind auch schwere Proseminare weniger anspruchsvoll als leichte Seminare. Nach eurem Vortrag wird dieser daher deutlich ausführlicher analysiert und nachbesprochen, als es bei einem Seminar der Fall wäre.

\subsection{Und wozu mache ich das?}
Für ein Seminar gibt es drei Gründe.

Der erste, und wohl offensichtlichste ist, dass es im Bachelor Pflicht ist.

Der zweite und wohl ebenso offensichtliche ist, dass Seminare die Möglichkeit bieten, Themengebiete kennen zu lernen, die in den Grundvorlesungen nicht vorkommen oder nur angeschnitten werden, oder sich in bestimmten Bereichen zu vertiefen.

Der dritte und nicht ganz so offensichtliche Grund ist, dass die meisten Profs Seminare nutzen um die Studentinnen kennenzulernen, die bei ihnen Bachelorarbeiten schreiben wollen. In der reinen Mathe geht das sogar so weit, dass es relativ unmöglich ist, eine Bachelorarbeit zu bekommen, ohne vorher ein Seminar gehört zu haben. Die angewandte Mathe sieht das nicht ganz so eng, aber auch da ist ein Seminar oder eine Vorlesung bei der entsprechenden Professorin empfehlenswert. In der Physik haben Seminare keinen so großen Stellenwert wie in der Mathe und Info, in den meisten Fällen hören Bachelorstudentinnen genau eines -- ihr Pflichtseminar. 
