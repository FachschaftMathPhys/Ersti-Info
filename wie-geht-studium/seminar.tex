\section{Seminare}

In eurem Studium werdet ihr früher oder später über Seminare und Proseminare stolpern, in der Mathe früher und mehr als in der Physik. Seminare dienen dazu, dass ihr euch selber ein Thema erarbeitet und dann lernt, das vorzustellen.

Seminare fangen immer mit der Seminarvorbesprechung an. Alle Leute, die an dem Thema interessiert sind, treffen sich und der Seminarbetreuer stellt die Themen der einzelnen Vorträge vor und danach werden sie an die Teilnehmer vergeben. Die Seminarvorbesprechungen in der Mathe finden meistens in der letzten Woche des alten Semester statt, in der Info und Physik in der ersten Woche des neuen Semesters.

Sobald ihr also euer Thema habt, habt ihr (je nach Thema) zwei Wochen bis ein Semester Zeit, um euch vorzubereiten. Ihr lest also das entsprechende Kapitel in dem Buch, nach dem das Seminar gehalten wird, oder beschäftigt euch sonst mit dem Thema. Im Unterschied zu Vorlesungen arbeitet ihr euch selber in das Thema ein, müsst euch Beweise oder Zusammenhänge selber überlegen und seid dann selbst gefordert, denn ihr müsst das zuvor Gelernte selber an die Tafel bringen.

Falls ihr irgendwo in eurer Vorbereitung über Dinge stolpert, die ihr nicht versteht, oder falls irgendwo Fragen auftauchen, die ihr selber nicht beantwortet bekommt, könnt ihr euren Seminarbetreuuer fragen, denn genau das ist sein Job.

\subsection{Proseminar oder Seminar?}
Für viele Leute ist ein Seminar das erste Mal, dass sie an der Tafel stehen. Weil in Seminaren der Inhalt doch stark im Vordergrund steht, gibt es in der Mathe zusätzlich zu Seminaren auch noch sogenannte Proseminare. Der Unterschied ist der folgende: Seminare sind dazu da, euch ein bestimmtes Thema, das für eine Vorlesung zu speziell wäre und zu wenig Leute interessiert, nahe zu bringen. Diese Themen sind meistens relativ komplex und ihr müsst euch selber mit dem Inhalt beschäftigen, selber Gedanken machen und quasi selber „Mathe“ produzieren. Der Fokus liegt also klar auf der Mathe und wenig auf eurem Vortrag. In Proseminaren hingegen sollte der Fokus auf dem Vortrag, und weniger auf dem Inhalt liegen. Der ist natürlich auch wichtig, und man sollte nicht erwarten, dass ein Proseminar inhaltlich einfach wird, aber meistens sind auch schwere Prosenimare weniger anspruchsvoll als leichte Seminare. Nach eurem Vortrag wird dieser daher deutlich ausführlicher analysiert und nachbesprochen, als es bei einem Seminar der Fall wäre.

\subsection{Und wozu mache ich das?}
Für ein Seminar gibt es drei Gründe.

Der erste, und wohl offensichtlichste ist, dass es im Bachelor Pflicht ist.

Der zweite und wohl ebenso offensichtliche ist, dass Seminare die Möglichkeit bieten, Themengebiete kennen zu lernen, die in den Grundvorlesungen nicht vorkommen oder nur angeschnitten werden, oder sich in bestimmten Bereichen zu vertiefen.

Der dritte und nicht ganz so offensichtliche Grundist, dass die meisten Profs Seminare nutzen um die Studierenden kennenzulernen, die bei ihnen Bachelorarbeiten schreiben wollen. In der reinen Mathe geht das sogar so weit, dass es relativ unmöglich ist, eine Bachelorarbeit zu bekommen, ohne vorher ein Seminar gehört zu haben. Die angewandte Mathe sieht das nicht ganz so eng, aber auch da ist ein Seminar oder eine Vorlesung beim entsprechenden Prof empfehlenswert. In der Physik haben Seminare keinen so großen Stellenwert wie in der Mathe, in den meisten Fällen hören Bachelorstudenten genau eines -- ihr Pflichtseminar.
