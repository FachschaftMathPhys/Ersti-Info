\section{Durchgefallen -- Was tun?}

In eurem Studium wird es aller Voraussicht nach hin und wieder vorkommen, dass ihr durch die eine oder andere Klausur durchfallt. Das ist an und für sich auch kein großes Problem. Zuerst einmal ist es aber wichtig, zu wissen, was es eigentlich heißt, „durchgefallen“ zu sein.

Die meisten Leute setzen „durchgefallen“ mit „die Klausur nicht bestanden haben“ gleich. Das ist nicht ganz falsch, aber auch nicht ganz richtig. Ihr müsst nämlich zwischen Klausur und Prüfungsleistung\footnote{ Der Unterschied zwischen Prüfungsleistung und Prüfungsversuch besteht primär sprachlich, beide bezeichnen meistens Klausur und Nachklausur als Einheit} unterscheiden. In fast allen Vorlesungen werden eine Klausur und eine Nachklausur geschrieben, die zusammen als ein Prüfungsversuch zählen. Erst wenn Ihr Klausur \emph{und} Nachklausur nicht bestanden habt, habt ihr die entsprechende Prüfungsleistung nicht bestanden. Ihr habt dann die Möglichkeit, die Prüfungsleistung zu wiederholen. Dazu hört ihr das Modul dann einfach nochmal, und habt dann ein weiteres Mal Klausur und Nachklausur vor euch. Besteht ihr beide nicht, könnt ihr einen „Härtefallantrag“ stellen, und versuchen, vor dem Prüfungsausschuss zu begründen, warum ihr vier Klausuren nicht bestanden habt. In der Physik habt ihr zusätzlich in zwei Modulen einen dritten Prüfungsversuch.\footnote{Das nennt sich dann in der Physik „Jokerregelung“} Wie genau das geregelt ist, steht an entsprechender Stelle im Modulhandbuch. In der Informatik habt ihr die Chance in zusätzlich vier Modulen einen dritten Prüfungsversuch durchzuführen.

\subsection{Orientierungsprüfung}
Eine Ausnahme von dieser Regelung stellt die sogenannte \emph{Orientierungsprüfung} dar. Diese Prüfung soll feststellen, ob Ihr überhaupt geeignet seid, das Fach, das ihr studiert, zu studieren. Diese Prüfung müsst ihr, im Gegensatz zu allen anderen Prüfungen, bis spätestens zum Ende des dritten Semesters erbracht haben, und die Jokerregelung gilt hier nicht.

\subsection{Muss ich nochmal Zettel rechnen?}
Das kommt auf den Dozenten deiner Veranstaltung an. Es gibt einige, die finden, dass man die Zettel nicht nochmal rechnen muss, einige wollen, dass du die Zulassung noch mal neu erwirbst. So oder so ist es unglaublich hilfreich, die Zettel trotzdem noch mal zu rechnen. Zwei mal die gleiche Veranstaltung bei zwei verschiedenen Professoren ist eben doch ein Unterschied, und dass du Nachholbedarf hast, hast du ja schon gezeigt ;)
