\section*{Vorlesungen Informatik}

\subsection{Einführung in die Praktische Informatik}
\label{info1}
Die \vl{Einführung in die Praktische Informatik} (\gls{IPI}) ist für alle Informatik- und Mathe-Bachelors verpflichtend. Einstieg bilden einige Grundstrukturen und Abläufe in der Informatik, die im Verlauf dann angewendet werden müssen, um gegebene Probleme zu lösen. Das passiert dann meist mit einem C++-Programm, wobei aber auch das Denken in informatischen Strukturen immer mitschwingt. Idealerweise hat man am Ende genug Herangehensweisen angehäuft, um Aufgaben vor dem geistigen Auge zu modellieren und später in richtigen Code umsetzen zu können. Vor allem Mathematikerinnen sollten die Vorlesung nicht auf die leichte Schulter nehmen, auch wenn der geringe Aufwand dazu verleitet. Spätestens mit der \vl{Einführung in die Numerik} (\gls{Num0}) muss wieder programmiert werden -- euch darum drücken könnt ihr also nicht. Auch für Lehrämtlerinnen kann es interessant sein, die Vorlesung zu hören, um sich später den Einstieg in die Numerik zu vereinfachen.

\subsection{Programmierkurs}
\label{ipk}
Der \vl{Programmierkurs} (\gls{IPK}) ist für alle Informatik-Studentinnen eine Pflichtveranstaltung. Je nach Angebot wird er als Vorlesung mit zwei Semesterwochenstunden oder als Blockkurs in den Ferien gelesen. Im Gegensatz zur praktischen Informatik lernt ihr hier keine informatischen Konzepte, sondern ihr erlernt das Programmieren in der Sprache C++. Trotzdem sollte man diesen Kurs nicht zu sehr auf die leichte Schulter nehmen, denn die Klausur wird an einem PC mit Linux durchgeführt und eure Programme müssen korrekt laufen. In der Regel lernt ihr zunächst die grundlegenden Datentypen, Operationen und verschiedene Kontrollstrukturen, und am Ende wird objektorientiert programmiert, z.B.~mit Vererbung, Templates und den Methoden aus der Standard-Bibliothek.

\subsection{Einführung in die Technische Informatik}
\label{info2}
Die \vl{Einführung in die Technische Informatik} (\gls{ITI}) ist eine Pflichtvorlesung für alle Informatik-Studentinnen. Hier lernt ihr Konzepte aus der binären Logik, mit denen Prozessoren arbeiten. Nebenbei beschäftigt ihr euch mit verschiedenen Schaltungen und mit Rechenschemata in verschiedenen Zahlensystemen (z.B. Binärsystem, Umwandlung vom Oktal- ins Hexadezimal-System und umgekehrt, Addition und Multiplikation auf diesen Systemen usw.). Im letzten Drittel der Vorlesung lernt ihr etwas Allgemeines zu Rechnerarchitekturen. Insgesamt ist es für das erste Semester eine sehr schöne Vorlesung, die euch sanft ins Informatik-Studium einführt.


