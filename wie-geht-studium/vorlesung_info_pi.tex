\section{Vorlesungen}

\subsection{Einführung in die Praktische Informatik}
\label{info1}
Die Einführung in die Praktische Informatik (\gls{IPI}) ist für alle Informatik- und Mathe-Bachelors verpflichtend. Einstieg bilden einige Grundstrukturen und Abläufe in der Informatik, die im Verlauf dann angewendet werden müssen um gegebene Probleme zu lösen. Das passiert dann meist mit einem C++-Programm, wobei aber auch das Denken in informatischen Strukturen immer mitschwingt. Idealerweise hat man am Ende genug Herangehensweisen angehäuft um Aufgaben vor dem geistigen Auge zu modellieren und später in richtigen Code umsetzen zu können. Vor allem Mathematikerinnen sollten die Vorlesung nicht auf die leichte Schulter nehmen, auch wenn der geringe Aufwand dazu verleitet. Spätestens mit der „Einführung in die Numerik“ (\gls{Num0}) muss wieder programmiert werden -- euch darum drücken könnt ihr also nicht. Auch für Lehrämtlerinnen kann es interessant sein, die Vorlesung zu hören, um sich später den Einstieg in die Numerik zu vereinfachen.
