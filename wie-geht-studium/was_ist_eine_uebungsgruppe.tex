%\newpage\Large\mathphyssubsubsec{Lehramt Informatik}\small
\section{Was ist eine Übungsgruppe?}
\sidebar{
    \centering
    \includegraphics[width=3cm]{bilder/Doom_1.png}\\\vspace{1.8mm}
    \includegraphics[width=3cm]{bilder/Doom_2.png}\\\vspace{1.8mm}
    \includegraphics[width=3cm]{bilder/Doom_3.png}\\\vspace{1.8mm}
    \includegraphics[width=3cm]{bilder/Doom_4.png}
}

Zu jeder Anfängervorlesung in Mathe, Physik und Informatik finden einmal wöchentlich sogenannte Übungsgruppen statt, an denen ihr teilnehmen solltet. Sie dauern anderthalb Stunden und werden in der Regel von Studierenden höherer Semester geleitet. Eine Ausnahme bilden lediglich die Übungen zur Experimentalphysik I bis VI.

Hauptzweck der Übungen ist es, die wöchentlich in den Vorlesungen verteilten Aufgabenblätter zu besprechen. Diese Aufgaben gebt ihr ein paar Tage vor der Übung bei Eurem Übungsgruppenleiter zur Korrektur ab. Die Aufteilung in die einzelnen Übungsgruppen erfolgt in der ersten oder zweiten Semesterwoche. Wenn ihr am Ende des Semesters genügend Aufgaben richtig gelöst habt, dürft ihr die Klausur mitschreiben und erhaltet bei Bestehen selbiger einen Leistungsnachweis (einen Schein und einen Eintrag im Onlinesystem\footnote{https://lsf.uni-heidelberg.de -- Unter „Meine Funktionen“, „Prüfungsverwaltung“ \label{fn:hispos}}) und die Sache ist geritzt. Für richtig gelöste Aufgaben gibt es Punkte. Die zu erreichende Punktzahl ist in Mathe und Physik etwas unterschiedlich und abhängig von den Vorgaben des Profs. In der Regel müssen 50\% (Mathevorlesung) -- 60\% (Physik- und Informatikvorlesung) der Übungsaufgaben richtig gelöst werden, um für die Klausur zugelassen zu werden.

%Sowohl in der Physik, als auch in der Mathematik wurden trotz
%Protesten von Seiten der Fachschaft Klausuren eingeführt und sind
%inzwischen weitgehend akzeptiert. 		%% <-dieser Satz war 2013 schon sehr alt.
Sowohl in der Physik, als auch in der Mathematik und Informatik gibt es Klausuren, die über (Nicht-) Bestehen entscheiden und auch das Gros der Note ausmachen. Die für den Schein relevante Mindestpunktzahl wird von dem/der jeweiligen ProfessorIn festgelegt (und manches Mal nach unten korrigiert). Seit ein paar Jahren sammelt die Fachschaft neben Prüfungsberichten auch alte Klausuren, die bei der Vorbereitung auf eine solche hilfreich sein kann.

Zu den Aufgaben allgemein: Man sollte versuchen, diese möglichst in Gruppenarbeit mit anderen Studierenden zu lösen. Stures Abschreiben ist wenig sinnvoll, obwohl der Zeitaufwand gerade in Analysis und Lineare Algebra, enorm ist. Alles alleine rechnen zu wollen ist aber auch eher nicht angesagt. Dafür ist die Zeit viel zu kostbar. Gruppenarbeit ist deshalb sehr hilfreich, da Mensch sich alleine (Robinson Crusoe Methode) oft in Sackgassen rechnet. Verlorene Vorzeichen verwirren sehr gründlich und werden halt doch am schnellsten von daran Unschuldigen bemerkt.


Außerdem müsst ihr in den Übungsgruppen die von Euch abgegebenen Aufgaben zum Teil an der Tafel vorrechnen. Lasst Euch in den ersten Wochen nicht entmutigen. Man muss kein Genie sein, um am Ende des Semesters seine Scheine zusammenzubekommen. Wenn Euer Punktekonto dann am Semesterende die ca. 60\% überschritten hat, und ihr die Klausuren erfolgreich absolviert habt, bekommt ihr einen kleinen, weißen Zettel mit Stempel und Unterschrift, der bescheinigt (deswegen heißen die so), dass ihr den Stoff beherrscht, doch der Schein trügt. Aber das werdet ihr dann am allerbesten wissen. Je nach Abschlussziel kann oder muss man die „Scheine“ auch online im LSF\footref{fn:hispos} einsehen. In der Physik haben sich aufgrund der Bachelor-Umstellung inzwischen ECTS-Scheine durchgesetzt. Auf denen findet ihr neben der Note noch ein Ranking, das Euch sagt, ob ihr zu den Genies, Normalbegabten oder Faulen eures Semesters gehört.
