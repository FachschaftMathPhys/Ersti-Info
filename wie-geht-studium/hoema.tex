\label{mathephysik}

Die Mathematikausbildung für PhysikerInnen sieht vor, dass ihr die Lineare Algebra I im ersten Semester hört. Laut Modulhandbuch kann man sich dann vor dem zweiten Semester entscheiden, ob man mit Analysis II und III (was auch die MathematikerInnen hören) oder mit Höhere Mathematik für Physiker (HöMa) II und III weitermacht.\\

Was spricht für Höhere Mathematik für Physiker (HöMa)?\\
Die Vorlesung ist extra auf euch als PhysikerInnen zugeschnitten und legt ihren Schwerpunkt auf die Vorlesungen Analysis 1-3 in strafferer Form. Während Mathematikvorlesungen einer gewissen Freiheit unterliegen und es durchaus vorkommen kann, dass die/der DozentIn einen Schwerpunkt auf ihr/sein Forschungsgebiet legt, hört ihr in HöMa größtenteils auch nur jene Dinge, die in der Physik auch verwendet werden. Außerdem kommen Beispiele gerne aus der Physik und liegen euch deshalb vielleicht näher. Trotzdem handelt es sich nicht um eine Schmalspurversion, sondern um eine vollwertige Mathematikvorlesung, die auch für zukünftige Theoretiker nicht ungeeignet ist.\\

Was spricht für die Analysis?\\
Die Analysis bietet als eine für MathematikerInnen konzipierte Veranstaltung eine präzisere Formulierung der Definitionen, Sätze und insbesondere Beweise. Somit wird es möglich die mathematischen Hintergründe in der Physik besser zu durchdringen und weitergehende Verbindungen der Gebiete zu erkennen. Dieses tiefere Verständnis kann unter anderem in der theoretischen Physik oder auch in weiterführenden Matheveranstaltungen von Vorteil sein und lässt euch insgesamt mehr Freiheiten im weiteren Studienverlauf, besonders bezüglich der Mathematik. Zudem ist je nach DozentIn und Forschungsbereich auch eine gewisse Schwerpunktlegung (vor allem in der Analysis III) möglich.\\

Auf den ersten Blick mag es verwundern, dass man in die Analysis II einsteigen soll, ohne die erste Vorlesung dazu gehört zu haben. Dies ist theoretisch zumindest möglich, jedoch vermutlich mit ein wenig Mehraufwand verbunden. Trotzdem können mathematisch Ambitionierte natürlich auch die Analysis I im ersten Semester hören, da diese eine schöne Einführung in den Themenbereich Analysis und die damit verbundenen Methoden darstellt. Das kann einem vor allem in weiterführenden Vorlesungen weiterhelfen; auch wird es eine große Erleichterung für die Analysis II sein, wenn man schon ein wenig mehr mit der Materie und dem Dozenten vertraut ist. Andererseits werdet ihr mit dem Kursprogramm auch so schon stark ausgelastet sein. Wenn ihr es mit vier Vorlesungen versuchen wollt, solltet ihr euch nach zwei bis drei Wochen entschieden haben, ob ihr das im ersten Semester durchhaltet oder nicht, da es für den Übungsbetrieb ziemlich blöd ist, wenn in der Mitte des Semesters viele Leute aussteigen.
