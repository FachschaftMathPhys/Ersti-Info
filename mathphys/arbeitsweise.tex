% !TEX ROOT = ../ersti.tex
\section[Studierendenvertretung in der Fachschaft]{Studierendenvertretung in der Fachschaft -- Wer vertritt meine Interessen?}

Verkürzt heißt die Antwort: Vertritt Dich selbst!

Das Wort „Studierendenvertretung“ als Beschreibung für die Arbeit der Fachschaft MathPhys ist ein wenig missverständlich, weil man meinen könnte, dass einmal im Jahr eine Vertretung der Studierenden gewählt wird, die dann schon die Arbeit macht. Offiziell ist das auch so, weil das Landeshochschulgesetz es so vorschreibt. Tatsächlich meint „Studierendenvertretung“ hier jedoch die Vertretung der Studierenden durch sie selbst -- uns alle.

\subsection{Offenheit der Fachschaft}

Deshalb passiert die meiste Arbeit in den Fachschaftssitzungen und in Arbeitskreisen. In einer Sitzung sind alle Studis des jeweiligen Fachs willkommen, redeberechtigt und können an jedem Entscheidungsprozess gleichberechtigt mitwirken.

Offiziell wird die Fachschaft auf zwei Wegen legitimiert. Zum einen legitimiert die jährliche Wahl von Fachschaftsmitgliedern in den Fakultätsrat die Arbeit der Fachschaft an und in der Fakultät. Zum anderen wird die Arbeit der Fachschaft im Rahmen der Verfassten Studierendenschaft und außerhalb der Fakultät durch die jährliche Wahl der Fachschaftsräte legitimiert, die formal bei allen Entscheidungen das letzte Wort haben. Allerdings sind seit ewigen Zeiten zu beiden Wahlen nur Leute aus dem aktiven Kreis der Fachschaft angetreten \dots denn tatsächlich zieht die Fachschaft ihre Legitimation aus ihrer basisdemokratischen Offenheit. Statt nur einmal im Jahr zur Urne zu trotten, kannst Du jede Woche bei der Fachschaftssitzung selbst mit entscheiden! Die Fachschaftsräte sind an die dort getroffenen Entscheidungen gebunden.

\subsection{Mandatierung der Gremienvertreterinnen}
Wer als Vertreterin der Fachschaft in Gremien gewählt wird, ist nicht nur dem eigenen Gewissen verpflichtet, sondern der ganzen Fachschaft. Diese diskutiert alle Themen, die im Gremium wahrscheinlich relevant werden, findet eine Fachschaftsposition, und beauftragt die Gewählten, diese Position zu vertreten. Die Gremienvertreterinnen werden somit von der Fachschaftssitzung meistens imperativ mandatiert.

%\newpage

\subsection{Arbeitskreise und Arbeitsgruppen der Fachschaft}

In zwei großen Fakultäten mit Studis vom ersten Semester bis zur Doktorarbeit gibt es natürlich äußerst verschiedene Interessen, und jede setzt ihre eigenen Prioritäten. Das ist super, weil daraus total verschiedene Aktivitäten entstehen -- vom Spieleabend bis zur Vortragsreihe über IT-Sicherheit finden diverse Ideen im Fachschaftsumfeld engagierte Mitstreiterinnen für ihre Umsetzung.

Damit nicht die ganze Fachschaftssitzung darüber diskutieren muss, ob man neben Brettspielen auch Rollenspiele anbietet, gibt es verschiedene Arbeitskreise (AKs) der Fachschaft, die bei Bedarf in den Sitzungen berichten. Wenn eine Entscheidung oder Geld benötigt wird, entscheidet die Sitzung. Dort werden auch AK-Gründungen bekanntgegeben.

\subsection{Entscheidungsfindung in der Fachschaft: Die Vielfalt der Interessen aushandeln}

%\sidebar{\parbox{3.5cm}{%Infobox Konsensstufen
\vspace*{-2em}

\subsubsection*{Konsensstufen}
\footnotesize
Wenn die Konsensstufen in einer Fachschaftssitzung abgefragt werden, kannst du
durch die Zahl ausgestreckter Finger an deiner Hand anzeigen, wie Du zu einem
Vorschlag stehst.
\begin{description}
\item[1. Vorbehaltlose Zustimmung]
Ich finds gut so.
\item[2. Zustimmung trotz leichter Bedenken]
Ich habe ein wenig Sorge, aber halte den Vorschlag grundsätzlich für eine gute Idee.
\item[3. Enthaltung]
Ich möchte nicht an der Entscheidung mitwirken, aber trage deren Ergebnis mit.
\item[4. Beiseitestehen]
Ihr könnt das für euch so entscheiden, aber ich mach da nicht mit.
\item[5. Schwere Bedenken]
Kein Konsens! Ich finde den Vorschlag echt scheiße.
\item[Faust = VETO]
Kein Konsens! Dieser Vorschlag widerspricht meinen Prinzipien. Entweder wird er
unterlassen, oder wir müssen ab jetzt getrennte Wege gehen, möglicherweise
sogar gegeneinander arbeiten.
\end{description}
}} %das passt leider nicht

Die verschiedenen Interessen und Prioritäten der Studis sind aber auch eine Herausforderung. Die Fachschaft lebt davon, dass die verschiedensten Studis ihre Gedanken und ihre Position einbringen und vertreten. Es gibt also a priori keine „Fachschaftsmeinung“, es gibt nur die Fülle der Einzelmeinungen, die durch alle Anwesenden wieder verändert wird. Die verschiedensten Ansichten zu einer gemeinsamen Position und Handlungsweise zusammenzubringen, ist dann die anspruchsvolle Aufgabe der Fachschaftssitzung.

Methodisch verwenden wir dabei ein Konsensstufensystem. Das heißt, zuerst werden alle angehört, die etwas zur Diskussion beizutragen haben. Angesichts dessen werden Vorschläge gemacht, wie man in der Frage vorgehen könnte. Deren Vor- und Nachteile werden abgewogen. Schließlich macht jemand einen Vorschlag für das Vorgehen der Fachschaft, den diese Person als konsensfähig einschätzt. Zu diesem werden die Konsensstufen (vgl. Infobox!) unter den Anwesenden abgefragt.

Alle, die keine vorbehaltlose Zustimmung angezeigt haben, können sich dann dazu äußern, warum sie nicht einfach zustimmen. Leichte Bedenken bis größere Vorbehalte (Konsensstufe 2"=4) können so bei der Umsetzung des Vorschlags bedacht und berücksichtigt werden. Vielleicht ergibt sich daraus auch ein noch besserer Vorschlag.

Wenn kein Konsens erreicht wurde (Konsensstufe 5 oder Veto), wird die Diskussion unter Berücksichtigung der Beweggründe fortgeführt, um auszuloten, ob nicht doch ein konsensfähiger Vorschlag möglich ist. Wenn das nicht der Fall ist, muss die Entscheidung entweder unterlassen werden, oder man geht ab da getrennter Wege. Letzteres ist noch nicht geschehen, seit wir im Mai 2012 das Konsensstufensystem eingeführt haben.
Bisher ist immer eine für alle tragfähige Lösung gefunden worden. Wir sehen das als große Stärke des Konsensprinzips. Es schließt aus, dass jemand übergangen wird. Wie man vorgeht, wenn die Ansichten doch einmal unvereinbar sind, müsste besprochen werden, wenn es passiert. Wichtig ist hier nur, dass die Möglichkeit besteht, getrennter Wege zu gehen, wenn man feststellt, dass man politisch gegeneinander arbeiten möchte.

%\vfill
\vspace*{\parskip}\null

\noindent\framebox{\parbox{\textwidth}{%Infobox Konsensstufen
\vspace*{-2em}

\subsubsection*{Konsensstufen}
\footnotesize
Wenn die Konsensstufen in einer Fachschaftssitzung abgefragt werden, kannst du
durch die Zahl ausgestreckter Finger an deiner Hand anzeigen, wie Du zu einem
Vorschlag stehst.
\begin{description}
\item[1. Vorbehaltlose Zustimmung]
Ich finds gut so.
\item[2. Zustimmung trotz leichter Bedenken]
Ich habe ein wenig Sorge, aber halte den Vorschlag grundsätzlich für eine gute Idee.
\item[3. Enthaltung]
Ich möchte nicht an der Entscheidung mitwirken, aber trage deren Ergebnis mit.
\item[4. Beiseitestehen]
Ihr könnt das für euch so entscheiden, aber ich mach da nicht mit.
\item[5. Schwere Bedenken]
Kein Konsens! Ich finde den Vorschlag echt scheiße.
\item[Faust = VETO]
Kein Konsens! Dieser Vorschlag widerspricht meinen Prinzipien. Entweder wird er
unterlassen, oder wir müssen ab jetzt getrennte Wege gehen, möglicherweise
sogar gegeneinander arbeiten.
\end{description}
}}

\subsection{Die schöne Theorie und die reale Umsetzung}

Dir wird bestimmt einiges an der hier vorgestellten Arbeitsweise komisch vorkommen, vielleicht hast Du auch Einwände. Wir laden Dich zunächst ein, es Dir einfach mal anzuschauen! Wenn dann noch Diskussionsbedarf und Kritik an der Arbeitsweise besteht, kann und soll dies unbedingt angesprochen werden. Nichts von dem, was hier beschrieben wurde, ist in Stein gemeißelt; unsere Arbeitsweise wird ständig an unsere Erfordernisse angepasst.

Wir sind uns dessen bewusst, dass partizipative Demokratie zeitlich viel Einsatz verlangt. Solange wir die Studiengänge nicht so umgestaltet kriegen, dass allen genug Zeit dafür bleibt, ist das auch ein Legitimationsproblem. Durch Transparenz über aktuelle Themen und Entscheidungen schaffen wir Abmilderung. Doch bisher bleibt unsere Öffentlichkeitsarbeit hinter unserem Anspruch zurück. Da Fachschaftsarbeit freiwillig und ehrenamtlich ist, wird eben genau das gemacht, wofür sich jemand (ggf. mit ein bisschen Erwartungsdruck\dots) freiwillig Zeit nimmt. Wenn Du Lust darauf hast, bist Du herzlich dazu eingeladen, die Öffentlichkeitsarbeit zu verbessern! Wir freuen uns, und es gibt immer Leute, von denen man hier einiges lernen kann.

Natürlich sind Basisdemokratie und Konsensprinzip auf dem Papier leichter, als in der Realität. Man ist daran gewöhnt, dass Konflikte nicht durch Aushandlung eines Konsens gelöst, sondern durch Abstimmung entschieden werden. Das heißt, man hat wenig Übung darin, Kompromisse zu finden, und viel Übung darin, die eigene Meinung ohne Abstriche durchzusetzen, was Gift für die Konsensfindung ist. Doch wir haben bisher die Erfahrung gemacht, dass das Konsensprinzip, obwohl es zunächst unproduktiver wirkt, wesentlich bessere Ergebnisse hervorbringt, und dadurch mit ein bisschen Übung effizienter ist als der Abstimmungsmodus.

Vielleicht mag es auf Dich so wirken, als seien alle in den Sitzungen ein eingeschworener Haufen, die eigentlich eh die gleichen Meinungen vertreten. Aber genau das versuchen wir zu verhindern, der Konsens wird von jeder Person gebildet, die ihre Meinung äußert. Egal in welche Richtung Deine Meinung geht, Du kannst sicher sein, dass sie zu einer Diskussion und einer neuen Meinungsbildung führt. Die Fachschaft sind alle, die sich einbringen wollen, mit all ihrer Persönlichkeit.

Dominanz mancher Leute in der Sitzung ist und bleibt ein Problem. Informationsgefälle versuchen wir durch kurze Inputs um 18 Uhr \gls{s.t.} vor der Sitzung abzubauen. Nachfragen ist erlaubt und erwünscht! Damit nicht einzelne die ganze Diskussion bestimmen, führen wir bei Bedarf Redelisten, wobei Leute vorgezogen werden, die erst wenig gesagt haben. Wenn Du Unbehagen mit Redeverhalten oder Diskussionsweise empfindest, teile das der Sitzung mit! Generell appellieren wir an unsere Reife. =)

Wir alle sind Lernende in fairem demokratischem Verhalten.

\subsection{Feedback zu Arbeitsweise \& Diskussionskultur}

Da Du als Ersti einen recht unverklärten Blick auf die Fachschaftssitzung hast, sollst Du Gelegenheit bekommen, Deine Eindrücke zu äußern. Damit wir Kritik, Anregungen und Streitfragen zur Arbeitsweise der Fachschaft in Ruhe gemeinsam besprechen können, soll nach ein paar Sitzungen die Diskussion darüber in der Fachschaftssitzung geführt werden. Den Bedarf dafür kannst Du einfach in einer Sitzung anmelden, dann machen wir einen Termin dafür fest.

Haben wir Dein Interesse geweckt? Gut, dann sehen wir uns am Mittwochabend um 18:00 Uhr!

\begin{center}
\large
\textbf{Fachschaftssitzungen}

\textbf{jeden Mittwoch um 18 Uhr \gls{c.t.}}
\end{center}
