%Infobox Konsensstufen
\vspace*{-1em}

\subsubsection*{Konsensstufen}
\footnotesize
Wenn die Konsensstufen in einer Fachschaftssitzung abgefragt werden, kannst du durch die Zahl ausgestreckter Finger an deiner Hand anzeigen, wie du zu einem Vorschlag stehst.
\begin{description}
\item[1. Vorbehaltlose Zustimmung] Ich finds gut so.
\item[2. Zustimmung trotz leichter Bedenken] Ich habe ein wenig Sorge, aber halte den Vorschlag grundsätzlich für eine gute Idee.
\item[3. Enthaltung] Ich möchte nicht an der Entscheidung mitwirken, aber trage deren Ergebnis mit.
\item[4. Beiseitestehen] Ihr könnt das für euch so entscheiden, aber ich mach da nicht mit.
\item[5. Schwere Bedenken] Kein Konsens! Ich finde den Vorschlag echt scheiße.
\item[Faust = VETO] Kein Konsens! Dieser Vorschlag widerspricht meinen Prinzipien. Entweder wird er unterlassen, oder wir müssen ab jetzt getrennte Wege gehen, möglicherweise sogar gegeneinander arbeiten.
\end{description}
