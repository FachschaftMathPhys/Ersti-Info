\section{Informatik-Vollversammlung}
Da die Meisten von Euch vermutlich nicht in jede Fachschaftssitzung kommen werden, wir aber doch gerne hin und wieder mal von euch allen hören wollen, was am Studiengang verbesserungswürdig wäre, veranstalten wir einmal im Semester die Informatik-Vollversammlung.

Dort erfahrt ihr von uns zunächst, was wir im letzten Semester so verbrochen haben und was wir so für die nächsten Semester planen. Danach sammeln wir Themen, über die ihr gerne reden möchtet. Diese können beliebig breit gefächert sein, von Vorschlägen zu unserer Arbeit über Beschwerden zu Vorlesungen bis hin zur Planung von Aktionen sind euch keine Grenzen gesetzt. Anschließend teilen wir uns in Arbeitskreise auf, in denen diese Themen besprochen werden. Dank des etwas kleineren Rahmens könnt ihr dort konstruktiv arbeiten und euch bei Kaffee und Kuchen austauschen.

Wenn dort irgendwelche Dinge besprochen werden, die von der Studienkommission in den Studiengang eingearbeitet werden sollten oder eine Veranstaltung betreffen, die dringend Verbesserung bedarf, besprechen wir die anonymisierten Protokolle der Arbeitskreise in der Fachschaftssitzung und geben uns alle Mühe, die Änderungsvorschläge so gut und schnell wie möglich umzusetzen.

Nach der Vollversammlung wird je nach Jahreszeit meist noch gegrillt oder ein bisschen Glühwein getrunken, gemütlich beisammengesessen und gequatscht.
