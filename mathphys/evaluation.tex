% !TEX ROOT = ../ersti.tex
\section{Lehrevaluation}
\label{eval}



\noindent Neben dem \hyperref[kummerkasten]{Kummerkasten} hast Du gegen Mitte des Semesters mit der \emph{Evaluation} noch die Möglichkeit, deinen Profs und Tutorinnen Feedback zu geben. Das läuft so ab, dass jemand von der Fachschaft (in der Physik) oder Assisteninnen des Profs (in der Info und Mathe) in die Vorlesung kommt, eine kleine Einleitung gibt und einen Bogen austeilt, den Du dann ausfüllst. In den Tutorien der Info und Mathe übernehmen die Tutorinnen die Evaluation, in der Physik ist die Tutorienevaluation Teil des Vorlesungsevaluationsbogen. Das Ganze dient zweierlei Zwecken: Die Studienkommissionen bekommen Einsicht in die Evaluationsergebnisse, sodass besonders schlechte (oder gute) Veranstaltungen mit den Dozentinnen nachbesprochen erden, um das Lehrniveau zu steigern bzw. zu halten. In der Physik wird auch regelmäßig ein Lehrpreis vergeben \footnote{\url{https://mathphys.fsk.uni-heidelberg.de/w/lehrpreise/}}. Zum Anderen werden in der Physik die Ergebnisse auch veröffentlicht (falls die Dozentin Ihre Einverständnis gegeben hat), sodass du nach passenden Vorlesungen und Tutorien suchen kannst. Du findest die Ergebnisse der Physik der letzten beiden Semester z.B. im Fachschaftsraum, sowie im \gls{KIP}-Foyer, im Arbeitsraum und am Philosophenweg am Infobrett im Treppenhaus.

\vfill
\eject

\begin{figure}[h]
\centering
        \includegraphics{bilder/eval_1.png}\\
        \includegraphics{bilder/eval_2.png}\\
        \includegraphics{bilder/eval_3.png}\\
        \includegraphics{bilder/eval_4.png}\\
        \includegraphics{bilder/eval_5.png}\\
\end{figure}

\vfill
