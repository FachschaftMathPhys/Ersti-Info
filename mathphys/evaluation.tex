%\section{Lehrevaluation}
\newpage\mathphyssecnobar{Lehrevaluation}
\label{eval}
\marginpar{
    \centering{
%        \vspace{20mm}
        \includegraphics{bilder/eval_1.png}\\
        \includegraphics{bilder/eval_2.png}\\
        \includegraphics{bilder/eval_3.png}\\
        \includegraphics{bilder/eval_4.png}\\
        \includegraphics{bilder/eval_5.png}\\
    }
}

\noindent Neben dem \hyperref[kummerkasten]{Kummerkasten} hast Du gegen Ende des Semesters mit der \emph{Evaluation} noch eine Möglichkeit, Deine Profs und TutorInnen zu bewerten. Das läuft dabei so ab, dass jemand von der Fachschaft (in der Physik) oder von der ZUV (in der Mathe) in die Vorlesung kommt, eine kleine Einleitung gibt und dann einen Bogen austeilt, den Du dann ausfüllst. In den Tutorien der Mathe übernehmen die Tutoren die Evaluation.  Das ganze dient zweierlei Zwecken: Zum Einen werden besonders schlechte (oder gute) Vorlesungen mit den Dozierenden nachbesprochen um das Lehrniveau zu steigern bzw. zu halten. Zum Anderen können Dir die Ergebnisse auch helfen, die richtigen Vorlesungen oder Übungsgruppen zu wählen. Wenn der/die DozentIn sein/ihr Einverständnis zur Veröffentlichung gegeben hat, dann findest Du die Ergebnisse der Physik der letzten Semester z.B. am Fachschaftsraum (im Infoständer neben den Zettelkästen im ersten Stock), sowie im \gls{KIP}, im Hörsaalgebäude der Physik (\Gls{INF} 308) und am Philosophenweg, jeweils an Infobrettern.
