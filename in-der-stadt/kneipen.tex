% !TEX ROOT = ../ersti.tex
\newcounter{zahl}
\newcommand{\place}[4]{\item[(\stepcounter{zahl}\thezahl) #1](#2)\\ #3\\\emph{Preis:} #4}

% Indikator für Frauenanteil? Nein!

\section{Bars, Kneipen \& Diskotheken}
$\star$ teuer\quad $\star\ \star$ noch teurer \quad $\star\ \star\ \star$ extrem teuer

% trim=l b r t
%\hspace*{-6mm}
\begin{figure*}
\altstadtkarte
\end{figure*}

\subsection{In der Altstadt:}
\begin{description}

%   \place{Alfredo}{Untere Straße}{Wirklich sehr leckere Pizza, der Chef sorgt für den echt italienischen Flair.}{$\star$}

    \place{Destille}{Untere Straße 16}{Besondere Shots, die jede in Heidelberg mal probiert haben sollte (\emph{Warmer Erpel}~\&~\emph{Gehängter}). Selbst an Tagen, an denen die Untere Straße leer ist, tanzen hier Leute auf den Tischen.}{$\star\ \star$}

    \place{Sonderbar\,/\,Betreutes Trinken}{Untere Straße 13}{Jede nur erdenkliche Form von Absinth, auch viel guten Rum und Whisky. Immer ordentlich was auf die Ohren (Hard\,\&\,Heavy). Keine Angst vor dem Wirt, einfach nicht auf den Mund gefallen sein. Oft sehr voll und vollgeräuchert.}{$\star\ \star$}

    \place{Eckstein}{am Fischmarkt 3}{Abgefahrene Kneipe. Je nach Wochentag ändert sich das Programm. Es gibt jedoch immer einen Kicker und reichlich Platz. Drei Mal wöchentlich Zaubershows.}{$\star\ \star$}

    \place{Mohr}{Untere Straße 3}{Spät abends meist so voll, dass man gar nicht mehr rein kommt. Drinnen wird dafür allerdings auf den Tischen getanzt. Donnerstags gibts zur Ladies’ Night kostenlosen Sekt für die Damen.}{$\star\ \star$}

    \place{Palmbräugasse}{Hauptstraße 185}{Hier gibts das selbstgebraute Palmbräu. Palmen gehören zwar nicht typisch zu Heidelberg, aber die Schnitzel in der Palm\-bräu\-gasse.}{$\star\ \star\ \star$}

    \place{Reichsapfel \& Lager}{Untere Straße 35}{Sehr geräumig. Moderner Vorderbereich und urigere Atmosphäre im hinteren Teil, welcher über den Innenhof zugänglich ist. Dort findet man oft Platz, wenn sonst alles voll ist.}{$\star\ \star$}

    \place{Mels}{Heiliggeiststraße 1}{Gewölbekeller, in dem seit Jahren die selbe Musik läuft, aber zumindest weiß man dann, was einen erwartet. Haben unter der Woche immer sehr gute Spezialangebote z.B. dienstags 123-Party (Bier \EUR{1}, Weizen \EUR{2}, Cocktails \EUR{3}).}{$\star\ \star$}

    \place{Cave 54}{Krämergasse 1}{Deutschlands ältester Jazzkeller. Kostet am Wochen\-ende Eintritt, hat dafür allerdings noch nach 3 Uhr geöffnet.}{$\star\ \star$}

    \place{Coyote Café}{Hauptstraße}{Einer der Orte, um eine Kneipentour durch die Altstadt starten zu lassen. Weizenbier, Cocktails und Shots sind brauchbar und brauchen nicht ewig. Am späteren Abend gibt es häufig eine Happy Hour, bei der Cocktails nur die Hälfte kosten.}{$\star\ \star$}

    %\place{Hard Rock Cafe}{Hauptstraße 142}{Montags Bier für \EUR{1}, ab 18 Uhr Cocktails für \EUR{4}. Musik wie man es erwartet, durchgehend Rock.}{$\star$}
    \place{Ben's Burgerbar}{Hauptstraße 142}{Hieß früher mal Hard Rock CafeAb 18 Uhr Cocktails für \EUR{4}. Musik wie man es erwartet, durchgehend Rock.}{$\star$}

%    \place{Havanna}{Neckarstaden 24}{Cocktailbar mit Möglichkeit zum Salsa tanzen.}{$\star\ \star\ \star$}

    \place{Hemmingways}{Fahrtgasse 1}{Hier lässt es sich das gesamte Jahr draußen sitzen, dank warmen Decken und Heizstrahlern. Außerdem kann man wunderbar den Neckar beobachten.}{$\star\ \star$}

    \place{Karl}{Lauerstraße 7-9}{Kneipe mit Billardtisch und Dartscheibe. Manchmal mit Live-Musik.}{$\star\ \star$}

    \place{Karlstorbahnhof}{Am Karlstor}{Richtig gute Diskothek (im Gebäude gibts auch Theater, Lesungen etc. -- viel Kultur) mit sehr variabler Musik. Was zum tanzen und weniger zum trinken, denn die Preise können sich meistens sehen lassen, genauso der Eintritt.}{$\star\ \star\ \star$}

    \place{Marstall}{Marstallhof}{Keine typische Kneipe, vielmehr Mensa mit Bier. Trotzdem gut geeignet, um sich zu treffen, zum Vorglühen und entscheiden, wo die weitere Party ihren Anfang nehmen soll.}{$\star$}

    \place{Maxbar}{Marktplatz 5}{Schöne Kneipe, bei der man tagsüber auf dem Marktplatz sitzen kann.}{$\star\ \star$}

    \place{Medoc}{Bismarckplatz}{Café Restaurant, das für ca.\,\EUR{5} aufwärts wechselnde Mittagsgerichte anbietet. Man kann draußen sitzen und den Betrieb auf dem Bismarckplatz beobachten.}{$\star\ \star$}

    \place{Orange}{Ingrimmstraße 26a}{Eine Kneipe wie ein Wohnzimmer. Eng aber gemütlich. Bietet sehr leckeres Bier aus Tschechien an, ist aber leider eher verraucht. Es gibt sogar Brettspiele.}{$\star\ \star$}

    \place{Regie}{Theaterplatz}{Riesenauswahl an Cocktails, die nach Filmen benannt und meist recht schick dekoriert sind. Cooles Specials- und Aktionensystem und leckere Flammkuchen.}{$\star\ \star\ \star$}

    \place{Tangente}{Kettengasse 23}{Hoher Juristinnenanteil und teils ältere Menschen. Türsteher und Gesichtskontrolle, dafür aber kein Eintritt. Hier kann man bis in die frühen Morgenstunden tanzen, man sollte allerdings keine Platzangst haben.}{$\star\ \star$}

    \place{Vater Rhein}{Untere Neckarstraße 20}{Legendär für seine \EUR{2,20}-Spaghetti bis zwei Uhr. Hier lässt sich der Abend gemütlich ausklingen. Stammkneipe vieler Stammtische in Heidelberg.}{$\star$}

	\place{Vetters}{Steingasse 9}{Nicht zum bleiben, aber für die Maß to go. Gibts da für \EUR{3} (+ Pfand), schmeckt hervorragend. Ansonsten gutbürgerliche Küche, älteres Klientel und viele Touristen. Brauen das Bier mit dem höchsten Stammwürzegehalt der Welt (33\%).}{$\star\ \star$}

    \place{Dubliners}{Hauptstraße 93}{Irish Pub mit Karaoke und Quiz-Nights (donnerstags).}{$\star\ \star\ \star$}

    \place{Shooters / Shooter Stars}{Heugasse 1}{Shotbar mit einer Auswahl von mehr als 300 verschiedenen Shots. Hier kann man beispielsweise eine „Nachklausur“ bestellen oder einen Shot-Bachelor machen.}{$\star\ \star$}

    \place{Metropol}{Kettengasse 21}{Billard-Kneipe mit günstigen Cocktails.}{$\star$}

    \place{Boho}{Kettengasse 11}{Bar in Neon-Optik. Hier werden überwiegend Charts gespielt.}{$\star\ \star$}
\end{description}



%%%%%%%%%
\subsection{In den Stadtteilen:}
\begin{description}

    \place{Bar 133}{Wohnheim \gls{INF} 133}{Wohnheimsbar, eigentlich nur für Bewohner der 1xx Wohnheime. Mittwochs und sonntags geöffnet mit gutem Angebot, günstigen Cocktails und Tischkicker.}{$\star$}

	\place{Comabar}{Comenius-Haus}{Bar des Comenius-Hauses direkt am Bunsengymnasium. Dienstags und donnerstags geöffnet, super Team, günstige Cocktails, Kicker und Tischtennisplatte vorhanden. Definitiv empfehlenswert.}{$\star$}

%    \place{Breidenbach Studios}{Hebelstraße 18}{Absoluter Hipster-Laden: Ehemalige Gasflaschenhandlung, die zu einem Künstlerhaus und Coworking-Space umgebaut wurde. Hier finden immer wieder großartige Partys statt.}{$\star\ \star\ \star$}

    \place{Cappuchino}{Bergheimer Straße 8}{Hippe Mischung von Kaffeehaus mit lautem Elektro.}{$\star\ \star$}

    \place{Gilberts Goldener Adler}{Handschuhsheimer Landstraße 96}{Ist eigentlich ein Restaurant, hat im Sommer aber auch einen netten Biergarten. Die Portionen sind groß und lecker. Stammlokal einiger Matheprofs.}{$\star\ \star$}

    \place{Halle 02}{Bahnstadt}{Electro-Freunde werden hier ihren Spaß haben, es gibt aber auch viele Mainstream Partys. Meistens sind diese auch recht voll und es herrscht gute Stimmung. Regelmäßig spielen hier bekanntere Bands.}{$\star\ \star$}

    \place{O'Reilly's}{Brückenkopfstraße 1}{Irish Pub mit Karaoke und Quiz-Nights.}{$\star\ \star\ \star$}

    \place{P11}{Am Römerkreis}{Nettes Café, das abends bis etwa eins Barbetrieb hat. Trotz der Nähe zum Römerkreis angenehme Atmosphäre. Geile Tapete! Hier kann man auch im Sommer draußen sitzen.}{$\star\ \star$}

    \place{Villa Nachttanz}{Im Klingenbühl 6}{Alternativer Kulturverein -- rechnet mit allem außer Mainstream. Sehr günstig, mit Lagerfeuer im Garten. Lohnt sich jedes mal.}{$\star$}

    \place{Ziegler}{Bergheimer Straße 1}{Kneipe mit Disco, welche allerdings Eintritt kostet und auch sonst recht hohe Preise hat. Oft auch Live Musik}{$\star\ \star\ \star$}

    \place{Zwitscherstube}{Blumenstraße 25}{Urige Kneipe mit Alt, Kölsch und original Underberggürtel. Hier kommen vor allem Fußballfans auf ihre Kosten. Wenn kein Fußball läuft, kann man super Skat spielen.}{$\star$}
\end{description}