\section{Geschichte der Ruprecht-Karls-Universität Heidelberg}
\label{geschichte}
Die Ruperto Carola wurde im Jahre 1386 mit päpstlicher Genehmigung von Kurfürst Ruprecht I. als Ruprechts-Universität Heidelberg gegründet. Sie ist die Dritte im Heiligen römischen Reich deutscher Nation nach Prag und Wien, also die älteste in den Grenzen des heutigen Deutschlands. Aufgrund der Spaltung der Kirche war es nötig geworden, eine Ausbildung eigener Theologen zu ermöglichen, da Sorbonne-Absolventen nicht mehr im römischen Reich in kirchliche Dienste treten durften. Marsilius von Inghen, der Gründungsrektor, wegen der Kirchenspaltung aus Paris geflohen, eröffnet die Universität im Oktober 1386 mit einer feierlichen Messe. Die Anfänge der Universität sind durch erhebliche Raumprobleme gekennzeichnet: Kirchen und Klostersäle werden für Vorlesungen genutzt. Erst später können eigene Gebäude für die Lehre errichtet werden. Im Zuge der Universitätsgründung werden die Stiftsbibliotheken der umliegenden Klöster vereinigt, um eine einigermaßen solide Ausbildung zu gewährleisten; die Büchersammlung wird dabei kontinuierlich erweitert (zum Beispiel durch vererbte Bestände der Augsburger Handelsfamilie Fugger) und im 16. Jahrhundert zur Bibliotheca Palatina vereinigt.

1556 wird die Universität im Zuge der Reformation in eine evangelische Landeshochschule umgewandelt. Kurfürst Ottheinrich führt normale bürgerliche Kleidung statt der sonst üblichen geistlichen Tracht für Studierende ein. Auch die finanzielle Situation verbessert sich stark durch die Übertragung von Kirchengut an die Universität. Später wird die Universität sogar als calvinistische Hochschule im „deutschen Genf“ bezeichnet. So blüht sie bis 1618 auf, die Studierendenzahlen wachsen, auch wenn die Universität im Vergleich zu anderen deutschen Universitäten immer noch zu den kleinen zählt.

\marginpar{
    \centering{
        \vspace{-50mm}
        \includegraphics[width=3cm]{bilder/a_wise_man_once_said_1.PNG}\\\vspace{13mm}
        \includegraphics[width=3cm]{bilder/a_wise_man_once_said_2.PNG}\\\vspace{13mm}
        \includegraphics[width=3cm]{bilder/a_wise_man_once_said_3.PNG}\\\vspace{13mm}
        \includegraphics[width=3cm]{bilder/a_wise_man_once_said_4.PNG}\\\vspace{13mm}
        \includegraphics[width=3cm]{bilder/a_wise_man_once_said_5.PNG}\\\vspace{13mm}
        \includegraphics[width=3cm]{bilder/a_wise_man_once_said_6.PNG}\\\vspace{13mm}
    }
}

Während des 30-jährigen Kriegs wird die Universität ziemlich stark beschädigt und der Lehrbetrieb muss immer wieder unterbrochen werden bis sie schließlich 1652 wiedereröffnet wird. Doch der Frieden hält nicht lange an. Im Zuge des Pfälzer Erbfolgekrieges wird die ganze Stadt Heidelberg verwüstet. Die Bibliotheca Palatina wird als Kriegskostenersatz an den Papst verschenkt. 

%% der lehrbetrieb wurde eingestellt und eigentlich gabs die Uni nicht. Davon erholt sich die Universität nur sehr langsam.

Mitte des 18. Jahrhunderts beginnen sich die Naturwissenschaften zu etablieren. Zuerst als Teil der philosophischen Fakultät entsteht 1752 der Lehrstuhl für Mathematik und Experimentalphysik. Diese Entwicklung setzt sich 1802 mit dem Übergang Heidelbergs an Baden fort. Die Universität erweitert nach dem ersten Großherzog von Baden ihren Namen zu „Ruprecht-Karls-Universität Heidelberg“. Sie ist von jetzt an staatlich finanziert und wird komplett reorganisiert. Die Naturwissenschaftlich-Mathematische Fakultät trennt sich von der Philosophischen und entwickelt sich vor allem unter dem Einfluss von Robert Bunsen, Gustav Kirchhoff und Hermann von Helmholtz. Georg Wilhelm Friedrich Hegel lehrt zwei Jahre an der philosophischen Fakultät, die medizinische Fakultät zieht Patienten aus aller Welt an. Trotzdem wird Heidelberg vor allem als juristische Universität gesehen. Auch kommt es in dieser Zeit zu Entwicklungen in der Frauengleichstellung. Im Jahr 1895 promoviert Katharina Windscheid als erste Doktorandin in Heidelberg. 1900 wird Georgine Sexauer als erste Studentin immatrikuliert und 1923 wird Gerta von Ubisch als Professorin für Botanik habilitiert.

Im 20. Jahrhundert setzt sich dieser Trend fort. Heidelberg ist eine weltoffene und liberale Universität, an der man auch eine große Zahl von ausländischen Studierenden findet. Das interdisziplinäre Gespräch wird gesucht, was der Uni ihren typischen „Heidelberger Geist“ verleiht, der in der Weimarer Republik auch häufig als demokratischer Geist bezeichnet wird.

Trotzdem wird die Studierendenschaft im Verlauf des 20. Jahrhunderts immer radikaler und mit dem Erstarken des Nationalsozialismus werden viele Professoren und Professorinnen entlassen und Studierende aus politischen und rassistischen Gründen ausgeschlossen. Bei der Bücherverbrennung auf dem Universitätsplatz 1933 nehmen viele Mitglieder der Universität aktiv teil; die Ruperto Carola ist als braune Universität verrufen. Die Athene wich dem Hakenkreuz, der lebendige Geist dem deutschen. % Diese Gender-Form von ProfessorInnen wurde wegen dem geschichtlichen Kontext gewählt. 

Nach dem zweiten Weltkrieg ist die Universität äußerlich zerstört, gravierender sind jedoch die Schäden die durch die nationalsozialistische Ideologie verbleiben. Der inneren Erneuerung widmeten sich maßgeblich Karl Jaspers und Karl Heinrich Bauer, die eine neue Satzung ausgearbeitet und in den folgenden Jahren die Universität stark erweitern. 

Trotz aller wissenschaftlichen Erneuerung bleibt das reaktionäre Geflecht an der Universität erhalten - Athene hat ihren Platz zurück erobert, doch der lebendige freie Geist ist nicht zurück gekehrt. Gegen diese Verkrustungen begehrt die Studierendenbewegung Ende der 60er Jahre auf. Mit der polizeilichen Räumung von besetzten Universitätsgebäuden und Verhaftung von Studierenden findet diese Bewegung in Heidelberg wie auch ganz Deutschland ein jähes Ende. Als Reaktion und Bestrafung wurden studentische Mitbestimmungsrechte massiv beschnitten und seit dem nur teilweise wieder eingeführt (hier fehlt das - Verweis auf HOPO artikel).

Die Universität gewinnt stetig Studierende und wird mit der Erschließung des Neuenheimer Feldes stark erweitert. Bis zum Jubiläumsjahr 1986 wächst die Zahl der Studierenden auf ca. 27\,000 an. Heidelberg erarbeitet sich in Deutschland und darüber hinaus einen Ruf als forschungsstarke Universität.

Die neuesten Entwicklungen stammen aus dem Jahr 2007. Seitdem darf sich die Universität „Exzellenzuni“ nennen, denn ihr Zukunftskonzept „Heidelberg: Realising the Potential of a Comprehensive University“ wurde als förderungswürdig ausgewählt. 
