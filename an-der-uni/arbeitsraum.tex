\section{Arbeitsraum}
Seit einigen Semestern gibt es einen betreuten studentischen Arbeitsraum in der Physik.
Diesen haben wir vor allem für euch eingerichtet.
Das Tolle daran ist, dass drei TutorInnen eingestellt wurden, um Euch bei
Fragen und Problemen zur Verfügung zu stehen. Das läuft sehr gut und wir haben
viel positive Resonanz bekommen, besonders für die TutorInnen. Das Konzept 
ist eine Art Bedarftutorium. Das heißt die TutorInnen sind für Eure Fragen da, die
oft in den regulären Übungsgruppen keinen Platz haben.   Besonders oft verliert man nämlich im
Übungszetteldschungel den Überblick, was wirklich wichtig ist. Genau dabei
können einem aber oft ältere Studis helfen. Auch sollen sie dir helfen deine
Übungszettel besser zu lösen, indem sie hilfreiche Tipps geben. 

Der Raum ist während der Vorlesungszeit immer Montags bis Donnerstags von 13
bis 19 Uhr und Freitags bis 17 Uhr geöffnet. Neben ein paar Standardbüchern gibt
es dort auch eine Kaffeemaschine und einen Wasserkocher.

Leider war zu Redaktionsschluss noch nicht klar, ob der Arbeitsraum im nächsten Semester im KIP im Foyer
oder am Philosophenweg sein wird, aber das bekommt ihr dann sicher durch Plakate mit.
