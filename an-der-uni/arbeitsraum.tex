\section{Arbeitsraum}
\label{sec:arbeitsraum}
Seit einigen Semestern gibt es einen betreuten studentischen Arbeitsraum in der Physik. Diesen haben wir vor allem für euch eingerichtet. Das Tolle daran ist, dass einige Tutorinnen eingestellt wurden, um Euch bei Fragen und Problemen zur Verfügung zu stehen. Das läuft sehr gut und wir haben bisher viel positive Resonanz bekommen, besonders für die Tutorinnen. Das Konzept ist eine Art Bedarfstutorium. Das heißt die Tutorinnen sind für Eure Fragen da. Besonders Fragen und Probleme, die in den regulären Übungsgruppen keinen Platz haben, könnt ihr hier diskutieren. Oft verliert man nämlich im Übungszetteldschungel den Überblick, was wirklich wichtig ist. Genau dabei können einem aber ältere Studis helfen. Auch sollen sie dir helfen deine Übungszettel besser zu lösen, indem sie dir hilfreiche Tipps geben. Klarerweise sollen sie dir keine Lösungen geben, sondern dich auf dem Weg dahin unterstützen.

Zu finden sind die Tutorinnen im KIP im 2. Stock im Foyer vor den Seminarräumen. Während der Vorlesungszeit sitzt dort immer Montags bis Donnerstags von 13 bis 19 Uhr und Freitags bis 17 Uhr eine Tutorin, die ihr mit Hilfe eines großen Plakats erkennen könnt. Neben ein paar Standardbüchern gibt es dort auch eine Kaffeemaschine und einen Wasserkocher, wo ihr Kaffee und Tee zu günstigen Preisen bekommen könnt.

Falls du also mal an einem Zettel verzweifelst, komm einfach vorbei und lass dir helfen.
