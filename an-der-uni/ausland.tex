% !TEX ROOT = ../ersti.tex
%\section{Studium im Ausland}
\newpage\mathphyssecnobar{Studium im Ausland}%FIXME
Man hört immer wieder von den tollen Erfahrungen und Möglichkeiten, die ein Studium im Ausland bietet. Die Leute schwärmen davon, wie schön es an der Uni so-und-so war. Aber wie kommt man überhaupt dahin?

In den Bachelor-Studienordnungen der Physik und der Mathe heißt es, „Bei der Anerkennung von Studienzeiten, Studien- und Prüfungsleistungen, die außerhalb Deutschlands erbracht wurden, sind die von Kultusministerkonferenz und Hochschulrektorenkonferenz gebilligten Äquivalenzvereinbarungen sowie Absprachen im Rahmen von Hochschulpartnerschaften zu beachten. Bei Zweifeln an der Gleichwertigkeit kann die Zentralstelle für ausländisches Bildungswesen gehört werden.“ Das ist nicht sonderlich hilfreich, wenn man sich nicht unbedingt durch Tonnen von Gesetzestext quälen will. Auf der Homepage der jeweiligen Fakultät findet ihr eine Liste von Unis, mit denen Austauschprogramme laufen (die also anerkannt werden), einige generelle Infos und ein paar Erfahrungsberichte, die ihr bei Interesse anschauen solltet. Die Gleichwertigkeit von Studienleistungen muss allerdings nicht von euch überprüft werden, die Beweispflicht liegt bei der Fakultät. Lasst euch also nicht beirren und hört im Zweifel die Veranstaltung im Ausland einfach, die sollen erst mal zeigen, dass das nicht das gleiche war.

Allerdings sollte ein Auslandssemester im Moment nicht euer erstes Problem sein -- während der ersten beiden Semester ist es ganz einfach fachlich nicht möglich. Bis zur Einführung der Bachelor- und Masterstudiengänge konnte man (wenigstens in der Mathe) an diesen Programmen nur nach abgeschlossenem Grundstudium teilnehmen, inzwischen wird der Zeitraum letztes Jahr Bachelor/erstes Jahr Master empfohlen. Das bedeutet, dass ihr nach dem zweiten Semester damit beginnen solltet, euch umzuschauen, insbesondere in Bezug auf Bewerbungsvoraussetzungen und -fristen.

AuslandsBAföG müsst ihr übrigens separat beantragen, die Zu\-stän\-dig\-keit dafür hängt vom jeweiligen Ausland ab, eine Liste der zuständigen Ämter findet man z.B. unter \url{AuslandsBAföG.de}. In Heidelberg kann man sich auch direkt an das Dezernat Internationale Beziehungen (früher Akademisches Auslandsamt, AAA)\footnote{\url{http://www.uni-heidelberg.de/studium/kontakt/auslandsamt/}} wenden, das ihr in der Seminarstraße 2 findet. Ihr müsst dann ins Zimmer 139.

\paragraph{Öffnungszeiten} Mo. -- Do. 10 -- 15 \qquad Fr. 10 -- 13

Für das Urlaubssemester ist die Studierendenadministration in der Seminarstraße 2 zuständig. Mehr Infos und das Formular zum Urlaubssemester findet ihr unter:
https://www.uni-heidelberg.de/studium/imstudium/formalia/beurlaubung.html

