% !TEX ROOT = ../ersti.tex
\section{Studium im Ausland}
Man hört immer wieder von den tollen Erfahrungen und Möglichkeiten, die ein Studium im Ausland bietet. Die Leute schwärmen davon, wie schön es an der Uni so-und-so war. Aber wie kommt man überhaupt dahin?

Auf der Homepage der jeweiligen Fakultät findet ihr eine Liste von Unis, mit denen Austauschprogramme laufen, zum Beispiel über ERASMUS oder bilaterale Abkommen, die also üblicherwiese anerkannt werden, einige generelle Infos und ein paar Erfahrungsberichte, die ihr bei Interesse anschauen solltet. Die Gleichwertigkeit von Studienleistungen muss allerdings nicht von euch überprüft werden, die Beweispflicht liegt bei der Fakultät. Lasst euch also nicht beirren und hört im Zweifel die Veranstaltung im Ausland einfach, die sollen erst mal zeigen, dass das nicht das Gleiche war.

Allerdings sollte ein Auslandssemester im Moment nicht euer erstes Problem sein -- während der ersten beiden Semester ist es ganz einfach fachlich nicht möglich. Bis zur Einführung der Bachelor- und Masterstudiengänge konnte man -- jedenfalls in der Mathe -- an diesen Programmen nur nach abgeschlossenem Grundstudium teilnehmen, inzwischen wird der Zeitraum letztes Bachelor Jahr\,/\,erstes Master Jahr empfohlen. Das bedeutet, dass ihr nach dem zweiten Semester damit beginnen solltet, euch umzuschauen, insbesondere in Bezug auf Bewerbungsvoraussetzungen und -fristen.

Auslands-BAföG müsst ihr übrigens separat beantragen, das zu\-stän\-dige Amt dafür hängt vom jeweiligen Ausland ab\footnote{\url{https://www.auslandsbafoeg.de/}}. In Heidelberg kann man sich auch direkt an das Dezernat Internationale Beziehungen, früher Akademisches Auslandsamt, (AAA)\footnote{\url{http://www.uni-heidelberg.de/studium/kontakt/auslandsamt/}} wenden, das ihr in der Seminarstraße 2 findet. Ihr müsst dann ins Auslandsstudium Info-Zimmer 139\footnote{Öffnungszeiten: \auslandsinfooeff}.

Sehr wichtig ist noch zu wissen, dass man Urlaubssemester beantragen kann, damit die Fachsemesterzahl -- vor allem wichtig für alle, die Inlands-BAföG erhalten -- nicht weiter läuft. Dafür läuft die Hochschulsemesterzahl weiter. Man kann sich dann die im Ausland erworbenen \gls{LP} -- nach Absprache mit dem Prüfungssekretariat -- anrechnen lassen.

Für das Urlaubssemester ist die Studierendenadministration in der Seminarstraße 2 zuständig. Mehr Infos und das Formular zum Urlaubssemester findet ihr online\footnote{\url{https://www.uni-heidelberg.de/studium/imstudium/formalia/beurlaubung.html}}.

