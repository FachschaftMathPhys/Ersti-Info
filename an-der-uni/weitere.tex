% !TEX ROOT = ../ersti.tex
\section{Weitere Angebote}

\subsection{Upstream Mathematik-Mentorinnenprogramm}
Das Interdisziplinäre Zentrum für Wissenschaftliches Rechnen (IWR) an der Uni Heidelberg bietet seit 2013 mit Upstream ein Mentorinnenprogramm für Mathematik-interessierte Frauen. Upstream ist ein Netzwerk für junge Mathematikerinnen in allen Stufen der Ausbildung -- von Schülerinnen ab der 10. Klasse, Studentinnen, Doktorandinnen, Nachwuchswissenschaftlerinnen bis hin zu Professorinnen. In regelmäßigen Abständen organisieren wir Meet \& Greet-Treffen, Public Lectures, Diskussionsrunden und Workshops zu Themen, in denen Schlüsselqualifikationen und studiengangsspezifische Inhalte vermittelt werden. Dabei nehmen erfahrene Mathematikerinnen die Mentorinnenrolle ein. Sie stehen den Teilnehmerinnen zur Seite, beraten und beantworten Fragen rund um das Studium und dem Berufsleben.

Wenn Du Mitglied werden willst, schicke einfach etwa eine halbe Seite Motivationsschreiben per E-Mail an: upstream@iwr.uni-heidelberg.de. Die Aufnahme in das Programm ist nicht an Noten oder Wettbewerbserfolge geknüpft; Wir suchen vor allem begeisterte Mathematikerinnen auf allen Ebenen. Weitere Infos unter: \url{www.mathcomp.uni-heidelberg.de/programs/upstream}

\subsection{Beratungsangebote für behinderte Studierende}
Ganz viele Informationen zum Studium mit Behinderung findet ihr zum einen auf den Seiten der Uni\footnote{\url{http://www.uni-heidelberg.de/studium/kontakt/handicap/}}, zum anderen im Dschungelbuch des StuRa\footnote{\url{https://dschungelbuch.stura.uni-heidelberg.de/index.php/Behinderte_studieren}}.

\subsection{Kooperationsabkommen mit der Universität Mannheim}
Zwischen der Universität Heidelberg und der Universität Mannheim gibt es ein Kooperationsabkommen, was Studierenden in Mathematik, Informatik und Physik ermöglicht, kostenlos als Gaststudentin Vorlesungen an Fakultät für Wirtschaftsmathematik und Wirtschaftsinformatik zu hören und sich hier in Heidelberg anrechnen lassen. Unter anderem werden in Mannheim Mathematik-Vorlesungen in Kryptographie, angewandte Algebra, Spieltheorie usw. angeboten. Dieses Kooperationsabkommen bietet Vorteile für die Studierenden, die in der Nähe von Mannheim wohnen, und/oder um Vorlesungen zu hören, die es in Heidelberg nicht (mehr) gibt oder unregelmäßig angeboten werden. Informatiker können auch die dortigen Grundlagenvorlesungen in Mathematik hören und sich für das Grundstudium anrechnen lassen. Im Gegensatz zu Heidelberg fängt aufgrund der internationalen Ausrichtung der Uni Mannheim das Sommersemester als Frühling-Sommer-Semester Mitte Februar und das Wintersemester als Herbst-Winter-Semester Mitte September an. Auch in Mannheim muss für jede Vorlesung eine Klausurzulassung in Form von 50\% der Übungszettelpunkte erreicht werden. Jedoch zählt in Gegensatz zu unseren Prüfungsordnung eine Klausur als ein Prüfungsversuch. Dafür muss die erste Klausur nicht geschrieben werden, um die Nachklausur zu schreiben. Die Anmeldung zur Vorlesung als Gaststudentin ist zugleich die Anmeldung zu einer der beiden Klausur und wird beim Studienbüro 1 für Wirtschaftsmathematik und Wirtschaftsinformatik bei Herrn David Steiner durchgeführt. Sowohl erfolgreiche als auch fehlgeschlagene Prüfungsversuche werden nicht direkt an das Sekretariat in der Uni-Heidelberg geschickt, sondern müssen hier in Heidelberg im jeweiligen Prüfungssekretariat angerechnet werden. Hierfür werden die Prüfungsleistungen tabellarisch (ähnlich wie ein Transcript) sowohl per E-Mail als auch postalisch zugeschickt. Das Studienbüro 1 befindet sich im Quadrat L1, 1. 

\textbf{Wichtige Weblinks:} 
\begin{description}
\item[Anreise:] \url{http://www.uni-mannheim.de/1/service/anfahrt_lageplan/}
\item[Studienbüros:] \url{https://www.uni-mannheim.de/studienbueros/kontakt/}
\end{description}
