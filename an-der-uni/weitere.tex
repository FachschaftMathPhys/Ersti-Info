% !TEX ROOT = ../ersti.tex
\section{Weitere Angebote}

\subsection{Mathematik-Mentorinnenprogramm \\Upstream}
Das Interdisziplinäre Zentrum für Wissenschaftliches Rechnen (IWR) an der Uni Heidelberg bietet seit 2013 mit Upstream\footnote{\url{www.mathcomp.uni-heidelberg.de/programs/upstream}} ein Mentorinnenprogramm für Mathematik-interessierte Frauen. Upstream ist ein Netzwerk für junge Mathematikerinnen in allen Stufen der Ausbildung -- von Schülerinnen ab der 10. Klasse, Studentinnen, Doktorandinnen, Nachwuchswissenschaftlerinnen bis hin zu Professorinnen. In regelmäßigen Abständen organisieren wir Meet\,\&\,Greet-Treffen, Public Lectures, Diskussionsrunden und Workshops zu Themen, in denen Schlüsselqualifikationen und studiengangsspezifische Inhalte vermittelt werden. Dabei nehmen erfahrene Mathematikerinnen die Mentorinnenrolle ein. Sie stehen den Teilnehmerinnen zur Seite, beraten und beantworten Fragen rund um das Studium und das Berufsleben.

Wenn du Mitglied werden willst, schicke einfach etwa eine halbe Seite Motivationsschreiben per E-Mail an: upstream@iwr.uni-heidelberg.de. Die Aufnahme in das Programm ist nicht an Noten oder Wettbewerbserfolge geknüpft; Wir suchen vor allem begeisterte Mathematikerinnen auf allen Ebenen.

\subsection{Kooperationsabkommen mit der \\Universität Mannheim}
Zwischen der Universität Heidelberg und der Universität Mannheim gibt es ein Kooperationsabkommen, was Studierenden in Mathematik, Informatik und Physik ermöglicht, kostenlos als Gaststudentin Vorlesungen an Fakultät für Wirtschaftsmathematik und Wirtschaftsinformatik zu hören und sich hier in Heidelberg anrechnen lassen. 
Somit kann man das vielfältige Vorlesungsangebot nutzen und insbesondere Module wie Kryptographie, angewandte Algebra und Spieltheorie belegen, die so nicht in Heidelberg angeboten werden.