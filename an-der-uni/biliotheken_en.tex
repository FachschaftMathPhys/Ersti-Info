\newpage
\section{Bibliotheken/Library}
\begin{table*}
\centering
% Diese Tabelle wurde mühsam angeordnet. Bitte nicht umbrechen, sondern scrollen!
%~ \newcommand{\bibKIP}{}
\begin{tabular}{lll@{ -- }l@{\quad}r@{ -- }l@{ Uhr}}
\toprule
Name                                    & Adress                                                                                                                       & \multicolumn{4}{l}{"Opening Hours}    \\
\midrule
\multirow{5}{*}{\gls{UB} (Check-Out Counter)}    & \multirow{2}{*}{\href{http://www.openstreetmap.org/?mlat=49.40966&mlon=8.70594&zoom=17&layers=M}{Plöck 107-109 (Altstadt)}}   & Mo & Fr                    & 9    & 19 \\
                                        &                                                                                                                               & \multicolumn{2}{l}{Sa}     & 9    & 17 \\[-0.7\defaultaddspace]
                                        & \multicolumn{1}{c}{}                                                                                                                                                \\[-0.7\defaultaddspace]
                                        & \multirow{2}{*}{\href{http://www.openstreetmap.org/?mlat=49.41767&mlon=8.66836&zoom=17&layers=M}{INF 368, 3. floor (Feld)}}   & Mo & Fr                    & 9    & 19 \\
                                        &                                                                                                                               & \multicolumn{2}{l}{Sa}     & 9    & 13 \\
\cmidrule{1-6}
\multirow{5}{*}{\gls{UB} (Reading room)}    & \multirow{2}{*}{\href{http://www.openstreetmap.org/?mlat=49.40966&mlon=8.70594&zoom=17&layers=M}{Plöck 107-109 (Altstadt)}}   & Mo & Fr                    & 8:30 & 1 \\
                                        &                                                                                                                               & Sa & So                    & 9    & 1 \\[-0.7\defaultaddspace]
                                        & \multicolumn{1}{c}{}                                                                                                                                                \\[-0.7\defaultaddspace]
                                        & \multirow{2}{*}{\href{http://www.openstreetmap.org/?mlat=49.41767&mlon=8.66836&zoom=17&layers=M}{INF 368 (Feld)}}             & Mo & Fr                    & 8:30 & 22 \\
                                        &                                                                                                                               & Sa & So                    & 9    & 22 \\
\cmidrule{1-6}
\multirow{2}{*}{Mathematikon}           & \multirow{2}{*}{\href{http://www.openstreetmap.org/?mlat=49.41730&mlon=8.67580\#map=17/49.41730/8.67580}{INF 205}}            & Mo & Fr                    & 9    & 21 \\
                                        &                                                                                                                               & \multicolumn{2}{l}{Sa}     & 9    & 15 \\
\cmidrule{1-6}
\multirow{2}{*}{Physics}                 & \multirow{2}{*}{\href{http://www.openstreetmap.org/?mlat=49.41479&mlon=8.69686&zoom=17&layers=M}{Philosophenweg 16}}          & Mo & Fr                    & 9    & 12:30 \\
                                        &                                                                                                                               & \multicolumn{2}{l}{}       & 13:30    & 16:30 \\
\cmidrule{1-6}
\multirow{3}{*}{Public Library/Stadtbücherei}          & \multirow{3}{*}{\href{http://www.openstreetmap.org/?mlat=49.40638&mlon=8.6866&zoom=17&layers=M}{Poststraße 15}}               & Tu & Fr                    & 10   & 20 \\
                                        &                                                                                                                               & \multicolumn{2}{l}{Sa}     & 10   & 16 \\
                                        &                                                                                                                               & \multicolumn{4}{l}{Closed on Mondays!}\\
\cmidrule{1-6}
\multirow{2}{*}{Student Library/Studibücherei}          & \multirow{2}{*}{\href{http://www.openstreetmap.org/?mlat=49.41051&mlon=8.70518&zoom=17&layers=M}{Grabengasse 14}}               & Mo & Do                    & 11   & 17 \\
                                        &                                                                                                                            & \multicolumn{2}{l}{Fr}     & 11   & 14 \\
\cmidrule{1-6}
\multirow{2}{*}{Attention:}               & \multicolumn{4}{c}{\textbf{During the lecture free periods the opening hours are changed.}} \\                  
                                        & \multicolumn{4}{c}{\textbf{Please check the respective websites for reference.}} \\

\bottomrule
\end{tabular}

\end{table*}
Specialised literature is increadibly expensive most of the time. To avoid the complete ruin of the students, every university is equipped with an university library. In Heidelberg it is devided in three parts in compliance with the three sections of the university. The literature that's probably of the most interest to you, the one about Maths, Computer Science and Physics, is located in the branch office \gls{INF}~368, 3.~floor.

Besides books about physics, computer science and mathematics you can borrow medical literature there, as well as books for all the other faculties located in the Neunheimer Feld. In the main office of the university library at the Peterskirche, Plöck~107-109, and the Campus library in Bergheim on the other hand, you can find literature for the humanities and countless books for law and economy courses. Due to this trifold division, Heidelberg fairly early introduced a computer system („\gls{HEIDI}“\footnote{\url{http://katalog.ub.uni-heidelberg.de}} to the university library. Using Heidi and your university ID, which is written on your student ID card,you can borrow books. You can directly search for titles on the website and extend the loaning time of your borrowed books -- that might really save you an euro or two, because the fines for overdue books are pretty high. Many books are also downloadable as PDFs or are at least accessible online as e-books. At the start of each semester there are introductory courses to Heidi at the university library, but they are only reasonable to a limmited extend, because the system has a clear layout and is very self-explanatory with intuitive usability.

For example, you can check wether a book is available as e-book while searching for it. The respective search result contains an "online resource" marker in the bottom-left corner, if that is the case. On the respective page for the book there's a link "access online", or in german "online aufrufen", in the upper left corner which leads you to the online version after chosing "Universität Heidelberg" and logging in with your university ID. 
 
If you're looking for more specialised literature, you'll have to go to the department libraries. The department library for mathematics and computer science is located in building \-\gls{INF}~205 (on the ground floor in the east wing). Additional inforation and the opening hours can be foundon the website of the department library\footnote{\url{https://www.mathinf.uni-heidelberg.de/bib}}. Specialized physics literature are in the physics department library in Philosophenweg~16\footnote{\url{http://www.ub.uni-heidelberg.de/dezentral/bpa/standorte.html}}. The only disadvantage: The countless books and magazines can only be copied, not borrowed. If a book in the university library is already borrowed by someone else, it might be worth it to go to the public library ("Stadtbücherei") in Poststraße. It was extensivly renovated in 1993 and is since then equipped with a "new" computer system.To register you need nothing more than your ID card(not student ID). Since the rebovation you have to pay to borrow books, (\EUR{10}/Jahr).\\
Additinal information can be found on the website of the public library \footnote{\url{http://www.stadtbuecherei-heidelberg.bib-bw.de}}.

There's also the student library provided by the Studierendenwerk (student services)\footnote{\url{http://www.studentenwerk.uni-heidelberg.de/de/studibuecherei}}. The student library and the public library are especially well suited to with some relaxing read independent of your university studies.

