\section{Die kleine BAföG-Liste}
%\mathphyssecnobar{Die kleine BAföG-Liste}%FIXME

\paragraph{Was ist BAföG überhaupt?}
BAföG ist eine Abkürzung für \glqq Bundesausbildungsförderungsgesetz \grqq, das Gesetz, das in Deutschland die staatliche Unterstützung für die Ausbildung von Schülern und Studenten regelt. Ziel des BAföG ist es, jungen Menschen unabhängig von ihrem sozialen und finanziellen Hintergrund eine (schulische) Ausbildung oder ein Studium zu ermöglichen. Dafür kannst du einen \glqq Antrag auf Ausbildungsförderung \grqq stellen, um, wenn du bestimmte Voraussetzungen erfüllst, finanzielle Unterstützung zu erhalten. 

Wer von BAföG spricht, meint meistens nicht das Gesetz, sondern genau diese finanzielle Förderung.

\paragraph{Soll ich BAföG beantragen?}
JA! Ob du anspruchsberechtigt bist, ist für dich meistens nicht so einfach zu erkennen, deswegen stelle auf jeden Fall einen Antrag. Wenn dieser abgelehnt wird, hast du höchstens ein wenig Zeit verloren, wenn nicht, lohnt es sich für dich ziemlich sicher:

Für dich als Studi sind 50\% der Förderung Zuschuss, die andere Hälfte zinsloses Darlehen. Zurückzuzahlen ist dieses Darlehen frühestens fünf Jahre nach Studienabschluss, aber nur ab einer bestimmten Einkommenshöhe. Unter bestimmten Voraussetzungen (z. B. Versorgung eines Kindes) ist auch sonst einen Rückzahlungsaufschub möglich. Außerdem kann (z. B. bei besonders guten Studienleistungen oder frühzeitiger Rückzahlung) ein Nachlass auf die Darlehensschuld gewährt werden.

\paragraph{Wie viel gibt es und wie lange und ab wann?}
Für Studierende ohne Kinder gibt es bis zu \EUR{735}. Wie hoch der monatliche Förderungsbetrag ist, hängt ab von Mietverhältnis, Einkommen der Eltern, eigenem Vermögen, Krankenversicherung, Geschwistern, etc. Die Förderungsdauer richtet sich nach der Regelstudienzeit (Bachelor und konsekutiver Master = 6 + 4 Semester, aber es gibt fachspezifische Ausnahmen). Die Anspruchszeit zählt ab Immatrikulation und ist nicht aufsparbar! Die Auszahlung geschieht in der Regel erst, wenn der Bescheid bei euch eingegangen ist, bei dringender Bedürftigkeit ist ggf. ein Vorschuss (in voller BAföG-Höhe, der hinterher verrechnet wird) möglich.

\paragraph{Wo und bis wann muss ich den Antrag stellen?}
Das geht beim BAföG-Amt des Studierendenwerks im Marstallhof 3, und zwar Mo.\ -- Fr.,\ 8 -- 18 Uhr. Kurzberatung im Neuenheimer Feld gibt es Mo.\ -- Do.,\ 10.00 -- 17.00 Uhr und Fr.,\ 10.00 -- 15.00 Uhr. Das macht man am besten sobald man die Immatrikulationsbescheinigung hat, denn BAföG gibt es erst vom Antragsmonat an, wobei der frühestmögliche Termin der Semesteranfang ist.

\paragraph{Was muss ich beim Antrag beachten?}
Wichtig ist, den Antrag so früh wie möglich stellen, weil es teilweise zu langen Bearbeitungszeiten kommen kann. Der Antrag sollte idealerweise gleich vollständig sein, damit keine Rückfragen an dich nötig sind und der Antrag schneller bearbeitet werden kann. Ansonsten gilt: lieber schnell beantragen und Papiere nachreichen. Die Angaben sollten selbstverständlich wahrheitsgemäß sein, denn das Amt kann die Finanzdaten prüfen.

\paragraph{BAföG und nun? Was sonst noch zu beachten wäre.}
Man muss nach dem vierten Fachsemester seine Studienleistungen bescheinigen lassen! Wichtig ist außerdem,  familiäre und persönliche wirtschaftliche Veränderungen immer sofort zu melden (Schwester/Bruder in Ausbildung/Arbeit, Arbeitslosigkeit, etc.). Es gibt auch Auslands-BAföG und viele sonstige Ausnahmen, zu denen man sich am besten beraten lässt, denn es kann sich lohnen! Eigener Zuverdienst ist bis maximal \EUR{5\,416,32} Brutto im Jahr oder \EUR{451,36} im Monat möglich (auch hier gibt es Ausnahmen).\\[5mm]

\noindent Abschließend: Bei BaföG durchzublicken ist gar nicht so einfach, und auch uns ist das noch nicht hundertprozentig gelungen. Deswegen können sich auch in diesem Text Fehler eingeschlichen haben. Wir wollen an dieser Stelle aber auch gar keine allumfassende Anleitung für BaföG verfassen, sondern vor allem erreichen, dass du dich selbst informiert! Am sinnvollsten ist es, wenn du dich selbst mit dem Thema beschäftigst und dich beraten lässt.

Nutze hierzu die Sprechstunde vom Sozialreferat des \gls{StuRa} \footnote{\url{https://www.stura.uni-heidelberg.de/referate/soziales.html}}. Die helfen beim Antrag, erklären Ausnahmen und geben Tipps (um spätere Probleme zu vermeiden, bei denen sie natürlich auch helfen).
