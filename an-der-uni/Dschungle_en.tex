\section{What to do when you're having problems/Was tun bei Problemen}
\label{Jungle}
There you are: matriculated, found a place to stay (at least for now), maybe have already got acquainted with many offices and agencies. Time to get started with the "fun student life". Right?

But what does life at a new university look like? Everyone is talking about university fees -- do they concern me? When do I have to take my first exams? How can I combine my studies with a job?
%Aber was ist das eigentlich, studieren? Wie wird das Studium im ersten Semester aussehen? Wie lernt man denn an der Uni, wenn einem niemand so richtig vorschreibt, was wann wie zu machen ist? Überall hört man von Studiengebühren -- betrifft mich das eigentlich? Und wie ist das mit dem BAfÖG? Wann muss ich meine ersten Prüfungen ablegen? Wie kann ich Studium und Job unter einen Hut bringen? Bleibt neben dem Studium überhaupt noch Zeit, etwas anderes zu machen?

What's true for space can also be applied to the mental galaxie: Don't panic! You'll quickly acquire the overview you're lacking right now. And most important of all: You're not alone. The desire of this booklet is to introduce you to the most important places to go to. For example the General Guidance Counselling Office, Social and Legal Counselling Offices or the Psychosocial Counselling Office of the Studierendenwerk (Student Services)\footnote{Overview over the different counselling offers:\\\url{http://www.uni-heidelberg.de/studium/imstudium/beratung.html}}.


%Worüber ihr euch klar werden solltet, ist, was ihr euch vom Studium erwartet. Fragt euch also: Warum studiere ich? Welches Berufsziel schwebt mir vor? Warum gerade dieses Fach? Ihr müsst diese Entscheidungen für euch selbst treffen und Prioritäten setzen -- der Stundenplan bietet keine Anleitung dazu, die eigenen Fähigkeiten und Erwartungen im Studium umzusetzen.\enlargethispage{\baselineskip}%FIXME

%\begin{minipage}{1.5\textwidth}
%\hspace{-5mm}
%        \includegraphics[width=6cm]{bilder/dschungelbuch_1.png}
%        \hspace{10mm}
%        \includegraphics[width=6cm]{bilder/dschungelbuch_2.png}\\\vspace{5mm}
%\end{minipage}
