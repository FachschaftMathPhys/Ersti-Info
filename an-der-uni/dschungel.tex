\section{Was tun bei Problemen}
\label{dschungel}
Da seid ihr also: ordentlich immatrikuliert, auch schon eine Bleibe gefunden (zumindest vorläufig), möglicherweise bereits Bekanntschaft mit diversen Ämtern und Behörden geschlossen. Und nun kann es eigentlich losgehen mit dem „lustigen Studentinnenleben“. Oder?

Aber was ist das eigentlich, studieren? Wie wird das Studium im ersten Semester aussehen? Wie lernt man denn an der Uni, wenn einem niemand so richtig vorschreibt, was wann wie zu machen ist? Überall hört man von Studiengebühren -- betrifft mich das eigentlich? Und wie ist das mit dem BAfÖG? Wann muss ich meine ersten Prüfungen ablegen? Wie kann ich Studium und Job unter einen Hut bringen? Bleibt neben dem Studium überhaupt noch Zeit, etwas anderes zu machen?

Was im Weltall gilt, gilt auch in der geistigen Galaxie: Don't panic! Der Überblick, den ihr jetzt noch nicht habt, stellt sich noch ein. Vor allem seid ihr nicht allein. Ein Anliegen dieses Heftes ist es, euch die Anlaufstellen aufzuzeigen, wie z.\,B.\ die Zentrale Studienberatung, Sozial- und Rechtsberatung, oder die Psychosoziale Beratungsstelle des Studierendenwerks\footnote{Überblick über die einzelnen Beratungsangebote:\\\url{http://www.uni-heidelberg.de/studium/imstudium/beratung.html}}.

Worüber ihr euch klar werden solltet, ist, was ihr euch vom Studium erwartet. Fragt euch also: Warum studiere ich? Welches Berufsziel schwebt mir vor? Warum gerade dieses Fach? Ihr müsst diese Entscheidungen für euch selbst treffen und Prioritäten setzen -- der Stundenplan bietet keine Anleitung dazu, die eigenen Fähigkeiten und Erwartungen im Studium umzusetzen.\enlargethispage{\baselineskip}%FIXME

%\begin{minipage}{1.5\textwidth}
%\hspace{-5mm}
%        \includegraphics[width=6cm]{bilder/dschungelbuch_1.png}
%        \hspace{10mm}
%        \includegraphics[width=6cm]{bilder/dschungelbuch_2.png}\\\vspace{5mm}
%\end{minipage}
