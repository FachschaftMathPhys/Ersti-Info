%\section{Bibliotheken}
\mathphyssecnobar{Bibliotheken}%FIXME
\begin{table*}[tb]
\centering

% Diese Tabelle wurde mühsam angeordnet. Bitte nicht umbrechen, sondern scrollen!
%~ \newcommand{\bibKIP}{}
\begin{tabular}{lll@{ -- }l@{\quad}r@{ -- }l@{ Uhr}}
\toprule
Name                                    & Adresse                                                                                                                       & \multicolumn{4}{l}{"Offnungszeiten}    \\
\midrule
\multirow{3}{*}{\gls{UB} (Ausleihe)}    & \multirow{1}{*}{\href{http://www.openstreetmap.org/?mlat=49.40966&mlon=8.70594&zoom=17&layers=M}{Plöck 107-109 (Altstadt)}}   & Mo & Fr                    & 9    & 19 \\[-0.7\defaultaddspace]
                                        & \multicolumn{1}{c}{und}                                                                                                                                                \\[-0.7\defaultaddspace]
                                        & \multirow{1}{*}{\href{http://www.openstreetmap.org/?mlat=49.41767&mlon=8.66836&zoom=17&layers=M}{INF 368, 3. Stock (Feld)}}   & \multicolumn{2}{l}{Sa}     & 9    & 13 \\
\cmidrule{1-6}
\multirow{5}{*}{\gls{UB} (Lesesaal)}    & \multirow{2}{*}{\href{http://www.openstreetmap.org/?mlat=49.40966&mlon=8.70594&zoom=17&layers=M}{Plöck 107-109 (Altstadt)}}   & Mo & Fr                    & 8:30 & 1 \\
                                        &                                                                                                                               & Sa & So                    & 9    & 1 \\[-0.7\defaultaddspace]
                                        & \multicolumn{1}{c}{}                                                                                                                                                \\[-0.7\defaultaddspace]
                                        & \multirow{2}{*}{\href{http://www.openstreetmap.org/?mlat=49.41767&mlon=8.66836&zoom=17&layers=M}{INF 368 (Feld)}}             & Mo & Fr                    & 8:30 & 22 \\
                                        &                                                                                                                               & Sa & So                    & 9    & 22 \\
\cmidrule{1-6}

\multirow{2}{*}{Mathematikon}           & \multirow{2}{*}{\href{http://www.openstreetmap.org/?mlat=49.41730&mlon=8.67580\#map=17/49.41730/8.67580}{INF 205}}            & Mo & Fr                    & 8    & 21 \\
                                        &                                                                                                                               & \multicolumn{2}{l}{Sa}     & 9    & 16 \\
\cmidrule{1-6}
\multirow{2}{*}{Physik}                 & \multirow{2}{*}{\href{http://www.openstreetmap.org/?mlat=49.41479&mlon=8.69686&zoom=17&layers=M}{Philosophenweg 16}}          & Mo & Do                    & 9    & 19 \\
                                        &                                                                                                                               & \multicolumn{2}{l}{Fr}     & 9    & 17 \\
\cmidrule{1-6}
\multirow{3}{*}{Stadtbücherei}          & \multirow{3}{*}{\href{http://www.openstreetmap.org/?mlat=49.40638&mlon=8.6866&zoom=17&layers=M}{Poststraße 15}}               & Di & Fr                    & 10   & 20 \\
                                        &                                                                                                                               & \multicolumn{2}{l}{Sa}     & 10   & 16 \\
                                        &                                                                                                                               & \multicolumn{4}{l}{Montags geschlossen!}\\
\cmidrule{1-6}
\multirow{2}{*}{Studibücherei}          & \multirow{2}{*}{\href{http://www.openstreetmap.org/?mlat=49.41082&mlon=8.70733&zoom=17&layers=M}{Schulgasse 6}}               & Mo & Do                    & 11   & 17 \\
                                        &                                                                                                                            & \multicolumn{2}{l}{Fr}     & 11   & 14 \\
\bottomrule
\end{tabular}

\end{table*}

Fachliteratur ist meistens unglaublich teuer. Um den totalen Ruin der Studierenden zu vermeiden, hat jede Universität Bibliotheken. In Heidelberg ist die \gls{UB} dreigeteilt, und zwar entsprechend der Dreiteilung der Universität. Die für euch wahrscheinlich interessanteste Literatur über Mathe, Informatik und Physik befindet sich in der Zweigstelle \gls{INF}~368, 3.~Stock.

Neben Physik-, Informatik- und Mathebüchern können hier auch medizinische Literatur und Bücher zu allen anderen Fakultäten, die im Neuenheimer Feld untergebracht sind, ausgeliehen werden. Im Hauptsitz der \gls{UB} an der Peterskirche, Plöck~107-109, und der Campus-Bibliothek in Bergheim hingegen findet sich die geisteswissenschaftliche Literatur sowie zahlreichen Jura- und VWL- Büchern. Bedingt durch diese Dreiteilung wurde in Heidelberg sehr früh ein Computersystem („\gls{HEIDI}“ \footnote{\url{http://katalog.ub.uni-heidelberg.de}}) in der UB eingeführt. Mit Hilfe von Heidi und eurer Uni-ID, die auf eurem Studiausweis steht, kann dort die gängige Fachliteratur ausgeliehen werden. Übers Internet lässt sich auch direkt nach Büchern suchen, außerdem können ausgeliehene Bücher online verlängert werden -- das kann einem durchaus den ein oder anderen Euro sparen, die Mahngebühren bei überzogenen Fristen sind nämlich empfindlich teuer. Zu Semesterbeginn finden in der UB Einführungskurse in Heidi statt, die jedoch nur bedingt sinnvoll sind, weil das System sehr übersichtlich und intuitiv aufgebaut ist und sich größtenteils selbsterklärend bedienen lässt.

Wer weiterführende spezielle Fachliteratur sucht, muss sich an die Bereichsbibliotheken der einzelnen Fakultäten halten. Für Mathe- und Informatik-Literatur befindet sich die Bereichsbibliothek im Gebäude\-\gls{INF}~205 (im Erdgeschoss auf der Ostseite). Weitere Informationen und Öffnungszeiten finden sich auf der Webseite der Bereichsbibliothek\footnote{\url{https://www.mathinf.uni-heidelberg.de/bib}}. Spezielle Physikbücher stehen in der Physik-Bereichsbibliothek im Philosophenweg~16\footnote{\url{http://www.ub.uni-heidelberg.de/dezentral/bpa/standorte.html}}. Einziger Nachteil: Die zahlreichen Bücher und Spezialzeitschriften können hier nur zum Kopieren ausgeliehen werden. Wenn ein Buch in der UB ausgeliehen ist, dann lohnt es sich manchmal auch, zur Stadtbücherei in der Poststraße zu gehen. Sie wurde 1993 aufwendig umgebaut und besitzt seitdem ebenfalls ein „neues“ Computersystem. Zum Anmelden braucht man nur einen Personalausweis. Die Ausleihe ist seit dem Umbau allerdings nicht mehr kostenlos (\EUR{10}/Jahr).\\ Weitere Information kannst du auch direkt bei der Stadtbücherei einholen\footnote{\url{http://www.stadtbuecherei-heidelberg.bib-bw.de}}.

Vom Studierendenwerk gibt es zusätzlich noch die Studierendenbibliothek\footnote{\url{http://www.studentenwerk.uni-heidelberg.de/de/studibuecherei}}. Diese wie auch die Stadtbücherei eignen sich natürlich insbesondere auch dazu, neben der ganzen Studiererei einfach mal abzuschalten und zu schmökern.
