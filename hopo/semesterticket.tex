% !TEX ROOT = ../ersti.tex

\section{Semesterticket}
Seit mehr als 20 Jahren hat Heidelberg ein Semesterticket, das es den Studierenden ermöglicht kostengünstig den öffentlichen Nahverkehr zu nutzen. Zur Einführung waren alle Beteiligten von der großen Resonanz überrascht und das Ticket entwickelte sich zu einem Erfolg. Enorme Preissteigerungen von 127\% in der bisherigen Geschichte haben jedoch dazu geführt, dass die Nutzerzahlen deutlich sinken und das Semesterticket von vielen als unattraktiv und überteuert wahrgenommen wird.

Ein Semesterticket muss stets eine Vielzahl von Interessen befriedigen. Es sollte ein attraktives und günstiges Ticket im Stadtbereich sein und das direkte Umfeld des Hochschulstandortes erschließen. Des Weiteren ist es wünschenswert die ländliche Region anzubinden und den dort wohnhaften Studierenden einen Umzug und hohe Mieten zu ersparen. Neben dem täglichen Pendelverkehr ist die Heimreise zum elterlichen Wohnsitz ebenfalls für eine Vielzahl von Studierenden ein Grund ein Semesterticket zu erwerben.

Das aktuelle Semesterticket in Heidelberg wird vom Verkehrsverbund Rhein-Neckar (VRN) angeboten und gilt für ein Semester. Es berechtigt zu Fahrten im gesamten Tarifgebiet außer der Westpfalz\footnote{\url{http://www.vrn.de/vrn/einfach-ankommen/linienplaene/stadt-und-netzlinienplaene/gesamtliniennetzplan/index.html}} -- „einem Schlauch von Polen nach Frankreich“\footnote{Zitat aus der Semesterticket Umfrage der FSK im Jahr 2013}. Die Ausdehnung ist in Ost-West Richtung sehr weitreichend, in Nord-Süd Richtung ist jedoch nach 20\,km von Heidelberg aus Schluss. Das Semesterticket finanziert sich aus einem Kaufpreis von aktuell \EUR{\semesterticket} und einem solidarischen Sockelbeitrag von \EUR{\sockelbeitrag}, den alle Studierenden mit dem Studierendenwerksbeitrag bei der Rückmeldung zahlen müssen -- auch wenn sie das Ticket nicht nutzen. Eine Heidelberger Besonderheit bei der Sockelfinanzierung ist, dass sie es allen Studierenden ermöglicht ab 19 Uhr innerhalb der Großwabe Heidelberg sowie den Waben Dossenheim, Eppelheim und Leimen kostenlos mit Bus und Bahn fahren zu können -- der Studiausweis gilt dabei als Fahrschein.

\subsection{Konflikt ums Semesterticket}
\marginpar{
    \centering
    \vspace{1mm}
    \includegraphics[width=3cm]{bilder/straba.pdf}
}
Seit Herbst 2008 verhandeln das Verkehrsreferat der Studierendenvertretung und das Studierendenwerk mit dem VRN über einen neuen Vertrag. Die Studis fordern ein Ende der enormen Preissteigerungen, um auch in Zukunft mit dem Semesterticket ein günstiges Nahverkehrsangebot zu gewährleisten. Da der VRN jedoch weiter an Preissteigerungen von ca. 10\% pro Jahr festhalten will, stand das Semesterticket vor dem Aus. Nach einem Übergangsvertrag, auf den der VRN sich im WS 09/10 in letzter Minute eingelassen hat, wurde im WS 13/14 erneut verhandelt. Diesmal kam der Kommunalwahlkampf in Heidelberg den Studis zu Gute, aber trotz massiven Drucks aus der Politik und Zuschüssen durch den Gemeinderat ist das Semesterticket nach wie vor recht teuer und der VRN wird die Preise weiterhin regelmäßig erhöhen.

Mittlerweile steigt das Semesterticket im Preis jedes Semester. Der StuRa und vor allem dessen Verkehrsreferat setzt sich dafür ein, die Preise so gering wie möglich zu halten, was kein leichtes Unterfangen ist.

\subsection{Semesterticket vs. Fahrrad fahren}

Heidelberg ist eine eher kleine übersichtliche Stadt in welcher alles bequem mit dem Fahrrad erledigt werden können -- meist sogar schneller als mit Bus und Bahn. Dank der milden Temperaturen ist dies auch im Winter durchaus möglich. Wenn ihr in Heidelberg selbst wohnt ist daher gut zu überlegen ob sich ein Semesterticket lohnt oder man wie viele Andere das Fahrrad nutzt.

Ebenfalls wurde im Sommersemester 2016 eine Umfrage zu einem Sockelbeitrag von \EUR{2,40} für die Nutzung von NextBike (in ganz Deutschland, überall dort wo es NextBike gibt) gleichzeitig zu den Wahlen durchgeführt. Dies wurde sehr knapp (52\% zu 48\%) abgelehnt. Das Verkehrsreferat ist jedoch weiter im Gespräch mit dem VRN um ein passenderes Angebot für die Studierende zu finden.

Weitere Informationen zu den Verhandlungen und dem Semesterticket findest du unter: \url{http://www.stura.uni-heidelberg.de/semesterticket}

\subsection{Die Straßenbahn mitten durchs Feld?}

Jede von euch hatte bestimmt schon die Idee, dass es eigentlich gar nicht so schlecht wäre, wenn die Straßenbahn auch durchs Feld fahren würde. Bessere Verbindungen zum Hauptbahnhof und die Altstadt wären schon schön. Nicht immer das elend lange rumstehen in viel zu heißen Bussen an viel zu lang anhaltenden roten Ampeln. Nach kurzem Schwärmen fällt euch dann aber auch ein, dass das gar nicht so einfach ist. Es gibt viel zu wenig Platz an den Stellen, wo der Bus im Moment fährt. Dann sind da auch ganz schön viele hochkarätige Institute, Kliniken und andere Unternehmen, die mitentscheiden wollen, was vor ihren Türen passiert. Es wird also nicht einfach werden und man müsste mit viel Gegenwind rechnen.

So oder so ähnlich wird es sich wohl auch damals dem Heidelberger Oberbürgermeister gegangen sein. Darum lud er 2008 den Rektor der Universität, sowie hochrangige Vertreterinnen des Universitätsklinikums und des Deutschen Krebsforschungszentrums ein, um über seine Pläne zu reden.\footnote{Eine Pressemitteilung der Uni findet ihr unter \url{https://www.uni-heidelberg.de/presse/news08/pm280415-9str.html}} Damals wurde von Seiten der Universität verschiedene Anforderungen gestellt, auf welche wohl auch eingegangen wurde. Das Ganze ging also seinen Weg.

Fast forward to 2014. Die Fronten haben sich verhärtet. Es bedeutet beim Erfolg keiner Gruppe, dass der 3. Weltkrieg beginnt, jedoch bedeutet eine Entscheidung eine Verschlechterung der Umstände auf Jahre, egal welche Entscheidung getroffen wird. Die Universität, die Max-Planck-Gesellschaft zur Förderung der Wissenschaft e.V. und die Stiftung Deutsches Krebsforschungszentrum wehren sich vehement. Es wird zwar berichtet, dass die Gespräche zwischen den Fronten mit Mediatorin Theresia Bauer in ihrer Position als Wissenschaftsministerin von Baden-Württemberg gut verlaufen\footnote{\url{https://www.heidelberg.de/mobinetz,Lde/Start/Aktuelles/15_12_2014+Strassenbahn+im+Neuenheimer+Feld.html}}, jedoch sprechen Fakten eine andere Geschichte. Ende 2014 wird ein vorläufiger Baustopp durch der Verwaltungsgerichtshof Baden-Württemberg angeordnet.\footnote{\url{http://www.wh.uni-heidelberg.de/index.php/en/ville-de-montpellier/47-wohnheime/hohlbeinring/heimzeitung/ws-2014-15/290-strassenbahn-durch-das-neuenheimer-feld}}

Mehr als ein Jahr dauern die Gespräche um sich zu einigen und Anfang 2016 ist es soweit. Das Regierungspräsidium genehmigt alle Planänderungen.\footnote{\url{https://www.heidelberg.de/hd,Lde/29_01_2016+strassenbahn+ins+neuenheimer+feld_+regierungspraesidium+genehmigt+alle+planaenderungen.html}} Damit kann es schon fast weiter gehen, haben sich wahrscheinlich die meisten schon gedacht. And the end is near!

Am 11. Mai 2016 verkündet der Verwaltungsgerichtshof, dass der „\textit{Planfeststellungsbeschluss für die Straßenbahn Im Neuenheimer Feld aufgehoben}“ ist.\footnote{\url{http://www.vghmannheim.de/pb/,Lde/Heidelberg_Planfeststellungsbeschluss+fuer+Strassenbahn+Im+Neuenheimer+Feld+aufgehoben/}} Damit ist laut Oberbürgermeister Dr. Eckart Würzner „\textit{die Campus-Bahn gestorben}“\footnote{\url{https://www.heidelberg.de/hd,Lde/11_05_2016+stellungnahme+von+oberbuergermeister+dr_+eckart+wuerzner+zur+entscheidung+des+verwaltungsgerichtshofs.html}}.
