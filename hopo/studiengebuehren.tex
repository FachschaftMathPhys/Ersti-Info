% !TEX ROOT = ../ersti.tex
\section[Allgemeine Studiengebühren ade]{Allgemeine Studiengebühren \\ade!}

%\begin{figure*}[t]
%\centering
%\includegraphics[width=0.77\textwidth]{bilder/studiengebuehren.png}
%\end{figure*}

Seit dem Sommersemester 2012 sind die allgemeinen Studiengebühren, die im Sommersemester 2007 in ganz BaWü in Höhe von \EUR{500} eingeführt wurden, wieder abgeschafft. Es erfolgte ein vollständiger finanzieller Ausgleich in Höhe von \EUR{280} aus Landesmitteln, genannt “Qualitätssicherungsmittel” (QSM), oder liebevoll: “QuaSiMi”. Es sind nur \EUR{280} statt \EUR{500}, weil durch Ausnahmeregelungen wie die legendäre “Geschwisterregelung” seit 2009 im Durchschnitt nur dieser Betrag pro eingeschriebener Person an allgemeinen Studiengebühren eingenommen wurde.

Zum Wintersemester 2017/18 wurden Studiengebühren in BaWü wieder eingeführt. Diesmal nicht für alle, sondern nur für Studierende, die aus dem Nicht-EU-Ausland\footnote{also zum Beispiel aus Indien, China, USA, Lateinamerika} kommen und für Studierende im Zweitstudium, die also bereits einen akademischen Abschluss besitzen. Die grüne Wissenschaftsministerin Theresia Bauer, die früher selbst für die Abschaffung allgemeiner Studiengebühren geworben hat, verspricht sich von diesen neuen Gebühren, Löcher im Haushalt zu stopfen. Die internationalen Studierenden müssen \EUR{1500} pro Semester bezahlen, wovon nur 20\% direkt an die Hochschulen fließen, somit auch nur dieser Anteil für eine Verbesserung der Studiensituation für internationale Studierende eingesetzt werden kann -- wenn dieses Geld nicht bereits durch die zusätzlich nötig gewordenen bürokratischen Instanzen aufgebraucht wird. Für ein Zweitstudium müssen Studierende \EUR{650} pro Semester zahlen. Dies schadet Gruppen, die ohnehin schon benachteiligt und in den Entscheidungsgremien unterrepräsentiert sind. Außerdem haben internationale Studierende schon genügend Hürden, da schrecken weitere finanzielle Hürden noch mehr von einem Studium an der Uni Heidelberg ab. Doch gerade von der interkulturellen Diversität durch unsere internationalen Studierenden profitiert der Hochschulstandort Heidelberg maßgeblich. Auch für Studierende im Zweitstudium werden durch die Gebühren zusätzliche Hürden -- neben der großen Überwindung eines Neustarts und der dadurch verlängerten Studienzeit -- geschaffen, wobei gerade Zweit-Studierende mit ihrem interdisziplinären Wissen gesucht sind. Momentan ist im Koalitionsvertrag zwischen den Grünen und der CDU festgeschrieben, dass keine allgemeinen Studiengebühren eingeführt werden. Aber dies kann sich bei den nächsten Landtagswahlen 2021 ändern und deshalb sollte das Thema Studiengebühren weiterhin kritisch im Blickfeld der Studierenden bleiben.

Die Qualitätssicherungsmittel als Ersatz für die entfallenen allgemeinen Studiengebühren sind nicht nur zusätzliche Landesmittel für die Hochschule, sondern zweckgebunden an die Verbesserung der Studienbedingungen. Außerdem wurde im Studien\-gebühren\-abschaffungs\-gesetz festgeschrieben, dass die QuaSiMi nur im Einvernehmen mit einer legitimierten Vertretung der Studierenden ausgegeben werden dürfen. Das heißt, keine Ausgabe kann gegen den Willen der Studis durchgesetzt werden. Das ist immerhin ein Schritt in Richtung Fairness und Gleichberechtigung.

Mittlerweile hat sich die Vergabe der Gelder an der Universität Heidelberg wieder komplett geändert. Es gehen nun nur noch 12\% über das Vorschlagsrecht der Studierenden in Studium und Lehre. Der Rest des Geldes geht in die Grundmittel der Universität und unterliegen direkt dem Rektor. Diese Mittel sind nun nicht mehr zweckgebunden und der Rektor kann diese so verteilen, wie er es für sinnvoll hält. 

\subsection{Verwendung der studentischen QSM}
Die Fachschaften und die Fachschaftsräte als ihre exekutiven Organe haben das alleinige Vorschlagsrecht über die studentischen QSM, welche sich für unsere Fakultäten auf jeweils ca. \EUR{100000} belaufen. Das Schreiben der entsprechenden Anträge stellt daher einen zentralen Bestandteil der Fachschaftsarbeit dar. In den letzten Semester hat die Fachschaft mittels QSM die Investition in folgende Arten von Projekten erwirkt:

\vspace{5mm}
\textbf{Mathematik und Informatik}
\begin{itemize}
\item \textbf{Hilfskraftmitte:l}\\ Tutorinnen und zusätzliche Übungsgruppen
\item \textbf{Ausstattung:}\\ Computerpools, Hörsäle, Seminarräume
\item \textbf{Materialien:}\\ Softwarelizenzen, Skripte, Bücher, eBooks\footnote{\url{http://www.ub.uni-heidelberg.de/helios/epubl/eb/Welcome.html}}, Zeitschriften
\end{itemize}

\vspace{5mm}
\textbf{Physik}
\begin{itemize}
\item \textbf{\gls{AP} + \gls{FP}-Versuche}\\
	Modernisierung bestehender Versuche, Einrichtung neuer Versuche
\item \textbf{Hilfskraftmittel}\\
	Tutorinnen und zusätzliche Übungsgruppen
\item \textbf{Materialien}\\
	Skripte, E-Learning-Angebote (Mampf)
\item \textbf{Exkursionen}\\
	CERN, Paul Scherrer Institut\dots
\end{itemize}

Im Allgemeinen sind wir immer offen für neue und konstruktive Ideen von allen Seiten. Falls ihr also Ideen habt oder euch im Verlauf eures Studiums Dinge auffallen, die in der Lehre verbessert werden könnten und sollten, dürft ihr euch gerne melden und bei der Ausgestaltung mithelfen.

\subsection{Gebührenfreiheit?}
Auch wenn ihr nicht von den aktuellen Studiengebühren betroffen seid, müsst ihr weiterhin als Teil der \EUR{\beitragssumme}, die ihr jedes Semester an die Uni überweisen müsst, einen „Verwaltungskostenbeitrag“ von \EUR{\verwaltungsbetrag} bezahlen. Dieser wird den Hochschulen auf ihren Etat angerechnet, sodass das Geld de facto als Ersparnis ans Land Baden-Württemberg geht. Weil das Land nach der nächsten Wahl die allgemeinen Studiengebühren als Mittel gegen die Anhebung des Hochschuletats wieder entdecken könnte, sollte man das Thema Studiengebühren nicht einfach als Teil der hochschulpolitischen Geschichte abhaken. Was gute und vielleicht weniger gute Argumente gegen Studiengebühren sind, führt ein zeitloser Artikel der (ehemaligen) Studierendenzeitung UNiMUT, „Das Richtige im Falschen (und das Falsche im Richtigen)“  \footnote{\url{http://unimut.fsk.uni-heidelberg.de/unimut/aktuell/1107377557}}, aus, auf den wir an dieser Stelle verweisen möchten.




