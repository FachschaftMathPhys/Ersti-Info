% !TEX ROOT = ../ersti.tex
%
% TODO: stefanzen

\section{Allgemeine Studiengebühren ade!}
%\mathphyssecnobar{Studiengebühren ade!}%FIXME typo
Seit dem Sommersemester 2012 sind die allgemeinen Studiengebühren, die im Sommersemester 2007 in ganz BaWü in Höhe von \EUR{500} eingeführt wurden, wieder abgeschafft. Es erfolgte ein vollständiger finanzieller Ausgleich in Höhe von \EUR{280} aus Landesmitteln, genannt "Qualitätssicherungsmittel" (QSM), oder liebevoll: "QuaSiMi". Es sind nur \EUR{280} statt \EUR{500}, weil durch Ausnahmeregelungen wie die legendäre "Geschwisterregelung" seit 2009 im Durchschnitt nur dieser Betrag pro eingeschriebener Person an allgemeinen Studiengebühren eingenommen wurde.

Zum Wintersemester 2017/18 werden Studiengebühren in BaWü wieder eingeführt. Diesmal nicht für alle, sondern nur für Studierende, die aus dem Nicht-EU-Ausland (also zum Beispiel aus Indien, China, USA, Lateinamerika) kommen und für Studierende im Zweitstudium, die also bereits einen akademischen Abschluss besitzen. Die grüne Wissenschaftsministerin Theresia Bauer, die früher selber für die Abschaffung allgemeiner Studiengebühren geworben hat, verspricht sich von diesen neuen Gebühren, Löcher im Haushalt zu stopfen. Die internationalen Studierenden sollen \EUR{1500} pro Semester bezahlen, wovon nur 20\% direkt an die Hochschulen fließen, somit auch nur dieser Anteil für eine Verbesserung der Studiensituation für internationale Studierende eingesetzt werden kann. Für ein Zweitstudium sollen Studierende \EUR{650} pro Semester zahlen. Dies schadet Gruppen, die ohnehin schon benachteiligt und in den Entscheidungsgremien unterrepräsentiert sind. Außerdem haben internationale Studierende schon genügend Hürden, da schrecken weitere finanzielle Hürden noch mehr von einem Studium an der Uni Heidelberg ab. Doch gerade von der interkulturellen Diversität unserer internationalen Studierenden profitiert der Hochschulstandort Heidelberg maßgeblich. Auch für Studierende im Zweitstudium werden durch die Gebühren zusätzliche Hürden - neben der großen Überwindung eines Neustarts und der dadurch verlängerten Studienzeit - geschaffen, wobei gerade Zweit-Studierende mit ihrem interdisziplinären Wissen gerade gesucht sind. Momentan ist im Koalitionsvertrag zwischen den Grünen und der CDU festgeschrieben, dass keine allgemeine Studiengebühren eingeführt werden. Aber dies kann sich bei den nächsten Landtagswahen 2021 ändern und deshalb sollte das Thema Studiengebühren weiterhin kritisch im Blickfeld der Studierenden bleiben.

Die QuaSiMi sind nicht nur zusätzliche Landesmittel für die Hochschule, sondern auch weiterhin zweckgebunden für die Verbesserung der Studienbedingungen. Außerdem wurde im Gesetz \footnote{StuGebAbschG vom 21.12.2011 \url{http://www.landesrecht-bw.de/jportal/portal/t/b11/page/bsbawueprod.psml/screen/JWPDFScreen/filename/StuGebAbschG\_BW\_jlr-StuGebAbschGBWrahmen.pdf}} festgeschrieben, dass die QuaSiMi nur im Einvernehmen mit einer legitimierten Vertretung der Studierenden ausgegeben werden dürfen. Das heißt, keine Ausgabe kann gegen den Willen der Studis durchgesetzt werden. Das ist immerhin ein Schritt in Richtung Fairness und Gleichberechtigung.

Mittlerweile sind diese Mittel zum Großteil in die Grundfinanzierung der Universität geflossen. Nur ca. 11\% der bisherigen Mittel werden noch durch Vorschläge ausgegeben, welche jedoch alleinig von den Studierenden bestimmt werden. Das restliche Geld wird teilweise weiter in die Grundfinanzierung der Fakultäten gesteckt, jedoch nicht mehr in der Höhe wie die QuaSiMi bisher, vor allem in die Lehre, floss.

\subsection*{Gebührenfreiheit?}
Weiterhin müsst ihr jedoch als Teil der \EUR{\beitragssumme}, die ihr jedes Semester an die Uni überweisen müsst, einen "Verwaltungskostenbeitrag" von \EUR{\verwaltungsbetrag} bezahlen. Dieser wird den Hochschulen auf ihren Etat angerechnet, sodass das Geld de facto als Ersparnis ans Land Baden-Württemberg geht. Weil das Land nach der nächsten Wahl die allgemeinen Studiengebühren als Mittel gegen die erste Anhebung des Hochschuletats seit 1996 wieder entdecken könnte, möchten wir das Thema Studiengebühren nicht einfach als Teil der hochschulpolitischen Geschichte abhaken, sondern in diesem Kapitel daran erinnern. Wir laden Euch darüber hinaus ein, Euch über die Hintergründe der Studiengebühren zu informieren, beispielsweise auf unserer Fachschaftshomepage \footnote{\url{https://mathphys.fsk.uni-heidelberg.de/w/hintergruende/historie-der-studiengebuehren}}.

\subsection*{Ein paar Argumente gegen allgemeine Studiengebühren}

Wie ihr sicher bemerkt habt, gehen wir davon aus, dass es gut sei, wenn es keine Studiengebühren gibt. Was veranlasst uns dazu, mal abgesehen von persönlicher Betroffenheit?
\begin{itemize}
\item {Barrieren zwischen Abi und Studium einreißen!}\\
	Ein Studium ist verglichen mit einer Ausbildung sehr teuer. Mit allgemeinen Studiengebühren wäre es nochmals knapp \EUR{100}/Monat teurer. Das schreckt nachweislich insbesondere Abiturientinnen ab, deren Eltern selbst nicht studiert haben. Außerdem schreckt es prozentual mehr Mädchen als Jungs ab. Kurzum: Menschen, deren Umgebung von ihnen traditionell weniger erwartet dass sie studieren, studieren weniger; gesellschaftliche Ungleichheit, Stratifizierung und überkommene Rollenbilder werden fortgeschrieben.
\item {Aufhebung von finanziell bedingten Nachteilen im Studium!}\\
	Studis, die knapp über die BAföG-Grenze fallen, sind am härtesten von Studiengebühren betroffen. Denn bei ihnen reicht das Geld der Eltern oft nicht, um Lebensunterhalt und Gebühren zu finanzieren. Sie müssen also jobben gehen und haben somit deutlich weniger Zeit, sich um ihr eigentliches Studium zu kümmern. Gerade in Zeiten der Master"=Zulassungsbeschränkung führt das im Endeffekt zu finanzieller Auslese.
\item {Bildung ist öffentliches Gut!}\\
	Die Einführung der Studiengebühren wurde von eingefrorenen Haushalten und massiven Kürzungen im Bildungsbereich seitens der Landesregierungen flankiert. Studiengebühren sind keine Notwendigkeit der Lage, sondern Politik. Unserer Auffassung nach ist Bildung jedoch eins der wichtigsten Güter einer Gesellschaft, und das meinen wir nicht im rein ökonomischen Sinn. Gerade die universitäre Bildung bietet (noch, vergleichsweise) viele Freiräume zum eigenständigen Denken. Dessen Bedeutung ist für eine Gesellschaft, die sich demokratisch nennen möchte, nicht zu unterschätzen. Deshalb sollte Bildung in den Landeshaushalt höchste Priorität haben.
\end{itemize}
%\vfill
%\begin{figure}[h]
%\centering{
%    \includegraphics[width=0.7\textwidth]{bilder/studiengebuehren.png}
%}
%\end{figure}
%\vfill

\subsection*{Verteilung der QuaSiMi an der Universität Heidelberg}

Die Studierenden in Heidelberg setzten durch, dass die Studiengebühren ausschließlich zur Verbesserung der Lehre verwendet werden durften. Um ausreichend Präsenz in den Diskussionen zu haben, wurde auch durchgesetzt dass Studierende in den früheren Kommissionen zur Verteilung der Gelder die absolute Mehrheit haben. 

Mittlerweile hat sich die Vergabe der Gelder wieder komplett geändert. Es gehen nun nur noch 11,764\% über Vorschlagsrecht der Studierenden in Studium und Lehre. Der Rest des Geldes geht in die Grundmittel der Universität und unterliegen direkt dem Rektor. Diese Mittel sind nun nicht mehr Zweckgebunden und der Rektor kann diese so verteilen, wie er es für sinnvoll hält. 


\subsection*{Verwendung der QuaSiMi}

Da der Gesamtbetrag des Geldes zentral an die VS geht, musste diese eine Vergabeverordnung\footnote{\url{https://www.stura.uni-heidelberg.de/fileadmin/Dokumente/VS/QSM-Ordnung.pdf}} erstellen und beschließen. In dieser sind Deadlines, bis wann die Vorschläge der einzelnen Fachbereiche eingegangen sein müssen, die Verteilung der Gelder und die Verwendung falls Gelder nicht komplett verplant werden, geregelt. Verwendungszwecke bzw. Sachen für was Geld ausgegeben werden darf, ist geregelt durch das Wissenschaftsministerium.

Die größten Defizite im Bereich Lehre waren an unseren Fakultäten das schlechte Betreuungsverhältnis, vor allem in den Übungsgruppen und Seminaren, sowie ein ungenügendes Serviceangebot. Um diese Defizite auszugleichen wurde und wird mit den Sondermitteln aus Studiengebühren und QuaSiMi insbesondere in folgende Bereiche investiert:

\vspace{5mm}
\textbf{Mathematik und Informatik}
\begin{itemize}
 \item {Mitarbeiterstellen}\\Lehr-/Servicepersonal, Lehraufträge, Assistentinnen
\item {Hilfskraftmittel}\\ Tutorinnen und zusätzliche Übungsgruppen
\item {Ausstattung}\\ Computerpools, Hörsäle, Seminarräume
\item {Materialien}\\ Softwarelizenzen, Skripte, Bücher, eBooks\footnote{\url{http://www.ub.uni-heidelberg.de/helios/epubl/eb/Welcome.html}}, Zeitschriften
\end{itemize}

Weiter wurden Mittel zur Modernisierung der mangelhaften Ausstattung der Hörsäle, Computerpools, sowie der Bibliothek investiert.

\vspace{5mm}
\textbf{Physik}
\begin{itemize}
\item {Mittelbaustellen}\\
	Medizinerausbildung, Studienberatung, Lehramtsausbildung, Aufbau des IUP (Glossar) Praktikums
\item {AP + FP Versuche (Glossar)}\\
	Modernisierung bestehender Versuche, neue Praktika im FP und dem IUP
\item {Öffnungszeiten}\\
	Erweiterung der Öffnungszeiten in der \gls{KIP}-Bibliothek sowie dem Studierendensekretariat
\item {Hilfskraftmittel}\\
	Tutorinnen und zusätzliche Übungsgruppen
\item {Materialien}\\
	Skripte, Bücher, eBooks\footnote{\url{http://www.ub.uni-heidelberg.de/helios/epubl/eb/Welcome.html}}, Zeitschriften
\end{itemize}

Da experimentelle Erfahrung enorm wichtig für einen Physiker ist, wurde und wird viel Geld in aktuellste Versuche investiert. Eine Bereicherung stellen sicher auch die Exkursionen nach vielen Kursvorlesungen dar (CERN, GSI\dots).

Im Allgemeinen sind wir immer offen für neue und konstruktive Idee von allen Seiten. Falls ihr also Ideen habt oder euch im Verlauf eures Studiums Dinge auffallen, die in der Lehre verbessert werden könnten und sollten, dürft ihr euch gerne melden und bei der Ausgestaltung mithelfen.
