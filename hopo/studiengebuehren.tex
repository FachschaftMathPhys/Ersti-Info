% !TEX ROOT = ../ersti.tex
%
% TODO: stefanzen

%\section{Studiengebühren ade!}
\newpage
\mathphyssecnobar{Studiengebühren ade!}%FIXME
Seit dem Sommersemester 2012 sind die Studiengebühren, die im Sommersemester
2007 in ganz BaWü in Höhe von \EUR{500} eingeführt wurden, wieder abgeschafft.
Es erfolgte ein vollständiger finanzieller Ausgleich in Höhe von \EUR{280} aus
Landesmitteln, genannt "Qualitätssicherungsmittel", oder liebevoll: "QuaSiMi".
Es sind nur \EUR{280} statt \EUR{500}, weil durch Ausnahmeregelungen wie die
legendäre "Geschwisterregelung" seit 2009 im Durchschnitt nur dieser Betrag pro
eingeschriebener Person an Studiengebühren eingenommen wurde.

Die QuaSiMi sind nicht nur zusätzliche Landesmittel für die Hochschule, sondern
auch weiterhin zweckgebunden für die Verbesserung der Studienbedingungen.
Außerdem wurde im Gesetz \footnote{StuGebAbschG vom 21.12.2011
\url{http://mwk.baden-wuerttemberg.de/fileadmin/pdf/gesetze/StuGebAbschG/GBl-2011_565_Studiengeb\%C3\%BChrenabschaffungsgesetz_21122011.pdf}}
festgeschrieben, dass die QuaSiMi nur im Einvernehmen mit einer legitimierten
Vertretung der Studierenden ausgegeben werden dürfen. Das heißt, keine Ausgabe
kann gegen den Willen der Studis durchgesetzt werden. Das ist immerhin ein
Schritt in Richtung Fairness und Gleichberechtigung.

Mittlerweile sind diese Mittel zum Großteil in die Grundfinanzierung der
Universität geflossen. Nur ca. 11\% der bisherigen Mittel werden noch durch
Vorschläge ausgegeben, welche jedoch alleinig von den Studierenden bestimmt
werden. Das restliche Geld wird teilweise weiter in die Grundfinanzierung der
Fakultäten gesteckt, jedoch nicht mehr in der Höhe wie die QuaSiMi bisher,
vorallem in die Lehre, floss.

\subsection*{Gebührenfreiheit?}
Weiterhin müsst ihr jedoch als Teil der \EUR{\beitragssumme}, die ihr jedes
Semester an die Uni überweisen müsst, einen "Verwaltungskostenbeitrag" von
\EUR{\verwaltungsbetrag} bezahlen. Dieser wird den Hochschulen auf ihren Etat
angerechnet, sodass das Geld de facto als Ersparnis ans Land Baden-Württemberg
geht. Weil das Land nach der nächsten Wahl die Studiengebühren als Mittel gegen
die erste Anhebung des Hochschuletats seit 1996 wiederentdecken könnte, möchten
wir das Thema Studiengebühren nicht einfach als Teil der hochschulpolitischen
Geschichte abhaken, sondern in diesem Kapitel daran erinnern.
%Dieser Absatz sollte dann rein, wenn unten der historische Exkurs nicht hineinkommt!
\iffalse
Wir laden Euch darüberhinaus ein, Euch über die Hintergründe der
Studiengebühren zu informieren, beispielsweise auf unserer Fachschaftshomepage
\footnote{http://mathphys.fsk.uni-heidelberg.de/studgeb-historie.html}.
\fi
%

\subsection*{Ein paar Argumente gegen Studiengebühren}

Wie ihr sicher bemerkt habt, gehen wir davon aus, dass es gut sei, wenn es
keine Studiengebühren gibt. Was veranlasst uns dazu, mal abgesehen von
persönlicher Betroffenheit?
\begin{itemize}
\item {Barrieren zwischen Abi und Studium einreißen!}\\Ein Studium ist
    verglichen mit einer Ausbildung sehr teuer. Mit Studiengebühren wäre es
    nochmals knapp \EUR{100}/Monat teurer. Das schreckt nachweislich insbesondere
    AbiturientInnen ab, deren Eltern selbst nicht studiert haben. Außerdem schreckt
    es prozentual mehr Mädchen als Jungs ab. Kurzum: Menschen, deren Umgebung von
    ihnen traditionell weniger erwartet dass sie studieren, studieren weniger;
    gesellschaftliche Ungleichheit, Stratifizierung und überkommene Rollenbilder
    werden fortgeschrieben.
\item {Aufhebung von finanziell bedingten Nachteilen im Studium!}\\Studis, die
    knapp über die BAföG-Grenze fallen, sind am härtesten von Studiengebühren
    betroffen. Denn bei ihnen reicht das Geld der Eltern oft nicht, um
    Lebensunterhalt und Gebühren zu finanzieren. Sie müssen also jobben gehen und
    haben somit deutlich weniger Zeit, sich um ihr eigentliches Studium zu kümmern.
    Gerade in Zeiten der Master"=Zulassungsbeschränkung führt das im Endeffekt zu
    finanzieller Auslese.
\item {Bildung ist öffentliches Gut!}\\Die Einführung der Studiengebühren wurde
    von eingefrorenen Haushälten und massiven Kürzungen im Bildungsbereich seitens
    der Landesregierungen flankiert. Studiengebühren sind keine Notwendigkeit der
    Lage, sondern Politik. Unserer Auffassung nach ist Bildung jedoch eins der
    wichtigsten Güter einer Gesellschaft, und das meinen wir nicht im rein
    ökonomischen Sinn. Gerade die universitäre Bildung bietet (noch,
    vergleichsweise) viele Freiräume zum eigenständigen Denken. Dessen Bedeutung
    ist für eine Gesellschaft, die sich demokratisch nennen möchte, nicht zu
    unterschätzen. Deshalb sollte Bildung in den Landeshaushalten höchste Priorität
    haben.
\end{itemize}
%\vfill
%\begin{figure}[h]
%\centering{
%    \includegraphics[width=0.7\textwidth]{bilder/studiengebuehren.png}
%}
%\end{figure}
%\vfill

\subsection*{Verteilung der QuaSiMi an der Universität Heidelberg}

Die Studierenden in Heidelberg setzten durch, dass die Studiengebühren
ausschließlich zur Verbesserung der Lehre verwendet werden durften. Um
ausreichend Präsenz in den Diskussionen zu haben, wurde auch durchgesetzt dass
Studierende in den früheren Kommissionen zur Verteilung der Gelder die absolute Mehrheit
haben. 

Mittlerweile hat sich die Vergabe der Gelder wieder komplett geändert. Es gehen
nun nur noch 11,764\% über Vorschlagsrecht der Studierenden in Studium und
Lehre. Der Rest des Geldes geht in die Grundmittel der Universität und
unterliegen direkt dem Rektor. Diese Mittel sind nun nicht mehr Zweckgebunden
und der Rektor kann diese so verteilen, wie er es für sinnvoll hält. 


\subsection*{Verwendung der QuaSiMi}

Da der Gesamtbetrag des Geldes zentral an die VS geht, musste diese eine
Vergabeverordung\footnote{\url{https://www.stura.uni-heidelberg.de/fileadmin/Dokumente/VS/QSM-Ordnung.pdf}}
erstellen und beschließen. In dieser sind Deadlines, bis wann die Vorschläge
der einzelnen Fachbereiche eingegangen sein müssen, die Verteilung der Gelder
und die Verwendung falls Gelder nicht komplett verplant werden, geregelt.
Verwendungszwecke bzw. Sachen für was Geld ausgegeben werden darf, ist geregelt
durch das Wissenschaftsministerium.

Die größten Defizite im Bereich Lehre waren an unseren Fakultäten das schlechte
Betreuungsverhältnis, vor allem in den Übungsgruppen und Seminaren, sowie ein
ungenügendes Serviceangebot. Um diese Defizite auszugleichen wurde und wird mit
den Sondermitteln aus Studiengebühren und QuaSiMi insbesondere in folgende
Bereiche investiert:

\vspace{5mm}
\textbf{Mathematik und Informatik}
\begin{itemize}
 \item {Mitarbeiterstellen}\\Lehr-/Servicepersonal, Lehraufträge, AssistentInnen
\item {Hilfskraftmittel}\\ TutorInnen und zusätzliche Übungsgruppen
\item {Ausstattung}\\ Computerpools, Hörsäle, Seminarräume
\item {Materialien}\\ Softwarelizenzen, Skripte, Bücher, eBooks\footnote{\url{http://www.ub.uni-heidelberg.de/helios/epubl/eb/Welcome.html}}, Zeitschriften
\end{itemize}

Weiter wurden Mittel zur Modernisierung der mangelhaften Ausstattung der Hörsäle,
Computerpools, sowie der Bibliothek investiert.

\vspace{5mm}
\textbf{Physik}
\begin{itemize}
\item {Mittelbaustellen}\\Medizinerausbildung, Studienberatung, Lehramtsausbildung, Aufbau des IUP (Glossar) Praktikums
\item {AP + FP Versuche (Glossar)}\\Modernisierung bestehender Versuche, neue
Praktika im FP und dem IUP
\item {Öffnungszeiten}\\Erweiterung der Öffnungszeiten in der \gls{KIP}-Bibliothek sowie dem Studierendensekretariat
\item {Hilfskraftmittel}\\ TutorInnen und zusätzliche Übungsgruppen
\item {Materialien}\\Skripte, Bücher, eBooks\footnote{\url{http://www.ub.uni-heidelberg.de/helios/epubl/eb/Welcome.html}}, Zeitschriften
\end{itemize}

Da experimentelle Erfahrung enorm wichtig für einen Physiker ist, wurde und
wird viel Geld in aktuellste Versuche investiert. Eine Bereicherung stellen
sicher auch die Exkursionen nach vielen Kursvorlesungen dar (CERN, GSI\dots).

Im Allgemeinen sind wir immer offen für neue und konstruktive Idee von allen
Seiten. Falls ihr also Ideen habt oder euch im Verlauf eures Studiums Dinge
auffallen, die in der Lehre verbessert werden könnten und sollten, dürft ihr
euch gerne melden und bei der Ausgestaltung mithelfen.

%VON 2008...
%Eine genauere aktuelle Auflistung findet ihr im aktuellen MathPhys-Info
%(siehe FS-Homepage\footnote{\url{http://mathphys.fsk.uni-heidelberg.de/mpi.html}}).

%Wenn das Erstiinfo zu fett wird, kann man das rausnehmen      ––– wird gemacht ;)
\iffalse
\subsection*{Historischer Exkurs: Geschichte der Studiengebühren -- Geschichte der Sparprogramme}

Im Wintersemester 1970/71 wurden in der Bundesrepublik Deutschland die
allgemeinen Studiengebühren, damals "Hörergeld", mit Hilfe von Protesten und
Boykotten abgeschafft.

Aber immer wieder gab es Bewegungen in der Politik, die Studiengebühren
forderten. Das ging meistens mit einem Bild von Universität einher, das diese
nicht als öffentliche Einrichtung zur Verbreitung des gesellschaftlichen Guts
Bildung, sondern als Anbieterin von Leistungen auf dem Markt der Ware Bildung
auffasst. Entsprechend wenig ist solche Politik bereit, Bildungseinrichtungen
eine angemessene finanzielle Ausstattung zukommen zu lassen.

Es wurden Rückmeldegebühren eingeführt um die Haushaltslöcher der Hochschulen
zu stopfen. Die Hochschulen mussten in BaWü Anfang der 1990er arge Kürzungen im
Haushalt hinnehmen, der 1996 für 10 Jahre eingefroren wurde (bis auf
Baumaßnahmen). Diese geplante finanzielle Austrocknung der Hochschulen wurde
übrigens 2006 um weitere 10 Jahre verlängert. Gleichzeitig wurde damals laut
darüber nachgedacht, Studiengebühren für Langzeitstudierende einzuführen. Diese
wurden in verschiedenen Ländern auch durchgesetzt (in BaWü zum WS 1998/99).

2002 wurde von der Kultusministerkonferenz ein allgemeines
Studiengebührenverbot festgeschrieben. Langzeitstudiengebühren waren in
bestimmten Ausnahmefällen erlaubt.

Gegen dieses Verbot klagten einige Bundesländer vorm Bundesverfassungsgericht,
darunter BaWü. Sie führten an, dass der Bund seine Gesetzgebungskompetenz
überschritten und in die Länderkompetenz eingegriffen habe.

Am 26.01.2005 fällte das Bundesverfassungsgericht das Urteil, dass ein Verbot
allgemeiner Studiengebühren nur gerechtfertigt sei, um „gleichwertige
Lebensbedingungen“ in den Ländern zu wahren. In diesem Fall sei aber kein
Anlass zu solcher Sorge gegeben. Somit wurde das Gesetz von 2002 gekippt und
der Weg für die allgemeinen Studiengebühren geebnet.

Schon im Dezember 2005 wurde in Baden"=Württemberg Minister Frankenbergs
Gesetzentwurf zur Einführung allgemeiner Studiengebühren von der
Landesregierung beschlossen. Studiengebühren in Höhe von \EUR{500} pro Semester
wurden zum Sommersemester 2007 eingeführt. Es gab weiter Proteste gegen die
Einführung, Demos und Boykotts wurden organisiert.

Der AK Studiengebühren der \gls{FSK} organisierte einen Boykott und es wurden
Klagen beim Verwaltungsgericht eingereicht. Wie an vielen anderen Universitäten
in Baden"=Württemberg wurden die Studierenden in Heidelberg von der
studentischen Vollversammlung dazu aufgefordert, die \EUR{500} nicht an die
Universität, sondern auf ein Treuhandkonto zu überweisen. Man hätte dann unter
Bezug auf das zurückgehaltene Geld als Druckmittel mit der Landesregierung
Gespräche begonnen. Leider wurde das beschlossene Quorum mit nur ca. 1200 statt
4500 eingegangenen Zahlungen nicht erreicht und der Boykott daher nicht
durchgeführt. Die Studiengebühren konnten damit nicht abgeschafft werden.

Auch im Bildungsstreik, der 2009-2011 die Bildungspolitik aufrüttelte, war die
Forderung nach Abschaffung der Studiengebühren (als einer Bildungsgebühr von
vielen) stets zentral. Darum organisierten viele baden"=württembergische
Bildungsstreik-Gruppen, auch in Heidelberg, nach den Erfahrungen aus Hessen und
NRW im Hinblick auf die Landtagswahl nochmals Demonstrationen im Januar 2011.
Von SPD über Grüne bis Linkspartei fand sich im Wahlprogramm zur Landtagswahl
2011 denn auch die Absichtserklärung wieder, die Studiengebühren abzuschaffen.

Zum Sommersemester 2012 hat nun tatsächlich die grün-rote Landesregierung ihr
Wahlversprechen eingelöst: Die Studiengebühren wurden, bei voller Kompensation
aus Landesmitteln, abgeschafft!

%Von Gebühren befreit bist du auch schon in diesem Semester, wenn du ein Kind
%unter 14 Jahren hast, ein Praxissemester absolvierst, ein Stipendium erhältst,
%du mindestens zwei (auch Stief"=/Halb"=) Geschwister hast die nicht diese
%baden"=württembergische Geschwisterregelung in Anspruch nehmen, oder du unter
%einer studienerschwerenden Behinderung leidest. Eine Befreiung aufgrund
%außerordentlicher Studienleistungen ist gesetzlich möglich, wird aber von den
%Fakultäten für Mathematik und Physik abgelehnt und darum nicht gewährt.

Für die fehlenden Gelder gab es auch Ausgleichszahlungen vom Land, die
sogenannenten Qualitätssicherungsmittel, von denen du schon weiter vorne in
diesem Heft erfahren hast.
\fi
