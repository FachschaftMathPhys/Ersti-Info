\section{Ein paar Worte zu Corona \ldots}

Für euch kommt in diesem Semester gleich zu Beginn noch eine Schwierigkeit hinzu, die ihr alle schon kennt und die ihr vielleicht auch fürchtet: Corona.
Und auch wenn für euch erst einmal alles neu ist -- wie bei all den Erstis vor euch auch -- und dann doch alles anders läuft -- wie vor euch noch bei keiner -- ist es für euch ebenso wichtig, am Anfang möglichst viele Kommilitoninnen kennenzulernen. Denn auch für euch gilt: Nur gemeinsam sind wir stark! Es kommen viele Herausforderungen auf euch zu, die gemeinsam leichter zu bewältigen sind und zudem vielleicht sogar Spaß machen können.

Zu Beginn sei dafür das berühmt-berüchtigte Zettelrechnen erwähnt, das auch ihr kennenlernen werdet. Jede Woche werdet ihr pro Veranstaltung ein Übungsblatt bearbeiten und abgeben. Alleine verzweifelt man da schnell und braucht sehr viel Zeit, gemeinsam kommt man schneller ans Ziel und wenn es auch manchmal vielleicht sehr knifflig ist, haben mehr Köpfe meistens auch mehr gute Ideen. Sich gegenseitig zu unterstützen und gemeinsam zu lernen gehört zum Studium einfach dazu. Es ist also wichtig, gleich zu Anfang Kontakte zu knüpfen –- aber wie funktioniert das? Hier bekommt ihr einen schwierigeren Start als die Jahrgänge vor euch. Trotzdem gibt es auch für euch Möglichkeiten.

Nutzt unbedingt das Rahmenprogramm des Vorkurses, um Kommilitoninnen zu treffen. Auch wenn bei euch einiges leider nicht stattfinden kann, so bieten sich doch Möglichkeiten. Im Vorkurs organisieren wir euch falls möglich, Wanderungen und Spieleabende in Präsenz.
Eure Tutorien werden, sofern irgendwie möglich, in Kleingruppen stattfinden, in der Physik werden \emph{Bubbles} gebildet, feste Gruppen, in denen ihr euch regelmäßig trefft.
Und habt ihr erst einmal nette Menschen getroffen, können wir euch nur empfehlen, mit den anderen etwas zu machen –- auch wenn im Winter das Outdoorprogramm sehr eingeschränkt ist. Trefft euch trotzdem, zum Kochen, (kleine) Spieleabende oder online, zum Telefonieren, online spielen (z.\,B.\,Jackbox, \ldots) oder einfach nur vor der Kamera Abendessen, damit ihr euch nicht so alleine fühlt. Denn das solltet ihr im Studium nicht und müsst es auch nicht.

Wir drücken euch für euren Studienbeginn trotz der widrigen Randbedingungen die Daumen, hoffen, ihr könnt euer Studium trotzdem genießen und wünschen einen möglichst guten Semesterstart!