\section{Digitale Tools im Corona-Alltag}
Da im Wintersemester ein Großteil der Lehre vermutlich online stattfinden wird, geben wir euch hier eine Übersicht, welches Tool wofür verwendet wird. Teilweise sind die Plattformen von der Universität selber gehostet, teilweise sind es externe Anbieter. Dementsprechend sind sie manchmal frei zugänglich oder ihr benötigt einen Account, meist euren Uni-Account. Tools, die als Funktion Onlinevorlesungen aufweisen, wurden im vergangenen Semester verwendet, es ist also möglich, dass in diesem Semester auch andere Tools verwendet werden. Welche Tools eure Professorinnen und Tutorinnen letztendlich verwenden werden und ob die Veranstaltungen synchron, das heißt bei allen Teilnehmerinnen zur selben Zeit, oder asynchron als on-demand Veranstaltung stattfindet, werden sie euch dann rechtzeitig mitteilen.

% \def\nl{\vspace{12pt}\\}
% \newcolumntype{L}[1]{>{\raggedright\arraybackslash}p{#1}} % linksbündig mit Breitenangabe
% \newcolumntype{C}[1]{>{\centering\arraybackslash}p{#1}} % zentriert mit Breitenangabe
% \newcolumntype{R}[1]{>{\raggedleft\arraybackslash}p{#1}} % rechtsbündig mit Breitenangabe

% \begin{table}[b!]
% 	\begin{tabular}{L{2cm} L{4cm} L{9cm}}
% 		\toprule
% 		\textbf{Tool} & \textbf{Was ist es?} &  \textbf{Wofür wird es verwendet?} \\ 
% 		\midrule
% 		\href{https://lsf.uni-heidelberg.de/}{Lehre-Studium-Forschung (LSF)} &  Campus-Management-System der~Uni & Alles, was mit Verwaltung zu tun hat: Vorlesungsverzeichnis, Informationen zu Personen, Gebäuden, Rückmeldung durchführen, Bescheinigungen ausdrucken. In anderen Studiengängen auch Anmeldungen für Veranstaltungen und Klausuren\nl
		
% 		\href{https://moodle.uni-heidelberg.de}{Moodle} & e-Learningplattform & Lernmaterialien, Zettelabgabe, Punkteeinsicht, Klausuranmeldung, organisatorische Informationen \nl

% 		\href{https://uebungen.physik.uni-heidelberg.de}{Übungs-gruppen-verwaltung} & Tool zum Organisieren von Übungsgruppen der Physik & Eintragen in Übungsgruppen, Anmeldung zu Veranstaltungen, Punkte für Übungszettel, Einsicht der Klausurnote \nl

% 		\href{https://muesli.mathi.uni-heidelberg.de}{MÜSLI} & Tool zum Organisieren von Übungsgruppen in der Mathe und Info & Eintragen in Übungsgruppen, Anmeldung zu Veranstaltungen, Punkte für Übungszettel, Klausuranmeldung, Einsicht der Klausurnote \nl

% 		\href{https://mampf.mathi.uni-heidelberg.de}{MaMpf} & Mathematische Medienplattform & Vorlesungsvideos, Beispielvideos, Quizzes, weitere Lernangebote \nl

% 		\href{https://heiconf.uni-heidelberg.de}{heiCONF} & Uni-internes Videokonferenztool basierend auf BigBlueButton & Onlinevorlesungen, Seminare, Sprechstunden, Tutorien \nl

% 		Zoom & Videokonferenztool & Onlinevorlesungen, Seminare, Sprechstunden, Tutorien \nl

% 		Cisco Webex & Videokonferenztool & Onlinevorlesungen, Seminare, Sprechstunden, Tutorien \nl 
		
% 		Microsoft Teams & Videokonferenztool & Onlinevorlesungen, Seminare, Sprechstunden, Tutorien \nl 

% 		Jitsi & open-source Videokonferenztool & Onlinevorlesungen, Seminare, Sprechstunden, Tutorien \nl
	
% 		Twitch & Live-streaming-Videoportal & Onlinevorlesungen (in Kombination mit Chat-Tool)\nl

% 		Discord & Kommunikations-plattform & Help-Desk, Zettelgruppen \nl

% 		Rocket-Chat \& Matrix-IM & Instant Messaging & Kommunikation mit Kommilitoninnen und Dozentinnen, z.\,B.\,um Fragen zu stellen \nl
					
% 		\bottomrule
% 	\end{tabular}
	
% \end{table}

\begin{description}
	\item[\href{https://lsf.uni-heidelberg.de/}{Lehre-Studium-Forschung (LSF)}] Das Campus-Management-System der~Uni bietet alles, was mit Verwaltung zu tun hat: Vorlesungsverzeichnis, Informationen zu Personen, Gebäuden, Rückmeldung durchführen, Bescheinigungen ausdrucken. In anderen Studiengängen auch Anmeldungen für Veranstaltungen und Klausuren.
	\item[\href{https://moodle.uni-heidelberg.de}{Moodle}] Dies ist die e-Learningplattform der Uni, auf der Lernmaterialien, Zettelabgabe, Punkteeinsicht, Klausuranmeldung, organisatorische Informationen angeboten werden.
	\item[\href{https://uebungen.physik.uni-heidelberg.de}{Übungsgruppenverwaltung}] Mit dieser Plattform mit integriertem Rocket-Chat werden Übungsgruppen in der Physik organisiert. Hier kann man sich Eintragen in Übungsgruppen, Anmelden zu Veranstaltungen, Punkte für Übungszettel und die Klausurnote einsehen.
	\item[\href{https://muesli.mathi.uni-heidelberg.de}{MÜSLI}] In der Mathe und Info werden mit diesem System Übungsgruppen organisiert. Man kann sich in Übungsgruppen eintragen, zu Veranstaltungen und Klausuren anmelden und Punkte für Übungszettel und die Klausurnote einsehen.
	\item[\href{https://mampf.mathi.uni-heidelberg.de}{MaMpf}] Auf dieser Mathematischen Medienplattform gibt es Vorlesungsvideos, Beispielvideos, Quizzes, weitere hypermediale Lernangebote.
	\item[\href{https://heiconf.uni-heidelberg.de}{heiCONF}] Uni-internes Videokonferenztool basierend auf BigBlueButton. Es wird genutzt für Onlinevorlesungen, Seminare, Sprechstunden und Tutorien
	\item[Zoom] Videokonferenztool für Onlinevorlesungen, Seminare, Sprechstunden, Tutorien 
	\item[Cisco Webex] Videokonferenztool für Onlinevorlesungen, Seminare, Sprechstunden, Tutorien 
	\item[Microsoft Teams] Videokonferenztool für Onlinevorlesungen, Seminare, Sprechstunden, Tutorien
	\item[Jitsi] open-source Videokonferenztool für Onlinevorlesungen, Seminare, Sprechstunden, Tutorien ohne Anmeldung
	\item[Twitch] Live-streaming-Videoportal für Onlinevorlesungen (in Kombination mit Chat-Tools)
	\item[Discord] Kommunikationsplattform für Help-Desk und Zettelgruppen 
	\item[Matrix-IM] Uni-internes Instant Messaging Tools zur Kommunikation mit Kommilitoninnen und Dozentinnen, z.\,B.\,um Fragen zu stellen.
\end{description}
		 
			
Es lohnt sich dementsprechend vor Beginn der Vorlesungszeit die jeweiligen Clients zu installieren, damit während des Semesters ein reibungsloser Ablauf gewährleistet ist.

Folgende Hinweise sind als Etikette hilfreich für synchrone Veranstaltungen:
\begin{itemize}
	\item Tretet dem virtuellen Raum schon ein paar Minuten vorher bei, damit die Veranstaltung pünktlich beginnen kann.
	\item Schaltet das Video aus, wenn es nicht benötigt wird, um Bandbreite zu sparen.
	\item Benutzt möglichst ein Headset mit Mikrofon, um eine angenehme Gesprächsqualität für alle zu gewährleisten.
	\item Wenn ihr nicht sprecht, schaltet bitte euer Mikrofon stumm, so können störende Nebengeräusche wie Tastaturanschläge vermieden werden.
	\item Nutzt die Möglichkeiten der jeweiligen Plattformen für Interaktionen, z.\,B.\,Handbehen, Chats, Umfrageabstimmungen.
	\item Habt Verständnis für die Dozentinnen und Tutorinnen, welche sicherlich auch lieber Veranstaltungen in Präsenz anbieten würden, als sich mit euch im Neuland zu treffen.
\end{itemize}